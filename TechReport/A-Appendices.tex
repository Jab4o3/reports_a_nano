\begin{appendices}

\chapter{Code}\label{chap:appendix:code}

%\begin{figure}
%	\centering
%	\begin{subfigure}
%		\begin{lstlisting}[frame=single,
%			numbers=left,
%			style=Matlab-Pyglike]
%% requirements
%V_o_tia = .5; % out voltage of tia
%V_o = 5; % out voltage of non-inverting amp
%
%% OPA795 specs
%GBW = 1.5*10^6;
%A_o_dB = 100; % DC open-loop gain in dB
%A_o = 10^(A_o_dB/20); % DC open-loop gain
%f_o = GBW/A_o; % open-loop cutoff frequency
%
%% BPW34 specs
%I_f = 50*10^-9; % max current from diode
%C_d = 25*10^-12; % diode capacitance 
%
%% parasitic capacitances
%C_cm = 2.2*10^-12; % common-mode capacitance
%C_df = 2*10^-12; % differential capacitance
%C_i = C_d + C_cm + C_df; % total parasitic capacitance
%		\end{lstlisting}
%		\subcaption{Requirements and given values}
%		\label{code:tia:consts}
%	\end{subfigure}
%	\\
%	\begin{subfigure}
%		\begin{lstlisting}[frame=single,
%			numbers=left,
%			style=Matlab-Pyglike]
%% R1 calculations
%gain = V_o_tia/I_f;
%gain_db = 20*log10(gain); % uncomment if you need gain in dB
%R_1 = gain*(1 + A_o)/A_o;
%%R_1 = 10e6; % realistic value for BPW34
%
%% C1 calculation
%omega_o = 2*pi*f_o;
%C_1_num = -2*omega_o*R_1*C_i + 1 + sqrt(12*A_o*C_i*R_1*omega_o - 3);
%C_1_den = 2*omega_o*R_1*(1 + A_o);
%C_1 = C_1_num/C_1_den;
%
%% system transfer function
%divisor = (C_i + C_1)*R_1; % normalization coefficient
%num = A_o*omega_o/divisor; % numerator
%s1 = (1 + omega_o*R_1*(C_i + (1 + A_o)*C_1))/divisor; % first-order s
%s0 = omega_o*(1 + A_o)/divisor; % zero-order s
%den = [1, s1, s0];
%tia = tf(num, den);
%Q = sqrt(s0)/s1;
%
%% cutoff calculations
%omega_c = bandwidth(tia);
%f_c = omega_c/(2*pi);
%
%% noise transfer function
%num = A_o*omega_o*[1 1/divisor];
%s1 = 1/divisor + omega_o*(1 + A_o*C_1*(C_1 + C_i));
%s0 = omega_o*(1 + A_o)/divisor; % 0th oreder term remains the same
%den = [1 s1 s0];
%tia_noise = tf(num, den);
%tia_noise1 = tia_noise;
%		\end{lstlisting}
%		\subcaption{$R_1$, $C_1$, gain and noise calculations}
%		\label{code:tia:calcs}
%	\end{subfigure}
%	\\
%	\begin{subfigure}
%		\begin{lstlisting}[frame=single,
%			numbers=left,
%			style=Matlab-Pyglike]
%disp('Component values')
%disp('R1: ' + R_1/10^6 + ' MOhm')
%disp('C1: ' + C_1*10^12 + ' pF')
%disp('System characteristics')
%disp('fc = ' + f_c/10^3 + ' kHz = ' +  omega_c/10^3 + ' krad/s')
%disp('Q (actual): ' + Q)
%disp('Q (ideal) : ' + 1/sqrt(3) + ' = sqrt(3)/3')
%% bode plot system
%figure
%opts = bodeoptions;
%opts.FreqUnits = 'Hz';
%opts.Title.String = 'Frequency response';
%bodeplot(tia, {10, 10^6}, opts);
%% impulse plot system
%figure
%ip = impulseplot(tia);
%% bode plot noise
%figure
%opts_n = bodeoptions;
%opts_n.FreqUnits = 'Hz';
%opts_n.Title.String = 'Noise frequency response';
%bodeplot(tia_noise, {10, 10^6}, opts_n);
%grid on
%		\end{lstlisting}
%		\subcaption{Results and plots}
%		\label{code:tia:plots}
%	\end{subfigure}
%	\caption{\glsfmtshort{tia} calculations}
%	\label{code:tia}
%\end{figure}

\begin{figure}[ht]
	\centering
		\begin{lstlisting}[frame=single,
numbers=left,
style=Matlab-Pyglike]
% requirements
V_o_tia = .5; % out voltage of tia
V_o = 5; % out voltage of non-inverting amp

% OPA795 specs
GBW = 1.5*10^6;
A_o_dB = 100; % DC open-loop gain in dB
A_o = 10^(A_o_dB/20); % DC open-loop gain
f_o = GBW/A_o; % open-loop cutoff frequency

% BPW34 specs
I_f = 50*10^-9; % max current from diode
C_d = 25*10^-12; % diode capacitance 

% parasitic capacitances
C_cm = 2.2*10^-12; % common-mode capacitance
C_df = 2*10^-12; % differential capacitance
C_i = C_d + C_cm + C_df; % total parasitic capacitance

% R1 calculations
gain = V_o_tia/I_f;
gain_db = 20*log10(gain); % uncomment if you need gain in dB
R_1 = gain*(1 + A_o)/A_o;
%R_1 = 10e6; % realistic value for BPW34

% C1 calculation
omega_o = 2*pi*f_o;
C_1_num = -2*omega_o*R_1*C_i + 1 + sqrt(12*A_o*C_i*R_1*omega_o - 3);
C_1_den = 2*omega_o*R_1*(1 + A_o);
C_1 = C_1_num/C_1_den;

% system transfer function
divisor = (C_i + C_1)*R_1; % normalization coefficient
num = A_o*omega_o/divisor; % numerator
s1 = (1 + omega_o*R_1*(C_i + (1 + A_o)*C_1))/divisor; % first-order s
s0 = omega_o*(1 + A_o)/divisor; % zero-order s
den = [1, s1, s0];
tia = tf(num, den);
Q = sqrt(s0)/s1;

% cutoff calculations
omega_c = bandwidth(tia);
f_c = omega_c/(2*pi);

% noise transfer function
num = A_o*omega_o*[1 1/divisor];
s1 = 1/divisor + omega_o*(1 + A_o*C_1*(C_1 + C_i));
s0 = omega_o*(1 + A_o)/divisor; % 0th oreder term remains the same
den = [1 s1 s0];
tia_noise = tf(num, den);
tia_noise1 = tia_noise;			
		\end{lstlisting}
	\caption{\glsfmtshort{tia} calculations}
	\label{code:tia}
\end{figure}

\begin{figure}[ht]
	\begin{lstlisting}[frame=single,
			numbers=left,
			style=Matlab-Pyglike]
% results and plots
disp('Component values')
disp('R1: ' + R_1/10^6 + ' MOhm')
disp('C1: ' + C_1*10^12 + ' pF')
disp('System characteristics')
disp('fc = ' + f_c/10^3 + ' kHz = ' +  omega_c/10^3 + ' krad/s')
disp('Q (actual): ' + Q)
disp('Q (ideal) : ' + 1/sqrt(3) + ' = sqrt(3)/3')
% bode plot system
figure
opts = bodeoptions;
opts.FreqUnits = 'Hz';
opts.Title.String = 'Frequency response';
bodeplot(tia, {10, 10^6}, opts);
% impulse plot system
figure
ip = impulseplot(tia);
% bode plot noise
figure
opts_n = bodeoptions;
opts_n.FreqUnits = 'Hz';
opts_n.Title.String = 'Noise frequency response';
bodeplot(tia_noise, {10, 10^6}, opts_n);
grid on
	\end{lstlisting}
	\caption{Plots of the \glsfmtshort{tia} calculations}
	\label{code:tia:out}
\end{figure}


\begin{figure}[ht]
	\begin{lstlisting}[frame=single,
		numbers=left,
		style=Matlab-Pyglike]
% values from the TIA calculation script
R_1 = 10e6; % transimpedance (DC component)
V_o_tia = .5; % out voltage of tia
V_o = 5; % out voltage of non-inverting amp

V_cc = 5; % positive rail
V_ee = -5; % negative rail
I_q_a = 1.5e-3; % maximum quiescent current of AD795
I_q_b = 0.8e-3; % maximum quiescent current of AD820
R_2 = 11e3; % see report
R_3 = 110e3; % see report
	
% TIA-stage power
I_f_a = V_o_tia/R_1;
P_l_a = (V_cc - V_o_tia)*I_f_a; % load power
P_q_a = (V_cc - V_ee)*I_q_a; % quiescent power
P_a = P_l_a + P_q_a; % total power
		
% NIA-stage power
I_f_b = (V_o - V_o_tia)/R_3;
P_l_b = (V_cc - V_o)*I_f_b; % load power
P_q_b = (V_cc - V_ee)*I_q_b; % quiescent power
P_b = P_l_b + P_q_b; % total power

% R2 power
P_R2 = V_o_tia*I_f_b;
	
% total power
P = P_a + P_b + P_R2;
I = P/(V_cc - V_ee);
disp('TIA power     : ' + P_a*10^3 + ' mW')
disp('NIA power     : ' + P_b*10^3 + ' mW')
disp('R2 power      : ' + P_R2*10^3 + ' mW')
disp('Total power   : ' + P*10^3 + ' mW')
disp('Supply current: ' + I*10^3 + 'mA')
	\end{lstlisting}
	\caption{Power calculation}
	\label{code:power}
\end{figure}

\begin{figure}[ht]
	\begin{lstlisting}[frame=single,
		numbers=left,
		style=Matlab-Pyglike]
tp = 20e-6; % pulse duration
scaling_f = 1500;
td = 10; % normalized dark  time
T = scaling_f * tp; % period
t = 1:scaling_f; 
y = zeros(1, length(t)); 
y(1) = 1; % init pulse cancel
y(scaling_f/2 + 1) = 1; % init pulse
y(scaling_f/2 + td + 1) = 1; % readout pulse
plot(t, y)
writematrix(y, 'pulse_20usD.csv')
disp('Pulse duration    : ' + tp*10^6 + ' us')
disp('Dark time         : ' + td*10^6*tp + ' us')
disp('Sequence period   : ' + T*10^3 + ' ms')
	\end{lstlisting}
	\caption{Power calculation}
	\label{code:t1_pulse}
\end{figure}

\end{appendices}