\chapter{Technical design} \label{chap:tech_design}
% describe specific implementation details
\section{Quantum sensing setup}
\section{Photodetector design}\label{chap:td:tia_design}
As an integral part of the sensing setup, the photodetector needs a lot of attention. Good amplification is crucial to increasing the \gls{snr} of the quantum setup. Based on the design ideas presented in the functional design (see Chapter \ref{chap:photodetection_design}), the photodetection circuit can be drawn up.

\subsection{First iteration}
As previously mentioned, the first iteration of the PCB uses component values provided by the client. There is no specific information for the design process, aside from the fact that the original designer made the photodetector for input currents in the range of several nanoamperes. Such low inputs require high $R_1$ values, however increasing the resistor value results in more biasing current going to the amplifier and a more limited dynamic range \cite{black2021tia}. Furthermore, it is entirely possible that the original design did not thoroughly consider the bandwidth needed for different protocols, as \gls{cwodmr} setups do not require broad bandwidths \cite{acharya2025compact}. Biasing is another concern with this photodiode setup. The configuration shown in Figures \ref{fig:tia} and \ref{fig:td:tia_circuit_parasitic_v1} has the diode in the photovoltaic mode, which is suitable for low-frequency, low-light operation. In spite of this setup being suited to \gls{cwodmr}, pulsed protocols require micro to nanosecond precision




%\begin{equation}\label{eq:gbw}
%	f_{c} = (1 + \frac{C_d + C_{cm} + C_{df}}{C_1})f_{rc} \approx f_{rc}
%\end{equation}
%
%Even though $f_{rc}$ provides an approximation for the cutoff frequency, the parasitic capacitances in the system will cause it to be slightly bigger. Equation \ref{eq:gbw} shows how a more accurate cutoff frequency $f_c$ can be calculated by taking into account $C_d$ and $C_{cm}$ (differential and common-mode capacitance respectively). In this case, $f_c$ tells us the gain bandwidth or the frequency range in which DC gain is retained.

\subsection{Second iteration} \label{chap:td:calc:v2}
Due to time constraints, this iteration uses the same topology as the first one. That being said, it is still designed analytically, taking into account the general practices, but also considering more subtle factors that might hinder the performance of the system. This section uses the mathematical principles presented in \cite{margan2012transimpedance, sackinger2017analysis, horsthemke2025all} to optimize the design of the existing amplifier. 

\begin{figure}[!ht]
	\centering
	\resizebox{.7\textwidth}{!}{%
		\begin{circuitikz}
			\tikzstyle{every node}=[font=\normalsize]
			\draw [ line width=0.5pt](1.5,10.75) to[empty photodiode,l={ \normalsize $D_1$}] (1.5,9);
			\draw [ line width=0.5pt](6.5,10.25) node[op amp,scale=1] (opamp2) {};
			\draw [ line width=0.5pt](opamp2.+) to[short] (5,9.75);
			\draw [ line width=0.5pt] (opamp2.-) to[short] (5,10.75);
			\draw [ line width=0.5pt](7.7,10.25) to[short](8,10.25);
			\draw [line width=0.5pt](5,9) to (5,8.25) node[ground]{};
			\draw [ line width=0.5pt](5,8.5) to[short] (5,9.75);
			\draw [ line width=0.5pt](1.5,10.5) to[short] (1.5,10.75);
			\draw [ line width=0.5pt](1.5,10.75) to[short] (2.75,10.75);
			\node at (4.5,10.75) [circ] {};
			\draw [ line width=0.5pt](4.5,10.75) to[short] (4.5,13.5);
			\draw [ line width=0.5pt](4.5,12.25) to[short] (5,12.25);
			\draw [ line width=0.5pt](4.5,13.5) to[short] (5,13.5);
			\draw [ line width=0.5pt](5,12.25) to[european resistor,l={ \normalsize $R_1$}] (7,12.25);
			\node at (4.5,12.25) [circ] {};
			\draw [ line width=0.5pt](7,12.25) to[short] (7.5,12.25);
			\draw [ line width=0.5pt](7.5,12.25) to[short] (7.5,11);
			\draw [ line width=0.5pt](7.5,12.25) to[short] (7.5,13.5);
			\draw [line width=0.5pt](5,13.5) to[C,l={ \normalsize $C_1$}] (7,13.5);
			\node at (7.5,12.25) [circ] {};
			\draw [ line width=0.5pt](7,13.5) to[short] (7.5,13.5);
			\draw [ line width=0.5pt](1.5,10.75) to[short] (1.5,10.5);
			\draw [line width=0.5pt](1.5,9) to (1.5,8.25) node[ground]{};
			\draw [line width=0.5pt, ->, >=Stealth] (1,10.25) -- (1,9.25);
			\node [font=\normalsize] at (0.6,9.75) {$-I_f$};
			\draw [ line width=0.5pt](6.5,10.25) node[op amp,scale=1] (opamp2) {};
			\draw [ line width=0.5pt](opamp2.+) to[short] (5,9.75);
			\draw [ line width=0.5pt] (opamp2.-) to[short] (5,10.75);
			\draw [ line width=0.5pt](7.7,10.25) to[short](8,10.25);
			\draw (8,10.25) to[short, -o] (8.25,10.25) node[] {$\ \ \ \ \ \ \ \ \ V_{out}$};
			\draw (7.5,11) to[short] (7.5,10.25);
			\draw [line width=1pt](5,9.75) to[C,l={ \normalsize $C_{df}$}] (5,10.75);
			\draw [line width=1pt](-0.25,9) to[C,l={ \normalsize $C_{d}$}](-0.25,10.5);
			\draw (0.25,10.5) to[short] (1.5,10.5);
			\draw (0.25,9) to[short] (1.5,9);
			\draw (3.75,10.75) to[short] (5,10.75);
			\draw [line width=1pt](3.75,8) to[C,l={ \normalsize $C_{cm}$}](3.75,10.75);
			\draw (3.75,8.5) to (3.75,8.25) node[ground]{};
			\draw [ line width=0.5pt](6.5,10.25) node[op amp,scale=1](opamp2) {};
			\draw [ line width=0.5pt](opamp2.+) to[short] (5,9.75);
			\draw [ line width=0.5pt] (opamp2.-) to[short] (5,10.75);
			\draw [ line width=0.5pt](7.7,10.25) to[short](8,10.25);
			\node at (3.75,10.75) [circ] {};
			\node at (5,10.75) [circ] {};
			\node at (5,9.75) [circ] {};
			\node at (1.5,9) [circ] {};
			\node at (1.5,10.5) [circ] {};
			\node at (7.5,10.25) [circ] {};
			\draw (-0.25,10.5) to[short] (0.25,10.5);
			\draw (-0.25,9) to[short] (0.25,9);
			\draw (2.75,10.75) to[short] (3.75,10.75);
		\end{circuitikz}
	}%
	\caption{Parasitic capacitances in a \gls{tia} circuit}
	\label{fig:td:tia_circuit_parasitic_v1}
\end{figure}

Introducing the real-world parasitic capacitances to the ideal \gls{tia} shown in Figure \ref{fig:tia} results in the circuit in Figure \ref{fig:td:tia_circuit_parasitic_v1}. The diode capacitance $C_d$, combined with the differential and common-mode capacitances of the amplifier ($C_{df}$ and $C_{cm}$), contributes a significant amount of capacitance to the circuit and can result in instability \cite{cherian2016you}.

\begin{equation}\label{eq:td:ci}
	C_i = C_{d} + C_{df} + C_{cm}
\end{equation}

The combined input capacitance (Equation \ref{eq:td:ci}) is needed when solving the equation arising from \gls{kcl} at the inverting input of the amplifier. $Z_1$ (Equation \ref{eq:td:z1}), or the combined impedance of $R_1$ and $C_1$, is another prerequisite for deriving the transimpedance $Z_t$ using \gls{kcl}. Additionally, the open-loop gain $A_{ol}$ is needed for the derivation of the transimpedance. Equation \ref{eq:td:aol} shows the open-loop gain, expressed with the \gls{dc} open-loop gain $A_o$ and the open-loop cutoff frequency $\omega_o$, which is equivalent to $\frac{v_{out}}{v_{in}}$.

\begin{equation}\label{eq:td:z1}
	Z_1(s) = \frac{1}{\frac{1}{R_1} + sC_1}
\end{equation}

\begin{equation}\label{eq:td:aol}
	A_{ol}(s) = A_o\frac{\omega_0}{s + \omega_0} = \frac{v_{out}}{v_{in}}
\end{equation}

Finally, the values in the \gls{kcl} equation (Equation \ref{eq:td:kcl}) can be substituted, from which $Z_t$ can be derived, as seen in Equation \ref{eq:td:zt}. 

\begin{equation}\label{eq:td:kcl}
	i_f = \frac{v_i}{\frac{1}{sC_i}} + \frac{v_{in} - v_{out}}{Z_1(s)}
\end{equation}

\begin{equation}\label{eq:td:zt}
	\begin{aligned}
	&Z_t(s) = \frac{v_{out}}{i_f} = (R_1\frac{A_o}{1 + A_o})\frac{\frac{\omega_o(1 + A_o)}{(C_i + C_1)R_1}}{s^2 + s\frac{1 + \omega_o(C_i + (1 + A_o)C_1)R_1}{(C_i + C_1)R_1} + \frac{\omega_o(1 + A_o)}{(C_i + C_1)R_1}}
	\end{aligned}
\end{equation}

Before moving on to the pole analysis, the impact of the quality factor $Q$ on the system response should be considered (see Equation \ref{eq:td:f_c_q}). Two system poles present when $Q > 0,5$. For $Q = \frac{\sqrt{3}}{3},$ the system has a Bessel response, which has the flattest group delay, however, if $Q = \frac{\sqrt{2}}{2}$, then the system has a Butterworth response, which means it has the flattest possible amplitude response. Although the group delay is not as important as the amplitude response, a Bessel response also results in smaller amount of overshoot and less jitter, which is why it is preferred. 

\begin{equation}\label{eq:td:poles_general}
	H(s) = H_0\frac{(-s_1)(-s_2)}{(s - s_1)(s - s_2)} = H_0\frac{s_1s_2}{s^2 + s(-s_1 - s_2) + s_1s_2}
\end{equation}

Based on this consideration, the poles can be found using the general form of a second order transfer function, shown in Equation \ref{eq:td:poles_general}. The general form also makes it obvious that the \gls{dc} gain $H_0 \approx R_1$, assuming that $A_0$ is very big.


\begin{equation}\label{eq:td:poles}
	\begin{cases}
		-s_1 - s_2 = \frac{1 + \omega_o(C_i + (1 + A_o)C_1)R_1}{(C_i + C_1)R_1} \\
		s_1s_2 = \frac{\omega_o(1 + A_o)}{(C_i + C_1)R_1}
	\end{cases}
\end{equation}

Taking the general-equation poles and their counterparts from the $Z_t$ expression yields the system of equations shown in Equation \ref{eq:td:poles}. Solving for one pole first results in a quadratic equation with roots shown in Equation \ref{eq:td:pole_roots}.

\begin{equation}\label{eq:td:pole_roots}
	s_{1, 2} = -\frac{1 + \omega_o(C_i(1+A_o)C_1)R_1}{2(C_i + C_1)R_1}\left(1 \pm j\sqrt{\frac{4\omega_o(1 + A_o)(C_i + C_1)R_1}{(1 + \omega_o(C_i + (1 + A_o)C_1)R_1)^2} - 1}\right)
\end{equation}

\begin{equation}\label{eq:td:pole_roots_general}
	s_{1, 2} = -\frac{3}{2}(1 \pm j\frac{\sqrt{3}}{3})
\end{equation}

Achieving a critically-damped system, can be done by getting the roots of the general form of a Bessel filter (as seen in Equation \ref{eq:td:pole_roots_general}) and using the imaginary parts to get Equation \ref{eq:td:cap_solve}. This expression can be solved with the quadratic formula again, finally resulting in an answer for the compensating capacitor. 

\begin{equation}\label{eq:td:cap_solve}
	\frac{4\omega_o(1 + A_o)(C_i + C_1)R_1}{(1 + \omega_o(C_i + (1 + A_o)C_1)R_1)^2} = \frac{4}{3}
\end{equation}

\begin{equation}\label{eq:td:cap_roots}
	C_1 = \frac{-2\omega_oR_1C_i \pm \sqrt{12A_oC_iR_1\omega_o - 3}}{2\omega_oR_1(1 + A_o)}
\end{equation}

Although Equation \ref{eq:td:cap_roots} says there are two solutions for the capacitor, the term under the square root will always be much bigger than the rest of the numerator\footnote{Assuming real-world amplifier and diode specifications, and realistic transimpedance gain requirements}, effectively determining the sign of the roots. Simply put, this leaves only one root possible, as capacitors cannot have negative capacitance values.

To further characterize the system, the cutoff frequency $\omega_c$ and damping factor $Q$ can be calculated, because the system adheres to the general form, shown in Equation \ref{eq:td:f_c_q}. Substituting in the pole expressions from Equations \ref{eq:td:poles_general} and \ref{eq:td:poles}.

\begin{equation}\label{eq:td:f_c_q}
	H(s) = H_o\frac{\omega_c^2}{s^2 + s\frac{\omega_c}{Q} + \omega_c^2}
\end{equation}

\begin{equation}
	\begin{cases}
		\omega_c = \sqrt{\frac{\omega_o(1 + A_o)}{(C_i + C_1)R_1}}\\
		Q = \omega_c\frac{(C_i + C_1)R_1}{1 + \omega_o(C_i + (1 + A_o)C_1)R_1}
	\end{cases}
\end{equation}


In the process of designing the \gls{tia} circuit, the noise should be considered. Figure \ref{fig:td:design:tia_noise} shows the thermal noise generated by the resistor (modeled as the current source $i_{nR}$) and the differential noise of the amplifier (modeled as the current source $i_{nA}$)\footnote{Other noise generators, like the gate shot noise, channel noise and induced gate noise can also be calculated, but because they largely depend on the underlying technology of the \gls{fet} op amp, they have not been investigated}.

\begin{figure}[!ht]
	\centering
	\resizebox{.7\textwidth}{!}{%
		\begin{circuitikz}
			\tikzstyle{every node}=[font=\normalsize]
			\draw [ line width=0.5pt](1.5,10.75) to[empty photodiode,l={ \normalsize $D_1$}] (1.5,9);
			\draw [ line width=0.5pt](1.5,10.75) to[empty photodiode,l={ \normalsize $D_1$}] (1.5,9);
			\draw [ line width=0.5pt](6.5,10.25) node[op amp,scale=1] (opamp2) {};
			\draw [ line width=0.5pt](opamp2.+) to[short] (5,9.75);
			\draw [ line width=0.5pt] (opamp2.-) to[short] (5,10.75);
			\draw [ line width=0.5pt](7.7,10.25) to[short](8,10.25);
			\draw [line width=0.5pt](5,9) to (5,8.25) node[ground]{};
			\draw [ line width=0.5pt](5,8.5) to[short] (5,9.75);
			\draw [ line width=0.5pt](1.5,10.5) to[short] (1.5,10.75);
			\draw [ line width=0.5pt](1.5,10.75) to[short] (2.75,10.75);
			\node at (4.5,10.75) [circ] {};
			\draw [ line width=0.5pt](4.5,10.75) to[short] (4.5,13.5);
			\draw [ line width=0.5pt](4.5,12.25) to[short] (5,12.25);
			\draw [ line width=0.5pt](4.5,13.5) to[short] (5,13.5);
			\draw [ line width=0.5pt](5,12.25) to[european resistor,l={ \normalsize $R_1$}] (7,12.25);
			\node at (4.5,12.25) [circ] {};
			\draw [ line width=0.5pt](7,12.25) to[short] (7.5,12.25);
			\draw [ line width=0.5pt](7.5,12.25) to[short] (7.5,11);
			\draw [ line width=0.5pt](7.5,12.25) to[short] (7.5,13.5);
			\draw [line width=0.5pt](5,13.5) to[C,l={ \normalsize $C_1$}] (7,13.5);
			\node at (7.5,12.25) [circ] {};
			\draw [ line width=0.5pt](7,13.5) to[short] (7.5,13.5);
			\draw [ line width=0.5pt](1.5,10.75) to[short] (1.5,10.5);
			\draw [line width=0.5pt](1.5,9) to (1.5,8.25) node[ground]{};
			\draw [line width=0.5pt, ->, >=Stealth] (1,10.25) -- (1,9.25);
			\node [font=\normalsize] at (0.75,9.75) {$I_f$};
			\draw [ line width=0.5pt](6.5,10.25) node[op amp,scale=1] (opamp2) {};
			\draw [ line width=0.5pt](opamp2.+) to[short] (5,9.75);
			\draw [ line width=0.5pt] (opamp2.-) to[short] (5,10.75);
			\draw [ line width=0.5pt](7.7,10.25) to[short](8,10.25);
			\draw (8,10.25) to[short, -o] (8.25,10.25) node[] {$\ \ \ \ \ \ \ \ \ V_{out}$};
			\draw (7.5,11) to[short] (7.5,10.25);
			\draw (3.75,10.75) to[short] (5,10.75);
			\draw [ line width=0.5pt](6.5,10.25) node[op amp,scale=1] (opamp2) {};
			\draw [ line width=0.5pt](opamp2.+) to[short] (5,9.75);
			\draw [ line width=0.5pt] (opamp2.-) to[short] (5,10.75);
			\draw [ line width=0.5pt](7.7,10.25) to[short](8,10.25);
			\node at (7.5,10.25) [circ] {};
			\draw (2.75,10.75) to[short] (3.75,10.75);
			\draw [ line width=0.7pt](4.5,15) to[american current source,l={ \normalsize $i_{nR}$}] (7.5,15);
			\draw (4.5,13.5) to[short] (4.5,15);
			\draw (7.5,15) to[short] (7.5,13.5);
			\node at (4.5,13.5) [circ] {};
			\node at (7.5,13.5) [circ] {};
			\draw [ line width=0.7pt](3.75,10.75) to[american current source,l={ \normalsize $i_{nA}$}] (3.75,8.75);
			\draw (3.75,8.75) to (3.75,8.25) node[ground]{};
		\end{circuitikz}
	}%
	\caption{Noise sources in a \gls{tia} circuit}
	\label{fig:td:design:tia_noise}
\end{figure}

Calculating $i_{nR}$ can be done using the Johnson-Nyquist formula, as shown in Equation \ref{eq:td:noise_res}. Given the fact that the temperature $T$ and noise cutoff frequency $\delta{f}$ cannot be modified in this case, $R_1$ is  the only variable that can be used to lower the thermal noise current.

\begin{equation}\label{eq:td:noise_res}
	i_{nR} = \sqrt{\frac{4kT\Delta{f}}{R_1}}
\end{equation}

Combining the noise sources leads to Equation \ref{eq:td:noise_sources}, which also shows that the noise can be modeled as voltage sources.  

\begin{equation}\label{eq:td:noise_sources}
		i_n(f) = \sqrt{i_{nR}^2 + i_{nA}^2} \Leftrightarrow
		v_n(f) = \sqrt{v_{nR}^2 + v_{nA}^2}
\end{equation}

In order to get a better understanding of the noise, its transfer function can be calculated in much the same way as the transimpedance (Equations \ref{eq:td:aol} - \ref{eq:td:zt}), leading to Equation \ref{eq:td:noise_tf}. While it might not be immediately obvious from the mathematical expression, the noise gain is \num{0} \unit{\decibel} at low frequencies, but it peaks when $f \approx f_c$. Chapter \ref{chap:td:v2} shows the noise frequency response obtained from the system equation.

\begin{equation}\label{eq:td:noise_tf}
	H_n(s) = \frac{v_o}{v_n} = \frac{A_o\omega_o(s + \frac{1}{(C_i + C_1)R_1})}{s^2 + s(\frac{1}{(C_i + C_1)R_1} + \omega_o(1 + A_o\frac{C_1}{C_i + C_1})) + \frac{\omega_o(1 + A_o)}{(C_i + C_1)R_1}}
\end{equation}


\subsection{Third iteration}
While the second version of the photodetector focuses on improving the TIA performance, the third iteration is more about adapting the design for the setup. The most important considerations for the setup are function and size. Form factor constraints arise from the fact that the setup needs to be as small and adhere to the Thorlabs mounting standard. Photodetection also needs to not require many external connections or ideally be standalone. This is feasible for the data flow, as the photodetection PCB only needs light and an output to work. Unlike the data connections, version two of the design has a suboptimal power delivery system, as it uses three pins for the supply of power. 

Three alternatives methods for power delivery were considered, all of which only need power and ground to work. The first one is an onboard split-rail power supply, which would take the power input and would output a fixed positive and negative voltage. Figure \ref{fig:td:des:adafruitsplitrailboost} shows an example of what a split-rail supply implementation might look like. There are also external option \cite{clark2025boost}, which might be necessary in the future, if the setup size needs to be reduced even more.  

\begin{figure}[ht]
	\centering
	\includegraphics[width=0.4\linewidth]{img/adafruit_split_rail_boost}
	\caption{Example of a split-rail power supply implementation using the TPS65133 \glsfmtshort{ic} (image credit to \cite{ti2015tps65133})}
	\label{fig:td:des:adafruitsplitrailboost}
\end{figure}


A simpler solution would be to use a virtual ground driver \gls{ic}, as it would provide a reference point at half of the supply voltage. Such a supply would also be good for decreasing the size, as the \gls{pcb} needs minimal additions to satisfy the power needs of the photodetector. Additionally, a virtual ground circuit, which behaves similarly, can be made using two voltage regulators. The example in Figure \ref{fig:td:des:virtground} is a slightly more stable implementation


\begin{figure}[ht]
	\centering
	\includegraphics[width=0.3\linewidth]{img/virt_ground}
	\caption{Example of a virtual ground driver using the TLE2426 \glsfmtshort{ic} (image credit to \cite{ts2022ground})}
	\label{fig:td:des:virtground}
\end{figure}


Another possibility is to implement a charge pump, which inverts the input voltage. This setup would only need half of the input voltage required for the virtual ground solution, effectively enabling it to be powered via a standard \gls{usb} connector.

\begin{figure}[ht]
	\centering
	\includegraphics[width=0.5\linewidth]{img/chrg_pump}
	\caption{Example of a charge pump inverter using a TPS6040x family \glsfmtshort{ic} (image credit to \cite{instruments2015tps6040x}}
	\label{fig:td:des:chrgpump}
\end{figure}



\begin{table}
\begin{tabular}{|c||c|c|c|}
	\hline
										& Split-rail power supply 	& Virtual ground & Charge pump  \\
	\hline
	\glsfmtshort{ic}					& TPS65133		 			& TLE2426 	     & TPS60400 	\\
	\hline
	$V_{in}$ (\unit{\volt}) 			& 5  			 			& 10 	   		 & 5  			\\
	\hline
	$V_{out}$ (\unit{\volt})			& \textpm 5 				& \textpm 5 	 & -5 			\\
	\hline
	$I_{out}$ (\unit{\m\A})				& 250  						& 40 			 & 60 			\\
	\hline
	Current draw (\unit{\mu\A})			& 15						& 300 	 		 & 210  \\
	\hline
\end{tabular}
\caption{Comparison of power supply options}
\label{tab:power_supply}
\end{table}

The specifications of the different power supply solutions are shown in Table \ref{tab:power_supply}, from which can be deduced that there is a correlation between complexity and performance. The split-rail power supply, as the most complex solution, has the biggest current output range and lowest current consumption, but as illustrated in Figure \ref{fig:td:des:adafruitsplitrailboost}, there are a number of other components that are required for proper implementation. In contrast, the virtual ground supply concept has relatively poor specifications compared to the other options, but it is the only solution that does not employ a switching mechanism, thus avoiding any possible noise. Lastly, the charge pump inverter offers reasonable performance, while not needing as many extra components as the split-rail supply. Furthermore, it addresses the stability concerns that might arise when using a virtual ground \gls{ic}. 

Ultimately, the decision 



\section{Photodetection simulation}
Simulation is an important part of the designing process, as digital models of devices can be more detailed than the mathematical models presented thus far. Using different simulation programs, the real-world performance can be approximated and the math can be validated.


\subsection{First iteration}
\begin{figure} [!ht]
	\centering
	\includegraphics[width=0.7\linewidth]{img/TIA_TINA_SIM_1P6}
	\caption{First iteration of the photodetector circuit}
	\label{fig:photodetecog}
\end{figure}


Figure \ref{fig:photodetecog} shows the schematic of the circuit. The capacitor $C_4$, as well as the current source $I_1$ are used to simulate the behavior of a photodiode.

Tina-TI was used to simulate and visualize the DC gain and frequency response of the system, as seen in Figure \ref{fig:tina2probes}. The signals $V_{ot}$ and $V_{out}$ correspond to the output of the transimpedance and non-inverting amplification stage respectively. The AC plot (Figure \ref{fig:tinaac2probes}) shows the cutoff, at \num{10,77} \unit{\kilo \hertz}, and the gain inside the gain bandwidth, which is \num{160,82} \unit{\decibel}. The DC plot (Figure \ref{fig:tinadc2probes}) shows the voltage with respect to the current and demonstrates the linearity of the system in the range of \numrange{0}{46,31} \unit{\nano \ampere}. After $V_{out}$ reaches \num{5} \unit{V}, the output remains fixed, because it cannot exceed the voltage provided to the amplifier. 

\begin{figure}[ht]
	\centering
	\begin{subfigure}[1a]{.49\linewidth}
		\includegraphics[width=\linewidth]{img/tina_AC_2probes}
		\caption{Frequency response}
		\label{fig:tinaac2probes}
	\end{subfigure}
	\hfill
	\begin{subfigure}[1b]{.49\linewidth}
		\includegraphics[width=\linewidth]{img/tina_DC_2probes}
		\caption{DC gain}
		\label{fig:tinadc2probes}
	\end{subfigure}
	\caption{Simulation of the first iteration of the photodetector}
	\label{fig:tina2probes}
\end{figure}




\subsection{Second iteration}\label{chap:td:v2}

A MATLAB script was written to calculate the component values based on the calculations in Chapter \ref{chap:td:calc:v2}. 

MATLAB calculates the component values with high precision and sourcing components with the exact values is not feasible, which is why components with standard values will be used in the setup. Simulations with standardized components were also done to compare the mathematically-ideal setup to the one in practice. 

\begin{table}[!ht]
	\centering
	\begin{tabular}{|c||c|c|c|}
		\hline
		\glsfmtshort{tia} version & v2 (Ideal) & v2 (Real.)	    & v1 		\\
		\hline
		$C_1$ (\unit{\pico \farad}) & \num{0,96904} & \num{1}	    & \num{1,6} 		\\
		\hline
		$f_c$ (\unit{\kilo \hertz}) & \num{22,0768} & \num{21,167}  &  \num{11,2422}	\\
		\hline
		
		$Q$  					    & \num{0,57735} & \num{0,55996} &  \num{0,35487}   \\
		\hline
	\end{tabular}
	\caption{Photodetector parameters with a \num{10} \unit{\mega \ohm} feedback resistor $R_1$}
	\label{tab:td:params}
\end{table}

Table \ref{tab:td:params} contains the MATLAB calculation results and shows the values of $f_c$ and $Q$ when only $C_1$ changes. $R_1$ is kept the same, as it is equivalent the \gls{dc} transimpedance $Z_{tDC}$\footnote{For realistic op-amps, $\frac{A_o}{A_o + 1} \to 1$, which means $Z_{tDC} \equiv R_1$ can be assumed to simplify explanations}. Importantly, the results show that the previous iteration had an overdamped response. Although the quality factor with realistic component values is also lower than the ideal $Q = \frac{\sqrt{3}}{3}$, the system sill behaves approximately like a Bessel filter.


Using these values, the system model was calculated and plotted in Figure \ref{fig:td:bodev1v2comparison}. Certain differences can be seen between the two iterations. Most notably, the cutoff frequency $f_c$ of the first iteration is somewhat smaller than that of the second iteration. Furthermore, it can be observed that both the ideal and realistic versions of the system behave similarly. 

\begin{figure}[ht]
	\centering
	\includegraphics[width=0.7\linewidth]{matlab/bode_v1_v2_comparison}
	\caption{Bode plot of the system with different capacitors}
	\label{fig:td:bodev1v2comparison}
\end{figure}





\section{Photodetection implementation}
Creating the physical \gls{pcb} is a more straightforward process than the design and simulation, but it still needs to be discussed. 

\subsection{First iteration}
Figure \ref{fig:photodetecogpcb} shows the back side of the PCB. It hosts all components, except for the photodiode, which sits unobstructed on the front side. Requirements for the physical dimensions were also set by the client. The PCB needs to fit the Thorlabs mount standard, since the rest of the setup also uses it. 

\begin{figure}[ht]
	\centering
	\includegraphics[width=0.7\linewidth]{img/photodetec_og_pcb}
	\caption{First iteration of the photodetection PCB}
	\label{fig:photodetecogpcb}
\end{figure}

\subsection{Second iteration}
Due to the fact that the topology of the first iteration was kept, the \gls{pcb} does not need any modifications in order to work with the circuit parameters. However, after a discussion with the client, a ground plane was added. Usually such an addition would be needed in high-speed and/or high-power use cases, but the circuit only deals with low-speed, very-low-power signals. The reason for the ground plane in the updated design is broader quantum protocol support. While \gls{cwodmr} operates at low frequencies, future expansion of the sensing setup might require support for different high-speed pulsing sequences. In that case, the component values can be modified again, without the need for making a new board. Figure \ref{fig:td:imp:photodetecv2gppcb2} shows the new board design. The updated board also has better routing than the previous version, which was done to optimize the noise performance.

\begin{figure}[ht]
	\centering
	\includegraphics[width=0.7\linewidth]{img/photodetec_v2_gp_pcb2}
	\caption{Second iteration of the photodetection PCB}
	\label{fig:td:imp:photodetecv2gppcb2}
\end{figure}



%\section{OLIA implementation}
