\documentclass{report}
\usepackage[utf8]{inputenc}
%easy styling
\usepackage[a4paper,top=2cm,bottom=2cm,left=3cm,right=3cm,marginparwidth=1.75cm]{geometry}
%IEEE-style paper references
\usepackage[style=ieee]{biblatex}
%more math
\usepackage{amsmath}
\usepackage{amssymb}
%better references
\usepackage{hyperref}
%images
\usepackage{graphicx}

\pagestyle{headings}


\title{Learning report}
\author{Vladislav Serafimov}
%\date{June 2025}


\begin{document}
\maketitle

\chapter{Introduction}
\begin{figure}[h]
    \centering
    \includegraphics[width=.8\textwidth]{vmodel.png}
    \caption{V-model}
    \label{fig:vmodel}
\end{figure}

There are various reasons for using the V-model (Figure \ref{fig:vmodel}) for this project. My previous thesis was based on the waterfall model, but this lead to some outcomes which were not to my liking and which I believe the V-model addresses more than adequately. 

As an extension of the waterfall methodology, the V-model offers a linear approach to problems, which suits hardware projects particularly well. Consequently, both models are reliant on well-established requirements and solid planning. Testing is also an important differentiating aspect of the models. Unlike more software-oriented methodologies, the two offer the more straightforward approach of sequential testing, instead of more convoluted alternatives like iterative testing. 

While similar, there are differences that make the V-model more suitable for the project. One of its main advantages is the much more thorough testing system. In contrast with the waterfall model, which is entirely linear, the V-model provides verification and validation opportunities. The extra focus on testing is exactly what I previously found lacking in the waterfall model. Another benefit is the slightly less rigid structure of the methodology, as this gives more room for further refinement of the goals and the more involved participation of the client.

Although the V-model is better suited to the project, I found my implementation of it lacked in some regards. For example, it asks for well-defined goals from the start, however I could not define the specifics of the assignment early on, in part due to my limited amount of knowledge I had in the field of quantum physics.

%The waterfall model (see Figure \ref{fig:vmodel}) was used for this project. There are various reasons for that, but one of the main ones is that the sequential approach was more suited to the way I wanted to approach the project. Finishing each phase before moving to the next guaranteed the structure my work and gave me intermediate milestones to aim for. Another major reason for picking the waterfall model was the documentation structure. Having every phase flow makes it so that every following document is based on the previous. This aids in creating a more structured workflow. Alternatively, I could have used the Scrum model, which is much more focused on continuous, cyclical improvement. However, using it would mean spending much more time on planning and project management than with the waterfall model. Furthermore, using Scrum would suggest a project which involves implementing and improving a product in an iterative manner. I do not need to build upon already implemented features, which makes this aspect of the model unnecessary and maybe even counterproductive. 


%In spite of how useful the waterfall model was, I did encounter some issues with the way I applied it. Because I have previously only worked on very low-level projects, I did not anticipate the amount of testing I needed to do alongside the implementation and integration phases of the project. If I work on high-level coding project like this one in the future, I will make sure to plan for more testing during the realization phase.


\chapter{Self-evaluation}
% structure: .1 - expectation, .2 - evidence, .3 - evaluation
This chapter discusses what I did well during the project and what I could improve on by relating back to the competences that need to be covered during a graduation assignment. Every competence is discussed and the activities related to it are put forward as evidence of its application.

\section{Analyze}\label{chap:analyze}%done
% The analysis of relationships between processes, products and data flows within the context of the environment
\subsection{Expectation}
Analytical skills have many application and are a vital part of a graduation project. They need to be used to contextualize knowledge, which is the first step to applying it to the particular project. One of the most obvious uses of these skills is when doing the analysis of the requirements, but an analytical approach needs to be taken with the products as well. This also involves relating them back to the goals and requirements, in the reflective manner of the V-Model.

\subsection{Evidence}
The analysis competence played a big part in this thesis. I used it to extract the goals and project boundaries from the requirements, as I have done in all of my previous projects. Furthermore, I had to apply this competence extensively to the photodetector. As there was an existing design that needed improvement, I had to decide which possible upgrades would benefit the client and which were not as useful.

\subsection{Evaluation}
Analyzing various aspects of a project is one of my favorite tasks and I also believe I am quite proficient at it. While I think I delivered on the expectations of the client, I also think there might have been other solutions that, for example, would improve the photodetection circuit even more. A broader analysis would have shown a better overview that showcased more alternatives, but due to the time constraints, I was not eager to spend more time on exploring other options.

\section{Design} \label{chap:design}%done
%Design of an ICT system on the basis of set specifications, and within predefined frameworks.
\subsection{Expectation}
Similar to the analyze competence, design heavily relies on previous knowledge. Where they differ is how and what they are applied to. Designing involves the application of theoretical knowledge, in combination with practical guidelines where necessary, in order to come up with a system and predict its behavior. This competence can be applied to tests and measurements as well, because making a standardized routine uses theoretical and practical knowledge much like system design.

\subsection{Evidence}
As the main goal of the project was to develop a photodetector, I spent a lot of time on designing a solution that combined my knowledge of control systems and analog electronics. The different iterations of the system all had intentional design choices behind them, a detailed explanation of which is presented in the technical report. In addition, all tests were also meticulously designed to measure the metrics that mattered most for the client while also being replicable and following testing protocols.

\subsection{Evaluation}
There are a number of similarities between design and analysis, which is why I think I implemented this competence well. Additionally, design is even more technical, which is why I think it suits me. The main thing I need to improve on, in my opinion, is the scope of considerations for the design. Just like in Chapter \ref{chap:analyze}, I believe I should broaden the number of variables I consider during the design process. Currently, I have a very focused approach, which is good for saving time up front, but results in more system iterations and thus more time spent on a single system over time.

\section{Realize}%done
%Build an ICT system on the basis of a specific design and within a predefined framework
\subsection{Expectation}
Realization is one of the most practical competences. It is exercised extensively during the implementation and integration phases of the project, which require the application of practical skills and a proficiency at using laboratory equipment. Another important aspect of realization is that it follows the previously-established design.

\subsection{Evidence}
The project had a big focus on implementation. I had to make boards I designed, as well as solder ones that were later put to use in the setup. The integration of the photodetector subsystems was also carried out. Afterwards, it was integrated with the rest of the quantum sensing setup. Lastly, the software component was also realized in line with the competence expectations. 

\subsection{Evaluation}
As I mentioned, I worked on realizing both software and hardware solutions for the project, as well as the integration with the other systems of the sensing setup. This process went well, although I saw that there are some areas for improvement. Most importantly, I think I need to invest more effort in honing my practical skills. For example, other students from different projects in the same research group could achieve analogous results to mine, but in a shorter time frame. The theoretical knowledge was not an issue, but applying it was not always as straightforward as I would like it to be. 

\section{Control/Verify}%done
%Management and control of all activities aimed at the process of development and deployment of ICT systems and of ICT service management.
\subsection{Expectation}
Control and verification plays an important role in ensuring the systematic tackling of the project in its various stages. Tight control of activities is needed to mitigate any problems that might arise while working on the project. The verification of results also builds on the control structures by creating a feedback loop, which enables the data-driven management of the project. In particular, this competence is most clearly present in the testing activities.  


\subsection{Evidence}
To maintain structure in my work, I documented every stage of the project while I was executing it. This helped me to systematized the work for every individual phase by coming up with the tasks that needed to be completed at the start of a phase. Another benefit of this method of documentation is that the structure of the report often resembles the structure of the working process. Identically, the testing activities also employed a meticulous methodology based on setting and tracking key performance metrics, which were first declared in the report before being implemented.

\subsection{Evaluation}
Over the numerous projects I have done during my degree, I think I have become proficient at controlling the activities of projects. However, during this project I ran into issues that could not be foreseen or avoided, like for example equipment availability. They showed me that I still need to come up with a systematic way of dealing with unforeseen problems. In terms of testing, I feel like I used a solid methodology, but I think the measurements would have been even better if I accounted for and controlled the environmental factors.


%todo: start from here
\section{Manage}%done
%Monitoring and organising (time, money, quality, information, organisation), cooperating and communicating.
\subsection{Expectation}
Project management is an extension of the control competence, but on a more general level. Instead of being concerned with the individual stages of the project, this competence deals with organizational matters. Communication with the client and dissemination of information is also a part of this competence.

\subsection{Evidence}
I used several management techniques, among which the daily task sheet is the most important. I used it to keep track of what I did. Additionally, I employed version control software for data integrity, but it also doubled as a system to keep track of what was done and what needed to be done. The company coach had access to both in order to ensure complete transparency. Other more direct methods were more commonly used to communicate with the client. In particular, biweekly  meetings were held with other interns from the research group where everyone shared their progress.

%My management skills were put to the test during this project. Most of the project was done individually, which meant I was the sole person in charge of organization and the only one who would be impacted by my choices. This meant I had a lot of freedom to tailor my work environment to myself by, for example, working earlier than most people and switching from technical work to writing documentation whenever I felt like I needed to. However, certain parts of the project required me to collaborate with people. This created situations that were a balancing act and where I needed to find compromises that would suit all of the parties involved. For example, because the project was done as a collaboration between a company and a Saxion research group, the requirements of both parties needed to be addressed. Some were accepted, others were either reworked or refused, because of the circumstances surrounding the project.

\subsection{Evaluation}
%Project management is a difficult skill to master, as it involves planning and organizing resources that differ from project to project. This is exactly why I feel the need to improve my problem solving consistency. Dealing with certain issues is harder for me, especially if it involves multiple people. In the example with the project requirements of this project, I did resolve the problem at the end, but the process was slower and less efficient than I would have liked it to be.  

\section{Advise}%done
%Advice regarding the reorganisation of processes and / or data flows. Advise on new ICT systems to be developed or purchased on the basis of an analysis and in consultation with stakeholders.
\subsection{Expectation}
%Giving advice is a key part of every graduation assignment. Students are expected to become experts and advise the company how the project can be continued. In case the company wants to further a project after the completion of a graduation thesis, the student will have already outlined what needs to be developed or reworked in the future.

\subsection{Evidence}
%The main purpose of the project is to answer the question that is asked in the introduction of the technical report and to advise the team on what to do after the end of my graduation period. The presentation also serves a similar purpose. Both are created in part to introduce the next student or researcher involved in the project and to inform their future work.

\subsection{Evaluation}
%One of my main points of improvement when it comes to giving advice is communicating complex topics. I still need to work on the way I relay information to people with different levels of technical skills. In past projects, I have struggled with the final presentation. Striking a balance between technical depth and accessibility is something I am always consciously trying to achieve with every presentation, but I will have to get better at it. 

\section{Research}\label{chap:research}
%Formulating / answering main- and sub-questions, using different sources (i.e. literature, web-pages, publications), validating sources.
\subsection{Expectation}
%Research is the foundation of the design and realization of the final product. A study of academic and professional literature, accompanied by good source selection and information filtering, is an almost universal prerequisite for any work to be scientifically viable.

\subsection{Evidence}
%While research was not the main focus of the project, it was still important for the proper completion of the final product. Devising a good design and implementing it is preceded by a solid research. The algorithm mentioned in Chapter \ref{chap:design} was based on data from studies on variables like solar radiation and solar energy based on the time of day.

\subsection{Evaluation}
%In terms of areas that need improvement, my company coach brought up the fact that I can overcomplicate research. I did this when researching the data simulation, but after discussing the issue with my coach I started working on fixing it.


\section{Professionalize}
%Developing and improving skills necessary for the other competences, communicating, reflecting.

\subsection{Expectation}
%The professionalization competence overlaps with many of the others. This is because, at its core, it is the ability to function in a professional context. While the scope of the competence is broad, communication and reflection are the most important aspects of it.   

\subsection{Evidence}
%Professionalization has been a big part of my studies so far. Having completed an internship in the past, I was prepared for a lot of the challenges of the professional environment when I started the project. In the last few months, I have had to navigate meetings with different stakeholders and to manage their expectations about the project.

\subsection{Evaluation}
%This competence is one of the harder to master for me. As previously mentioned in Chapter \ref{chap:design}, some of my soft skills have improved, but still need work. As professionalization is very reliant on communication, this proved to be an issue. I also feel like I need to work more on teamwork, because I worked on this project alone.  


\chapter{Self-analysis}
%The student should complete a written self-analysis of their own qualities and weaknesses in functioning in a professional context. This does not involve any subject-related aspects of the functioning but rather the professional aspects of the functioning.

%This project was my second time working on an assignment outside of my studies. Because of the previous experience, going into the project I knew how to deal with professional matters. However, I still had some difficulties.

%Overall, I believe all my competences were up to standard. In general, I am a more technical person, which was also the case during this project. In terms of soft skills, I also did better than before. My approach to tracking the progress of the project has been more structured than it was in previous projects, which has helped my coaches stay updated. 

%As previously mentioned, one of my skills that needs the most improvement is communication. In particular, I still need to work on how to communicate technical information to people who have a technical background, but are not specialists on the topic. This issue I am having is the main manifestation of a bigger problem I have: overcomplicating. I have been made aware that, even though my technical functioning is good, I have the tendency to explain my work in a convoluted manner. This makes it harder for people who are not acquainted with the project to understand my work.

\chapter{Personal development plan}
%The student describes in what direction he/she wishes to develop and how, in a 2 years outlook. In this context the student uses the outcome of the self-reflection, using the subject-related competences and the self-analysis for functioning in a professional context.
%Professionally, I want to develop in the field of embedded audio software/firmware. I based my personal development plan on points of improvement that will help me reach this main professional goal. Furthermore, I included skills that are generally useful and need to be refined further, no matter what specialization I choose. 

%In the relatively short time interval of 2 years, I want to improve many of my professional and technical abilities. However, I will mainly focus on communication. As previously mentioned, I want to get better at communicating with other technical people outside of my field of expertise. Because I need to do a second graduation assignment, I will have another chance to improve during my study. Afterwards, I will work on this skill in the workforce, as a part of a team. I will also work on my technical skills then. As this project was very high-level, I want to improve my technical skills by working a job with a focus on low-level programming or even circuit design. Another possibility I am considering is enrolling in a master's degree in the filed of embedded systems, which will help me gain more knowledge and also some practical skills. Finally, I will also try to create a more structured workflow, to which I am able to adhere in my future projects. I have had some issues while trying to be more consistent and I believe an organized routine will help me with that. 

%professional profile: what do I wanna be known/viewed as professionally


\end{document}