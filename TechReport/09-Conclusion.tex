\chapter{Testing and results}\label{chap:conclusion}
%This chapter discusses the results of the tests and what conclusions could be gathered from them. All tests were based on real scenarios, but the performance tests push the platform to its limits by replicating rare and improbable edge cases.

Table \ref{tab:test:tasks} shows all the tasks discussed in Chapter \ref{chap:goals} and their current status. While working on the project, the client decided that purchasing a lock-in amplifier is more worthwhile than implementing \gls{olia}, which is why tasks associated with its implementation are marked as canceled. The items with a reassigned status were given over to two students, who are also involved with the quantum sensing setup. The laser driver was reassigned, because a completely new design was needed, the making of which would require too much time. The pulse sequence script was finished and it was passed on to one of the other students for further refinement and possibly for the implementation of different protocols.

\begin{table}[!ht]
	\centering
	\begin{tabular}{|c||c|c|}
		\hline
		Number 	& Task 									& Status 							\\
		\hline
		1.1 	& Design photodetector 					& Completed							\\
		\hline
		1.2 	& Build \glsfmtshort{olia} 				& Canceled	 						\\
		\hline
		1.3 	& Set up laser driver					& Partially completed; reassigned	\\
		\hline
		2.1 	& Develop data acquisition software 	& Canceled							\\
		\hline
		2.2 	& Program pulse sequences 				& Mostly completed; reassigned	 	\\
		\hline
		3.1 	& Compare lock-in amplifiers			& Canceled  						\\
		\hline
		3.2 	& Test quantum sensing setup 			& In-progress  						\\
		\hline
	\end{tabular}
	\caption{List of tasks and their completion status}
	\label{tab:test:tasks}
\end{table}

%\begin{table}[ht]
%\centering
%\begin{adjustbox}{angle=0}
%\begin{tabular}{|l|l|l|}
%	\hline
%	Number &Task                                                                                                                                                                                     & Status  

%                            \\ \hline
%	1.1    & \begin{tabular}[c]{@{}l@{}}Deploy and configure at least \\ one OpenRemote instance\end{tabular}                                                                                         & \cellcolor[HTML]{34FF34}Completed   \\ \hline
%	1.2    & \begin{tabular}[c]{@{}l@{}}Establish communication to \\ the OpenRemote instance \\ using the HTTP and \\ MQTT protocols\end{tabular}                                                    & \cellcolor[HTML]{34FF34}Completed   \\ \hline
%	2.1    & \begin{tabular}[c]{@{}l@{}}Simulate IoT devices (smart homes)\\ that send and receive concurrent MQTT \\ data to OpenRemote and measure the \\ latency of the transmissions\end{tabular} & \cellcolor[HTML]{34FF34}Completed   \\ \hline
%	2.2    & \begin{tabular}[c]{@{}l@{}}Create a physical IoT device \\ setup using a platform like ESP32 \\ or Arduino and recreate the \\ tests from Goal 1.2\end{tabular}                          & \cellcolor[HTML]{34FF34}Completed   \\ \hline
%	2.3    & Integrate and test OpenRemote's Prophet project                                                                                                                                          & \cellcolor[HTML]{FE0000}Not started \\ \hline
%	2.4    & Test performance and create visualizations                                                                                                                                               & \cellcolor[HTML]{34FF34}Completed   \\ \hline
%	3.1    & Document technical progress                                                                                                                                                              & \cellcolor[HTML]{34FF34}Completed   \\ \hline
%	3.2    & Reflect on personal and professional development                                                                                                                                         & \cellcolor[HTML]{34FF34}Completed   \\ \hline
%	3.3    & Communicate project results                                                                                                                                                              & \cellcolor[HTML]{F8FF00}Ongoing     \\ \hline
%\end{tabular}
%\end{adjustbox}
%\caption{Goal completion}
%\label{tab:goal_com}
%\end{table}



%\section{Test goals and performance metrics}
\section{Test goals}

\subsection{\glsfmtshort{tia}}
Two methods of characterizing the \gls{tia} were used. As the main point of interest is the linearity of the amplifier, the first characterization method is to simulate photodiode readings at different light intensities by using a current source. By varying the \gls{dc} current level, either by sweeps or manually, the performance of the physical device can be compared to the simulations.

The second method supplements the results of the first one. It involves measuring the scattering parameters $S_{11}$ and $S_{21}$ and calculating the frequency response of the system. Equation \eqref{eq:test:s11s22} shows the formula used to approximate the transimpedance $Z_t$ for a load resistor $R_o$ \cite{sackinger2017analysis}. While this method offers insight into the \gls{ac} characteristics of the circuit, its accuracy is worse than that of the first method, due to the approximative nature of the test.

\begin{equation}\label{eq:test:s11s22}
	Z_t(f) \approx \frac{S_{21}(f)R_o}{1 - S_{11}(f)}
\end{equation}


%\begin{table}[ht]
%	\centering
%		\begin{tabular}{|c|c|c|c|}
%			\hline
%			Metric                      & API  & Unit(s) of measurement                                                 & Explanation                                                                               \\ \hline
%			Device provisioning latency & HTTP & seconds (s)                                                            & \begin{tabular}[c]{@{}c@{}}Time between creation \\ request and confirmation\end{tabular} \\ \hline
%			Message loss percentage     & MQTT & percent (\%)                                                           & \begin{tabular}[c]{@{}c@{}}Percent of MQTT \\ messages which are lost\end{tabular}        \\ \hline
%			Resource usage              & -    & \begin{tabular}[c]{@{}c@{}}percent \\ (CPU \%, RAM \%)\end{tabular} & \begin{tabular}[c]{@{}c@{}}CPU and RAM usage \\ when running tests\end{tabular}           \\ \hline
%		\end{tabular}
%	\caption{Metrics of the OpenRemote scalability tests}
%	\label{tab:metrics}
%\end{table}



\section{Test setup}
Characterizing the system requires precision, due to the small currents at its input. As the diode detects ambient light and cannot be set to one current level reliably, a model needs to be used instead. 


\section{Results and discussion}



\section{Known limitations}


\chapter{Conclusion}


\chapter{Recommendations}

