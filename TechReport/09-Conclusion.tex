\chapter{Testing and results}\label{chap:conclusion}
%This chapter discusses the results of the tests and what conclusions could be gathered from them. All tests were based on real scenarios, but the performance tests push the platform to its limits by replicating rare and improbable edge cases.

Table \ref{tab:test:tasks} shows all the tasks discussed in Chapter \ref{chap:goals} and their current status. While working on the project, the client decided that purchasing a lock-in amplifier is more worthwhile than implementing \gls{olia}, which is why tasks associated with its implementation are marked as canceled. The items with a reassigned status were given over to two students, who are also involved with the quantum sensing setup. The laser driver was reassigned, because a completely new design was needed, the making of which would require too much time. The pulse sequence script was finished and it was passed on to one of the other students for further refinement and possibly for the implementation of different protocols.

\begin{table}[!ht]
	\centering
	\begin{tabular}{|c||c|c|}
		\hline
		Number 	& Task 									& Status 							\\
		\hline
		1.1 	& Design photodetector 					& Completed							\\
		\hline
		1.2 	& Build \glsfmtshort{olia} 				& Canceled	 						\\
		\hline
		1.3 	& Set up laser driver					& Partially completed; reassigned	\\
		\hline
		2.1 	& Develop data acquisition software 	& Canceled							\\
		\hline
		2.2 	& Program pulse sequences 				& Mostly completed; reassigned	 	\\
		\hline
		3.1 	& Compare lock-in amplifiers			& Canceled  						\\
		\hline
		3.2 	& Test quantum sensing setup 			& In-progress  						\\
		\hline
	\end{tabular}
	\caption{List of tasks and their completion status}
	\label{tab:test:tasks}
\end{table}

%\begin{table}[ht]
%\centering
%\begin{adjustbox}{angle=0}
%\begin{tabular}{|l|l|l|}
%	\hline
%	Number &Task                                                                                                                                                                                     & Status  

%                            \\ \hline
%	1.1    & \begin{tabular}[c]{@{}l@{}}Deploy and configure at least \\ one OpenRemote instance\end{tabular}                                                                                         & \cellcolor[HTML]{34FF34}Completed   \\ \hline
%	1.2    & \begin{tabular}[c]{@{}l@{}}Establish communication to \\ the OpenRemote instance \\ using the HTTP and \\ MQTT protocols\end{tabular}                                                    & \cellcolor[HTML]{34FF34}Completed   \\ \hline
%	2.1    & \begin{tabular}[c]{@{}l@{}}Simulate IoT devices (smart homes)\\ that send and receive concurrent MQTT \\ data to OpenRemote and measure the \\ latency of the transmissions\end{tabular} & \cellcolor[HTML]{34FF34}Completed   \\ \hline
%	2.2    & \begin{tabular}[c]{@{}l@{}}Create a physical IoT device \\ setup using a platform like ESP32 \\ or Arduino and recreate the \\ tests from Goal 1.2\end{tabular}                          & \cellcolor[HTML]{34FF34}Completed   \\ \hline
%	2.3    & Integrate and test OpenRemote's Prophet project                                                                                                                                          & \cellcolor[HTML]{FE0000}Not started \\ \hline
%	2.4    & Test performance and create visualizations                                                                                                                                               & \cellcolor[HTML]{34FF34}Completed   \\ \hline
%	3.1    & Document technical progress                                                                                                                                                              & \cellcolor[HTML]{34FF34}Completed   \\ \hline
%	3.2    & Reflect on personal and professional development                                                                                                                                         & \cellcolor[HTML]{34FF34}Completed   \\ \hline
%	3.3    & Communicate project results                                                                                                                                                              & \cellcolor[HTML]{F8FF00}Ongoing     \\ \hline
%\end{tabular}
%\end{adjustbox}
%\caption{Goal completion}
%\label{tab:goal_com}
%\end{table}



%\section{Test goals and performance metrics}
\section{Test goals}

%\subsection{\glsfmtshort{tia}}
There are two suitable methods of characterizing the \gls{tia}. As the main point of interest is the linearity of the amplifier, the first characterization method is to simulate photodiode readings at different light intensities by modeling it as a current source. By varying the diode model current, either by sweeps or manually, the performance of the physical device can be compared to the simulations.

The second method supplements the results of the first one. It involves measuring the scattering parameters $S_{11}$ and $S_{21}$ and calculating the frequency response of the system. Equation \eqref{eq:test:s11s22} shows the formula used to approximate the transimpedance $Z_t$ for a load resistor $R_l$ \cite{sackinger2017analysis}. This method provides a better overview of the transimpedance over the whole bandwidth of the photodetector.

\begin{equation}\label{eq:test:s11s22}
	Z_t(f) \approx \frac{S_{21}(f)R_l}{1 - S_{11}(f)}
\end{equation}

Neither of these methods uses a real photodiode, which is why an integration test is also necessary. Using the laser driver, a square-wave signal is pulsed at different frequencies within the amplifier bandwidth. By comparing the signal output by the photodetector to the signal output by the function generator, the performance of the system can be determined.

%\begin{table}[ht]
%	\centering
%		\begin{tabular}{|c|c|c|c|}
%			\hline
%			Metric                      & API  & Unit(s) of measurement                                                 & Explanation                                                                               \\ \hline
%			Device provisioning latency & HTTP & seconds (s)                                                            & \begin{tabular}[c]{@{}c@{}}Time between creation \\ request and confirmation\end{tabular} \\ \hline
%			Message loss percentage     & MQTT & percent (\%)                                                           & \begin{tabular}[c]{@{}c@{}}Percent of MQTT \\ messages which are lost\end{tabular}        \\ \hline
%			Resource usage              & -    & \begin{tabular}[c]{@{}c@{}}percent \\ (CPU \%, RAM \%)\end{tabular} & \begin{tabular}[c]{@{}c@{}}CPU and RAM usage \\ when running tests\end{tabular}           \\ \hline
%		\end{tabular}
%	\caption{Metrics of the OpenRemote scalability tests}
%	\label{tab:metrics}
%\end{table}



\section{Test setup}


\subsection{Photodiode-model characterization}
Characterizing the system requires precision, due to the small currents at its input. As the diode detects ambient light and cannot be set to one current level reliably, a model needs to be used instead (see Figure \ref{fig:test:diode_equiv}). A voltage source can then be used to simulate the nanoampere photovoltaic current $I_f$ generated by the diode. This model assumes the parallel shunt resistance is so high that it can be omitted.

\begin{figure}[ht]
	\centering
	\resizebox{.5\textwidth}{!}{%
		\begin{circuitikz}
			\tikzstyle{every node}=[font=\normalsize]
			\draw[line width=0.5pt] (1,0) 
			to [european resistor, l={$R_i$}] (1,-2)
			(1, 0) node[circ]{}
			(1,-2) node[ground]{};
			
			\draw [line width=0.5pt] (0,0) 
			to [short, o-] (1, 0)
			to [capacitor, l={$C_i$}] (3,0)
			to [european resistor, l={$R_d$}] (5,0)
			to [capacitor, l={$C_d$}] (5,-2)
			(5, 0) node[circ]{}
			(5, -2) node[ground]{};
			
			\draw [line width=0.5pt] (5, 0)
			to [short] (6,0)
			(6,0) node[currarrow]{};
			
			\node at (-0.5, 0) {$V_i$};
			\node at (6.5, 0) {$I_f$};

		\end{circuitikz}
	}%
	\caption{Photodiode model}
	\label{fig:test:diode_equiv}
\end{figure}

Using this model, the diode current $I_f$ can be approximated using Equation \eqref{eq:test:i_f}. The approximation is true only if the input impedance $Z_i$ (see Equation \eqref{eq:test:z_i}) of the \gls{tia} is significantly lower than $R_d$ \cite{razavi2019transimpedance}. Keeping this in mind, a resistor six to seven orders of magnitude bigger was selected for the physical circuit.

\begin{equation}\label{eq:test:i_f}
	I_f = \frac{V_i}{R_d}
\end{equation}

\begin{equation}\label{eq:test:z_i}
	Z_i = \frac{R_1}{A_o + 1}
\end{equation}

\begin{figure}[ht]
	\centering
	\begin{tikzpicture}[
		% define box styles
		boxr/.style={rectangle, thick, draw=red!50!black, fill=red!5, minimum size=5mm},
		boxg/.style={rectangle, thick, draw=green!50!black, fill=green!5, minimum size=5mm},
		boxb/.style={rectangle, thick, draw=blue!50!black, fill=blue!5, minimum size=5mm},
		]
		% draw source/scope nodes
		\node[boxg] (fgen)								{Function generator};
		\node[boxg] (scope)		[right=15 mm of fgen] 	{Oscilloscope};
		% draw main system nodes
		\node[boxb] (phm) 	[below=of fgen]			{Photodiode model};
		\node[boxb] (amps) 	[below=of scope]		{Amplifiers};
		
		
		% draw lines
		\draw[->, thick]			(fgen.south)		to node[midway, left]{Signal} 	(phm.north);
		\draw[->, thick] 			(phm.east) 			to node[midway, above]{Current} 			(amps.west);
		\draw[->, thick]			(amps.north) 		to node[midway, right]{Signal} 				(scope.south);
	\end{tikzpicture}
	\caption{\gls{tia} characterization with photodiode model test setup}
	\label{fig:test:tia_dc:setup}
\end{figure}

\subsection{S-parameter characterization}
Scattering parameters, also known S-parameters, are often used to characterize \gls{rf} circuits and antennas by using a \gls{vna}. For this particular test, the $S_{11}$ and $S_{21}$ parameters are needed to calculate the transimpedance using Equation \eqref{eq:test:s11s22}. In order to measure both, the \gls{vna} needs to be configured as shown in Figure \ref{fig:test:s_params:setup}. $S_{11}$ can be measured with only port 1 of the \gls{vna} being connected to the photodetector. However, port 2 is needed to measure $S_21$ at the output of the photodetector.

\begin{figure}[ht]
	\centering
	\begin{tikzpicture}[
		% define box styles
		boxr/.style={rectangle, thick, draw=red!50!black, fill=red!5, minimum size=5mm},
		boxg/.style={rectangle, thick, draw=green!50!black, fill=green!5, minimum size=5mm},
		boxvna/.style={rectangle, thick, draw=green!50!black, fill=green!5, minimum size=10mm},
		boxb/.style={rectangle, thick, draw=blue!50!black, fill=blue!5, minimum size=5mm},
		]
		% draw source/scope nodes
		\node[boxg, minimum width=30mm] (vna) at (0, 0)							{\glsfmtshort{vna}};
		\node[boxg, minimum width=15mm]	(vna-in)[below left=0mm and -15.25mm of vna]	{Port 1};
		\node[boxg, minimum width=15mm] (vna-out)[below right=0mm and -15.25mm of vna]	{Port 2};
		\node[boxb] (detec) 	[below=10mm of vna]		{Photodetector};
		
		% draw lines
		\draw[thick] (vna-in.west) |- (-20mm, -5mm) |- (detec.west);
		\draw[thick] (vna-out.east) |- (20mm, -5mm) |- (detec.east);
		
		
	\end{tikzpicture}
	\caption{Example of a filled sequence buffer}
	\label{fig:test:s_params:setup}
\end{figure}

\subsection{Integration test}
The integration test has a similar test setup to the complete quantum sensing setup. However, the light from the laser directly illuminates the photodiode instead of a diamond sample \gls{nv}. This was done on purpose, because this test layout provides realistic information on the functioning of the photodetector when used in the quantum sensing setup. A filter was still placed in front of the photodetector to remove the green component of the laser and reduce the overall light intensity. When the tests were done, the new version of the laser driver was available and its \gls{pcb} included a laser, as seen in Figure \ref{fig:test:tia_integ:setup}. In addition to the diagram, Figure \ref{fig:test:setupinteg} shows what the physical measurement setup looks like. 

\begin{figure}[ht]
	\centering
	\begin{tikzpicture}[
		% define box styles
		boxr/.style={rectangle, thick, draw=red!50!black, fill=red!5, minimum size=5mm},
		boxg/.style={rectangle, thick, draw=green!50!black, fill=green!5, minimum size=5mm},
		boxb/.style={rectangle, thick, draw=blue!50!black, fill=blue!5, minimum size=5mm},
		]
		% draw source/scope nodes
		\node[boxg] (fgen)								{Function generator};
		\node[boxg] (scope)		[right=10 mm of fgen] 	{Oscilloscope};
		% draw main system nodes
		\node[boxb] (driver) 	[below=of fgen]			{Driver and laser};
		\node[boxb] (detec) 	[below=of scope]		{Photodetector};
		
		
		% draw lines
		\draw[->, thick, dashed] 	(fgen.east) 		to node[midway, above]{\glsfmtshort{ttl}} 	(scope.west);
		\draw[->, thick]			(fgen.south)		to node[midway, left]{\glsfmtshort{ttl}} 	(driver.north);
		\draw[->, thick] 			(driver.east) 		to node[midway, above]{Light} 			(detec.west);
		\draw[->, thick]			(detec.north) 		to node[midway, right]{Signal} 				(scope.south);
	\end{tikzpicture}
	\caption{Integration test setup diagram}
	\label{fig:test:tia_integ:setup}
\end{figure}


\begin{figure}[ht]
	\centering
	\includegraphics[width=0.7\linewidth]{img/setup_integ_alt}
	\caption{Integration test setup}
	\label{fig:test:setupinteg}
\end{figure}



\section{Results}
\subsection{Photodiode-model characterization}

\subsection{S-parameter characterization}
Unfortunately, the system could not be characterized by means of S-parameters. This is due to the lack of suitable equipment. The laboratory had a \gls{vna} that could only be used to measure S-parameters in the \gls{rf} range. There was another method which was tried, which involves using the waveform and oscilloscope channels of an Analog Discovery 2 as a \gls{vna}. While this method can cover the operating frequencies of the device, it can only measure the $S_{11}$ parameter of the device, because it functions as a network analyzer with a single port. Lastly, attempts at contacting researchers from the University of Twente for their equipment were made, but they did not respond.

\subsection{\glsfmtshort{tia} integration test}
Testing with 500 \unit{\hertz} results in the readings shown in Figure \ref{fig:test:integ_v1}. From the data, it can be seen that the photodetector exhibits a lower bandwidth than what was calculated for the \gls{tia}. This is caused by the photovoltaic-mode diode and the high intensity of the light, which lengthen the on time and fall time of the detector.


\begin{figure}[ht]
	\centering
	\includegraphics[width=.5\linewidth]{img/integ_test_500hz}
	\caption{Integration test of the first iteration of the photodetector (cutoff frequency $f_c$ = \num{10,77} \unit{\kilo\hertz}) with a focused close-proximity laser pulsing at 500 \unit{\hertz}}
	\label{fig:test:integ_v1}
\end{figure}


The previous tests were done with the laser placed less than 1 \unit{cm} away from the photodiode. Better measurements can be achieved by moving the laser further away and not aligning the center of the beam with the diode, both of which reduce the light intensity at the diode. This happens because overexposure in photovoltaic mode severely lengthens the rise and fall time of the photodiode, and this effect is further exacerbated by the extremely high sensitivity of the circuit. Figure \ref{fig:test:v1_test} shows how the photodetector performs under the improved conditions. These measurements present a cleaner response at higher frequency, but the low-frequency measurement in Figure \ref{fig:test:v1_test_50} demonstrates the ambient light effect on the precision of the circuit. When there is no direct laser light present, the photocurrent still generated over 200 \unit{\milli\volt} of voltage. This effect is not noticeable in the higher-frequency measurements, however, they demonstrate a significant increase in signal fall time.

\begin{figure}[ht]
	\centering
	\begin{subfigure}[1a]{.49\linewidth}
		\centering
		\includegraphics[width=\linewidth]{img/v1_test_50}
		\caption{Integration test with the laser pulsing at 50 \unit{\hertz}}
		\label{fig:test:v1_test_50}
	\end{subfigure}
	\hfill
	\begin{subfigure}[1b]{.49\linewidth}
		\centering
		\includegraphics[width=\linewidth]{img/v1_test_5k}
		\caption{Integration test with the laser pulsing at 5 \unit{\kilo\hertz}}
		\label{fig:test:v1_test_5k}
	\end{subfigure}
	
	\begin{subfigure}[2a]{.49\linewidth}
		\centering
		\includegraphics[width=\linewidth]{img/v1_test_10k}
		\caption{Integration test with the laser pulsing at 10 \unit{\kilo\hertz}}
		\label{fig:test:v1_test_10k}
	\end{subfigure}
	\hfill
	\begin{subfigure}[2b]{.49\linewidth}
		\centering
		\includegraphics[width=\linewidth]{img/v1_test_20k}
		\caption{Integration test with the laser pulsing at 20 \unit{\kilo\hertz}}
		\label{fig:test:v1_test_20k}
	\end{subfigure}
	
	\caption{Integration test of the first iteration of the photodetector (cutoff frequency $f_c$ = \num{10,77} \unit{\kilo\hertz}) with an unfocused medium-distance laser}
	\label{fig:test:v1_test}
\end{figure}

In Figure \ref{fig:test:v3_test}, the system demonstrates better high-frequency performance and slightly better low-frequency noise attenuation. That being said, there is a significant amount of noise present. Both its shape and frequency suggest the noise originates from the charge pump.

\begin{figure}[ht]
	\centering
	\begin{subfigure}[1a]{.49\linewidth}
		\centering
		\includegraphics[width=\linewidth]{img/v3_diode_50}
		\caption{Integration test with the laser pulsing at 50 \unit{\hertz}}
		\label{fig:test:v3_test_50}
	\end{subfigure}
	\hfill
	\begin{subfigure}[1b]{.49\linewidth}
		\centering
		\includegraphics[width=\linewidth]{img/v3_diode_5k}
		\caption{Integration test with the laser pulsing at 5 \unit{\kilo\hertz}}
		\label{fig:test:v3_test_5k}
	\end{subfigure}
	
	\begin{subfigure}[2a]{.49\linewidth}
		\centering
		\includegraphics[width=\linewidth]{img/v3_diode_10k}
		\caption{Integration test with the laser pulsing at 10 \unit{\kilo\hertz}}
		\label{fig:test:v3_test_10k}
	\end{subfigure}
	\hfill
	\begin{subfigure}[2b]{.49\linewidth}
		\centering
		\includegraphics[width=\linewidth]{img/v3_diode_20k}
		\caption{Integration test with the laser pulsing at 20 \unit{\kilo\hertz}}
		\label{fig:test:v3_test_20k}
	\end{subfigure}
	
	\caption{Integration test of the third iteration of the photodetector (cutoff frequency $f_c$ = \num{21,167} \unit{\kilo\hertz}) with an unfocused medium-distance laser}
	\label{fig:test:v3_test}
\end{figure}

\section{Discussion}
As a result of the tests, the first iteration of the photodetector can be used for \gls{cwodmr} measurements. The protocol can be executed with extremely low bandwidth photodetectors (e.g. 18 \unit{\hertz} \cite{acharya2025compact}). That being said, pulsed protocols cannot be achieved with the first iteration of the circuit due to their high frequency requirements. Higher frequencies also exacerbate the issue of overexposure of the sensor, which leads to decreased performance. The effect of the laser on the sensor needs to be considered when integrating the complete setup, as it can affect the measurements. There are cancellation techniques to measure only the \gls{nv} luminescence \cite{sewani2020coherent}, but if the photodiode is overexposed or the signal is getting limited by the power rail, the output will not be usable. 


\begin{figure}[ht]
	\centering
	\includegraphics[width=0.7\linewidth]{img/v3_ripple}
	\caption{Noise measurements in minimal-lighting conditions of the third iteration of the photodetector}
	\label{fig:test:v3ripple}
\end{figure}

The third iteration offers a solution that is more suitable for a compact setup. The frequency response is the same as the second version, but the onboard power supply inverter generates unwanted noise. As seen in Figure \ref{fig:test:v3ripple}, there is a large ripple, which is most likely the result of electromagnetic interference. It is also possible that the output filtering capacitor $C_O$ placed on the output of the charge pump does not match the design values, because it was found in a capacitor book in the FabLab. Due to the unknown manufacturer and model, the \gls{esr} of $C_O$ might be too big. Furthermore, there is a possibility that its capacitance is lower than 1 \unit{\micro\farad}, which would also explain the shape of the ripple.

\section{Known limitations}
The most obvious limitation is the lack of equipment for some tests, which lead to using less appropriate alternatives or having to cancel tests. For example, due to the lack of a suitable \gls{vna}, the only method for circuit characterization was using a diode model. The latter would also benefit from a real nanoamp current source instead of the makeshift source that had to be used.

There are many factors that affect the measurements, but perhaps most prominent are the environmental lighting sources. This includes ambient sunlight, which gets through the blinds of the lab, and reflections from the laser. Both significantly affect the photodetector, especially when there is no focused light present. Reflections, in particular, also contributed to the long falling edges in the integration tests.

Furthermore, the light source in the medium-distance integration tests had to be manually adjusted, meaning that the output signals do not represent the ideal performance. In fact, the measurements present results with slight power rail clipping, which is unavoidable with the testing setup. To achieve ideal readings, the laser would have to be affixed and the orientation adjusted, while the intensity at the given distance is measured.


\chapter{Conclusion}
In conclusion, the goal of the project, to enable detection of light signals for quantum sensing, was achieved. While there is still work to be done in order to achieve pulsed quantum protocols, the photodetector platform can be used as a reference for future designs.

According to the introduction in Chapter \ref{chap:intro}, the main purpose of the project was create a photodetector, but integration of the sensing setup was also important. Both were tackled during this project, with the focus being on photodetection. While setup integration also holds importance for the research group, most of the tasks related to it were redistributed among other students. Further refinements to the photodetector and the setup as a whole are possible. Some of the most promising ones are discussed in the recommendations in Chapter \ref{chap:recs}. Complete integration of the whole setup necessitates more work, but is achievable in the near future.

The first iteration of the photodetector is suitable for \gls{cwodmr} measurements with the quantum sensing setup. Despite its functionality, it also presents some minor integration problems, such the need for a dual-rail power supply. 

Version three of the device addresses the major integration concern of having a simple and small power supply solution. At the same time, it also incorporates the improved system response of the second iteration. Although the \gls{ac} characteristics of the circuit are irrelevant for \gls{cwodmr}, they become the most important aspect of the circuit when discussing pulsed protocols. However, the third version of the photodetector suffers a significant noise problem, caused by the inverting charge pump. Although this is a significant problem, its source and possible causes are known, which means that, in the future, the issue can be fixed without much trouble.

The fourth photodetector iteration presents a solution specifically designed to handle pulsed protocols. Even though the exact frequency and gain requirements for the circuit are unknown\footnote{Different papers often present different pulsing specifications. In addition, the client wants to adjust the pulsing specifications, so the necessary photodetector bandwidth can only be estimated}, the fourth version should be able to do $T_1$ measurements. Whether higher-frequency pulsed protocols can be achieved is still unknown, as the setup has not been completely integrated.


\chapter{Recommendations}\label{chap:recs}
There are several recommendations that are important for future implementations, as they may contribute to the precision, noise-suppression and stability of the photodetector and the quantum sensing setup as a whole. 

Firstly, increasing the photodetector bandwidth should be the primary concern of the people working on the setup integration. Version four of the photodetector has a much wider bandwidth, but it is bottlenecked by the photodiode, which is only stable at up to 350 \unit{\kilo\hertz} in the photovoltaic configuration \cite{photodiodephotodiode}. However, biasing a BPW34 diode in photoconductive mode will result in \numrange{2}{30} \unit{\nano\ampere} of dark current, rendering the actual signal unreadable.

Secondly, exploring avalanche photodiodes as an alternative to the p-i-n diode (BPW34) currently in use, might contribute to better measurements. Although they are widely used in low-intensity applications, avalanche diodes are quite expensive. The client wanted to keep the cost low for this project, but in the future more precise pulsed protocol measurements might benefit from the internal amplification and increased precision that are offered by avalanche photodiodes. There are also p-i-n alternatives that might provide better performance than the BPW34. InGaAs diodes are often used in telecommunication due to their stable, low-noise performance even at high frequencies. Their peak sensitivity is also usually in the red and infrared area of the light spectrum, which means the green light from the laser will be proportionally smaller than with the BPW34. Similar to avalanche photodiodes, the major drawback of InGaAs diodes is their price, which is much bigger than the BPW34. Some of the diode alternatives (InGaAs, avalanche and p-i-n) that were explored as possible replacements for the BPW34 were the Hamamatsu G6854-01, Hamamatsu G17190-003K, Luna PDU-V104, OSI FCI-InGaAs-1000 and Marktech MTAPD-06-009. A cheaper alternative is the OSRAM SFH 203 FA, which detects less green light, has a lower dark current and has better directional characteristics, possibly leading to less ambient light detection. In spite of the improved characteristics, their impact will most likely be marginal.

Additionally there are ways to improve performance without modifying the circuit. The simplest way to increase the \gls{snr} is to move the setup to the dark room in the laboratory. Doing this will result in less ambient light being picked up by the photodiode. For future implementations that are interested in decreasing the size of the setup, an enclosure should be designed. This way, the main noise source will be the dark current.


Lastly, if even greater bandwidth is desired in the future, transistor \glspl{tia} should be investigated. Transistors offer finer control over the parameters of the amplifier they constitute. This is, in part, due to the control over the technology of each transistor, but also because they are lower-level components than op-amps, which allows for more intricate layouts. Several sources on the matter were consulted \cite{sackinger2017analysis, santiago2021a, karimi2017silicon, analui2004bandwidth}, but because of time constraints and equipment availability, an op-amp-based solution was chosen instead.

