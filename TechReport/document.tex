\documentclass{report}

% Language setting
% Replace `english' with e.g. `spanish' to change the document language
\usepackage[english]{babel}
\usepackage{blindtext}
\usepackage{titlesec}

% Useful packages
\usepackage{amsmath} % math alignment
\usepackage{siunitx} % SI units
\sisetup{output-decimal-marker={,}} % make it so that \num uses commas for decimals
\usepackage{float}
\usepackage{caption}
\usepackage[skip=0.5ex]{subcaption} % using both subfigure and subcaption causes conflicts
%\usepackage{subfigure}
\usepackage{graphicx}
\usepackage{xcolor}
\usepackage[
colorlinks=true,
urlcolor=blue,
linkcolor=purple,
citecolor=red,
pdftitle={V. Serafimov Bachelor Thesis EEE}
]{hyperref}
%\usepackage{cleveref} % better refs, always load after hyperref
\usepackage[toc]{appendix}
\usepackage{markdown}
\usepackage[table,xcdraw]{xcolor}
\usepackage{adjustbox}
\usepackage{braket} % package for ket (quantum) brackets
\usepackage{csquotes} % quotes using \say{}
\usepackage{matlab-prettifier}

% Set page size and margins
% Replace `letterpaper' with `a4paper' for UK/EU standard size
\usepackage[a4paper,top=2cm,bottom=2cm,left=3cm,right=3cm,marginparwidth=1.75cm]{geometry}

\usepackage{circuitikz} % drawing circuits
\usepackage{tikz, pgfplots} % drawing diagrams
\usetikzlibrary{positioning} % for positioning tikz elements (e.g. right=of, left=of)
\usetikzlibrary{backgrounds} % for drawing in the background (using scope environment)
\usetikzlibrary{decorations.pathreplacing} % for drawing curly braces with tikz

% Code listings
\usepackage{listings}
\colorlet{punct}{red!60!black}
\definecolor{background}{HTML}{EEEEEE}
\definecolor{delim}{RGB}{20,105,176}
\colorlet{numb}{magenta!60!black}

\usepackage{listings}
\lstset{basicstyle=\ttfamily}
\lstset{breaklines=true}

\lstdefinelanguage{json}{
	basicstyle=\normalfont\ttfamily,
	numbers=left,
	numberstyle=\scriptsize,
	stepnumber=1,
	numbersep=8pt,
	showstringspaces=false,
	breaklines=true,
	frame=lines,
	backgroundcolor=\color{background},
	literate=
	*{0}{{{\color{numb}0}}}{1}
	{1}{{{\color{numb}1}}}{1}
	{2}{{{\color{numb}2}}}{1}
	{3}{{{\color{numb}3}}}{1}
	{4}{{{\color{numb}4}}}{1}
	{5}{{{\color{numb}5}}}{1}
	{6}{{{\color{numb}6}}}{1}
	{7}{{{\color{numb}7}}}{1}
	{8}{{{\color{numb}8}}}{1}
	{9}{{{\color{numb}9}}}{1}
	{:}{{{\color{punct}{:}}}}{1}
	{,}{{{\color{punct}{,}}}}{1}
	{\{}{{{\color{delim}{\{}}}}{1}
	{\}}{{{\color{delim}{\}}}}}{1}
	{[}{{{\color{delim}{[}}}}{1}
	{]}{{{\color{delim}{]}}}}{1},
	morestring=[b]",
}

\lstdefinelanguage{curl}{
	basicstyle=\normalfont\ttfamily,
	numbers=left,
	numberstyle=\scriptsize,
	stepnumber=1,
	numbersep=8pt,
	showstringspaces=false,
	breaklines=true,
	frame=lines,
	backgroundcolor=\color{background},
	literate=
	*{0}{{{\color{numb}0}}}{1}
	{1}{{{\color{numb}1}}}{1}
	{2}{{{\color{numb}2}}}{1}
	{3}{{{\color{numb}3}}}{1}
	{4}{{{\color{numb}4}}}{1}
	{5}{{{\color{numb}5}}}{1}
	{6}{{{\color{numb}6}}}{1}
	{7}{{{\color{numb}7}}}{1}
	{8}{{{\color{numb}8}}}{1}
	{9}{{{\color{numb}9}}}{1}
	{:}{{{\color{punct}{:}}}}{1}
	{,}{{{\color{punct}{,}}}}{1}
	{\{}{{{\color{delim}{\{}}}}{1}
	{\}}{{{\color{delim}{\}}}}}{1}
	{[}{{{\color{delim}{[}}}}{1}
	{]}{{{\color{delim}{]}}}}{1},
	morestring=[b]",
}


\lstdefinelanguage{ino}{
	basicstyle=\normalfont\ttfamily,
	numbers=left,
	numberstyle=\scriptsize,
	stepnumber=1,
	numbersep=8pt,
	showstringspaces=false,
	breaklines=true,
	frame=lines,
	backgroundcolor=\color{background},
	literate=
	*{0}{{{\color{numb}0}}}{1}
	{1}{{{\color{numb}1}}}{1}
	{2}{{{\color{numb}2}}}{1}
	{3}{{{\color{numb}3}}}{1}
	{4}{{{\color{numb}4}}}{1}
	{5}{{{\color{numb}5}}}{1}
	{6}{{{\color{numb}6}}}{1}
	{7}{{{\color{numb}7}}}{1}
	{8}{{{\color{numb}8}}}{1}
	{9}{{{\color{numb}9}}}{1}
	{:}{{{\color{punct}{:}}}}{1}
	{,}{{{\color{punct}{,}}}}{1}
	{\{}{{{\color{delim}{\{}}}}{1}
	{\}}{{{\color{delim}{\}}}}}{1}
	{[}{{{\color{delim}{[}}}}{1}
	{]}{{{\color{delim}{]}}}}{1},
	morecomment=[l][\color{gray}]{//},
	morestring=[b][\color{violet}]",
	morekeywords={const, void, char, int, float}, keywordstyle=\color{red},
}

% bibliography
\usepackage[
backend=biber,
style=ieee,
citestyle=numeric-comp, % combine severa comma-separated entries
]{biblatex}
\addbibresource{sources.bib}

% custom headers and footers
\usepackage{fancyhdr, lastpage}
\pagestyle{fancy}
\lhead{Vladislav Serafimov}
\rhead{Enabling photodetection for quantum sensing}
\cfoot{Page \thepage\ of \pageref{LastPage}}

% force chapter pages to use custom headers
\usepackage{etoolbox}
\patchcmd{\chapter}{\thispagestyle{plain}}{\thispagestyle{fancy}}{}{}

% glossary
\usepackage[automake, style=tree]{glossaries-extra} % autocompile and make nested abbreviations appear on separate lines
\setabbreviationstyle[acronym]{long-short} % first occurence shows long and short form
\newacronym{odmr}{ODMR}{Optically Detected Magnetic Resonance}
\newacronym[parent=odmr]{cwodmr}{CW-ODMR}{Constant-Wave \glstext{odmr}}
\newacronym[parent=odmr]{podmr}{P-ODMR}{Pulsed \glstext{odmr}}
\newacronym{ecad}{ECAD}{Electronic Computer-Aided Design}
\newacronym{gui}{GUI}{Graphical User Interface}
\newacronym{mri}{MRI}{Magnetic Resonance Imaging}
\newacronym{mw}{MW}{Microwave}
\newacronym{nv}{NV}{Nitrogen Vacancy}
\newacronym{olia}{OLIA}{Open Lock-In Amplifier}
\newacronym{pcb}{PCB}{Printed Circuit Board}
\newacronym[plural=TIAs]{tia}{TIA}{Transimpedance Amplifier}
\newacronym{ttl}{TTL}{Transistor-Transistor Logic}
\newacronym{snr}{SNR}{Signal-to-Noise Ratio}
\newacronym{kcl}{KCL}{Kirchhoff's Current Law}
\newacronym{dc}{DC}{Direct Current}
\newacronym{fet}{FET}{Field-Effect Transistor}
\newacronym[parent=fet]{mosfet}{MOSFET}{Metal-Oxide-Semiconductor \glstext{fet}}
\newacronym{ti}{TI}{Texas Instruments}
\newacronym{ic}{IC}{Integrated Circuit}
\newacronym{usb}{USB}{Universal Serial Bus}
\newacronym{spice}{SPICE}{Simulation Program with Integrated Circuit Emphasis}
\newacronym{ac}{AC}{Alternating Current}
\newacronym{gbp}{GBP}{Gain-Bandwidth Product}
\newacronym{vna}{VNA}{Vector Network Analyzer}
\newacronym{rf}{RF}{Radio-Frequency}
\newacronym{esr}{ESR}{Equivalent Series Resistance}
\makeglossaries

\title{Enabling photodetection electronics for 
	fluorescent diamond based quantum sensing}
\author{Vladislav Serafimov}

\begin{document}
	\maketitle
	
	
	
	\tableofcontents
	\chapter{Foreword}
	%This project was done as a graduation thesis for the Ambient Intelligence (AmI) research group at Saxion. I had a pleasant time working on this assignment and collaborating with the members of AmI. The professional environment was very productive and the facilities provided by AmI helped greatly with the project work.
	
	%I would like to thank the people who supported me throughout this project: my internship coach, Eyuel Ayele, and my Saxion coach, Yanin Kasemsinsup. The completion of the assignment is due in no small part to their constant guidance and advice. 
	
	\chapter{Summary}
	Quantum sensing is a developing technology with numerous applications in the real world. At present, there is a need for smaller and cheaper quantum sensors. The Applied Nanotechnology research group is one of the many organizations working on a setup with these criteria in mind. However, quantum sensing protocols do not use conventional electrical circuits for measurements, but instead use and manipulate light. Purely optical designs are not possible, which is why the light from the quantum setup needs to be translated to electrical signals. For this project, the student was tasked with solving the challenge of detecting low-intensity light by creating a compact and affordable photodetector. Furthermore, the integration of the different subsystems of the quantum sensing setup needed to be worked on. Using a methodological approach, in the form of the V-Model, the project was tackled and completed successfully. A working photodetector was built to be used for magnetic field measurements using one of the fundamental quantum sensing protocols, \gls{cwodmr}. Later, the detector was upgraded to support $T_1$ measurements. Partial integration tests were done, but complete setup integration has been established as a future project. For high-frequency protocols, modifications to the biasing of the diode might need to be made. Additionally, higher sensitivity can be achieved by changing the p-i-n photodiode for an avalanche or and InGaAs diode, although it is going to significantly increase the cost of the sensing setup.
	% be more specific
	% change question
	% specify stages of the question answering process
	% address 
	
	%An important part of the report is a summary. A summary is used to make clear to other possible interested parties, e.g., 
	%the management, the major purpose / problem / question to be solved / answered, how this has been tackled and the 
	%result of the assignment. 
	%Requirements on a summary:
	%• It must be readable independent from the report, it is not allowed to refer to chapters, figures.
	%• It does not contain figures or tables.
	%• It must summarise:
	%Why the project has been carried out (problem and goal)
	%The used methodology
	%The result (verification with the goal)
	%Major conclusions
	%Major recommendations
	%• It must be written in English, maximum 1 A4
	
	
	\chapter{Introduction}
This chapter introduces the assignment and some foundational concepts of quantum sensing.

\section{Background}
Nitrogen-vacancy (NV) centers \cite{enwiki:1301369588} are imperfections in the atomic structure of diamonds. The two types of NV centers are NV0 and NV-, as seen in Figure \ref{fig:nvcenter}, but the NV- structure is much more commonly used in quantum applications. These imperfections have the useful property of spin-dependent luminescence. This means that the spin of the NV center affects the frequency of the light emitted by the structure\footnote{The NV center only emits light after absorbing photons, a phenomenon called photoluminescence \cite{enwiki:1309081879}}. Using this quality of the NV structure, different environmental metrics (e.g magnetic fields) can be measured. 

The Applied Nanotechnology research group is working on a NV-center-based sensor setup. Processing data from the setup requires working with weak signals that are hard to distinguish from the environmental noise. While this is a significant problem, it is also a very common one. Because of this, there is already widely-used system used to isolate signals in such cases: the lock-in amplifier.

\begin{figure}[ht]
	\centering
	\includegraphics[width=0.7\linewidth]{img/nv_center}
	\caption{NV0 (a) and NV- (b) structures in diamonds (image credit to Haque et al \cite{haque2017overview})}
	\label{fig:nvcenter}
\end{figure}


\section{Purpose of the assignment}\label{purpose}
Implementing a lock-in amplifier is the main purpose of the assignment. To create a complete solution, there are several different functionalities and systems that need to be developed. 

Before doing anything else, the raw sensor data needs to be extracted and then fed to a lock-in amplifier. This should be done in a standardized manner, in order to facilitate testing with different devices. After establishing connection, a control interface needs to be implemented. It needs to be programmed so that it can control all necessary features of the lock-in amplifier. Following the development of the program, a custom photodetection circuit needs to be designed. The circuit should accommodate the sensors and lock-in amplifier. Lastly, an OLIA\footnote{Open Lock-In Amplifier (OLIA) is an open-source microcontroller-based lock-in amplifier. It uses common components, which makes it easy to build \cite{harvie2023olia}} circuit needs to be tested and compared to conventional lock-in systems. 


\section{Assignment specifications}\label{specifications}
As already explained, the assignment is quite broad and involves both hardware and software, causing the need for a number of different tools. 

Most of the hardware tools are already available at the Applied Nanotechnology lab. The lock-in amplifiers which will be used for the tests are the most important pieces of hardware. Zurich Instruments HF2LI is the benchmark lock-in amplifier. There are several different photodetectors available and the one which fits the project best will be picked at a later date. 

In terms of software, there is more freedom of choice. Interfacing with the HF2LI is done through proprietary software, but this is the only required program. There are various electronic computer-aided design (ECAD) software suites that offer the same base functionality. KiCad was selected because the client prefers open-source software. The program for retrieving data from the lock-in amplifiers can be written in both Python and MATLAB. Both languages have good integration with the main lock-in amplifier. They also offer graphic user interface (GUI) programming capabilities and are good for scientific computing overall.



\section{Scope of work}
%The scope of the project was extensively discussed with the company coach to ensure the wishes of AmI were feasible and clearly presented. The discussion resulted in the goals presented in this chapter and the MoSCoW prioritization list in Chapter \ref{analaysis_of_specs}

\subsection{Project boundaries}\label{project_boundaries}
% specify what moscow is and put it in front of the goals
The project boundaries were initially based on the assignment form, but were later discussed with the client and refined further. 

\textbf{Must have}
\begin{itemize}
	\item Hardware platform for photodetection
	\item Software for signal processing and visualization
\end{itemize}

\textbf{Should have}
\begin{itemize}
	\item Tests with different diamond samples
\end{itemize}

\textbf{Could have}
\begin{itemize}
	\item Tests with different quantum protocols
	\item OLIA implementation
	\item Tests comparing OLIA to market solutions
\end{itemize}

\textbf{Will not have}
\begin{itemize}
	\item Laser as a part of the hardware platform
	\item Driver upgrade
\end{itemize}


\subsection{Goals} \label{chap:goals}
% goals are tasks now, change to goals
Based on the MoSCoW priorities from Chapter \ref{project_boundaries}, a set of goals was created to further specify all items from each prioritization category. Every goal was designed so that its outcome results in a tangible project milestone (e.g. a deliverable).

\begin{itemize}
	\item[Goal 1]: Create a hardware setup, which measures and amplifies photodiode signals
	\item[Goal 2]: Develop software to process and visualize lock-in amplifier signals
	\item[Goal 3]: Compare the performance of different lock-in amplifiers
\end{itemize}

While these goals are practical, they are still not specific enough. To eliminate the possibility of confusion, a set of tasks were created. All tasks contribute to one of the three goals.

\begin{itemize}
	\item[Task 1.1]: Design a photodiode PCB, which can accommodate different lock-in amplifiers
	\item[Task 1.2]: Build an operable OLIA
	\item[Task 2.1]: Develop software that acquires signals and is then able to visualize them
	\item[Task 3.1]: Use key performance metrics to compare the OLIA implementation to market solutions
	\item[Task 3.2]: Measure OLIA performance using different diamond samples and quantum protocols
\end{itemize}

\textbf{Task 1.1} involves the design and production of a photodiode PCB. The PCB has to output signals that are not only compatible with lock-in amplifiers that are available on the market, but also with the OLIA. This part of the hardware design has the highest priority, which is why it will be done first. 

\textbf{Task 1.2} is to build an OLIA amplifier, which can be used at Applied Nanotechnology's laboratory. This will be done with the technical specifications and firmware provided by Harvie and de Mello \cite{harvie2023olia}. The necessity for an OLIA is low, because the Applied Nanotechnology research group already has two lock-in amplifiers.

\textbf{Task 2.1} is to write an application in Python or MATLAB. This can be done on a different setup, but ideally it will use the hardware setup from \textbf{goal 1}. Because the OLIA project uses open-source firmware that differs from proprietary solutions, there might need to be two separate applications. This task can only be completed once a measurement setup is built, so its execution will follow the first two tasks.

\textbf{Task 3.1} requires all previous tasks to be finished. The completed setup needs to be used to measure the performance of lock-in amplifiers available on the market and the OLIA implementation. SNR, bandwidth and stability are the main metrics that need to be compared.

\textbf{Task 3.2} is similar to \textbf{task 3.1}, but it is a much broader exploration of the performance of the lock-in amplifiers. Using different diamond samples and quantum protocols will show how the amplifier performs and how different conditions affect it. Because the task can be used to verify the setup from \textbf{goal 1}, it can also be done before \textbf{task 3.1}. Tests with varying diamond samples are more important to the client, which is why they will take precedence over tests with different quantum protocols.


\subsection{Deliverables}
The description of the tasks already provided context for the deliverables, but this subsection contains a formalized version of the deliverables.

\begin{enumerate}
	\item Photodetection PCB 
	\item OLIA implementation
	\item Software application
	\item Comparison visualization
	\item Technical documentation
\end{enumerate}

The only deliverable, which was not mentioned in Chapter \ref{chap:goals} is the technical documentation. This is because it should contain information about every task.

\section{Methodology}\label{methodological_approach}
The V-Model methodology was selected, as it is well-suited for low-level projects. Figure \ref{fig:vmodel} shows a diagram of the phases of the V-Model. Unlike some software-oriented models, the V-Model is very sequential. This can sometimes be seen as detrimental, but in this case it helps with structuring the project. Another benefit of this model is that there are multiple testing activities, which underpin the quality assurance. A contentious feature of the V-Model is the heavy reliance on the initial requirements. This need for deliberate project requirements can be hard to meet, especially if the client representative is not technically proficient. However, this is not the case in this project. The requirements were extensively discussed with the client representative, based on which the project boundaries in Chapter \ref{project_boundaries} were set up.

\begin{figure}[ht]
	\centering
	\includegraphics[width=0.7\linewidth]{img/vmodel}
	\caption{V-Model diagram}
	\label{fig:vmodel}
\end{figure}


\section{Report outline}
%TODO: add after everything else is done
%After the introduction, the report can be split into three parts, each corresponding to one of the phases in the waterfall model from Figure \ref{fig:phases}. 

%The functional design chapter introduces all the necessary prerequisite knowledge, which has been explored during the familiarization and research. It also describes the design process as the project moved from research to implementation. 

%After that, the technical design discusses the most important implementation and integration details. It also discusses practical details that might be useful for future work.

%The last three chapters, testing results, conclusion and recommendations, show the outcomes of the tests and present the explanation of those results. They also contain advice for future work, motivated by the test data interpretation and the conclusions drawn from them.

%The appendices are not integral to the report, but might prove useful to people looking to expand on the project.
	\include{06-Analysis_of_specifications}
	\chapter{Functional design} \label{chap:func_design}
\section{Background knowledge}
\subsection{Spin states}
Spin, at least in quantum mechanics, is the intrinsic angular momentum of a particle. Importantly, it differs from the angular momentum in classical mechanics, which is extrinsic. Spin characterizes systems of particles, usually electrons, using quantum entanglement. This phenomenon refers to the "entanglement", or spin correlation, of a set of particles.

These foundational concepts make it possible to describe quantum systems using various states. The most simple states, used as descriptors, are the energy states. Ground states refer to the system being in an energy minimum. On the other hand, excited states signify that the system has more energy than at its ground state. Additionally, there can be intermediate states during state transition.

While the aforementioned states describe system energy, they have no bearing on the spin. For the purposes of this project, only two spin states need to be explained. The first one is called singlet state. It occurs when an entangled system has a total spin of 0, caused by the mutual cancellation of spin. For example, for a system of two entangled electrons to be a singlet, the two spins would need to point in opposite directions. The second spin state is called triplet and it has a total spin of 1. Triplets can consist of, for instance, two unpaired electrons with aligned spins that sum up to 1. Singlets and triplets both have major distinguishing features and properties, which is why they can be used for quantum sensing. Aside from the difference in spin, triplets tend to have higher energy levels. They also exhibit attraction to magnetic fields, while singlets cannot be influenced directly by magnetism.

%\subsection{Energy levels and state transitions}
\subsection{Zeeman effect}

\subsection{Zero-field splitting}


\section{Quantum protocols}
There are a number of different quantum protocols, which differ in what they can measure, in how precisely they can measure it and in the complexity of the hardware they require to operate. CW ODMR is the main protocol this project is aimed at facilitating. As Saijo et al \cite{saijo2018ac} demonstrate, CW ODMR is relatively simple, while still detecting magnetic field with reasonable sensitivity. Pulsed ODMR does outperform CW ODMR \cite{zhang2020high}, but because of the added complexity working with it is a "Could have" (see Chapter \ref{project_boundaries}). Before being able to run pulsed ODMR on the setup at the lab, several protocols need to be implemented first \cite{sewani2020coherent}. $T_1$ measurements, which are one of the fundamentals of Magnetic Resonance Imaging (MRI), should be conducted first. Afterwards, Rabi oscillations 

\section{CW ODMR}
CW ODMR is a quantum protocol that has been used in many 

\section{Quantum sensing setup}
\section{Photodetection PCB}
\section{OLIA implementation}

	\chapter{Technical design} \label{chap:tech_design}
% describe specific implementation details
\section{Quantum sensing setup}
\section{Photodetection PCB}
\section{OLIA implementation}

	
	%\chapter{Testing} \label{testing}
	%\section{Data modeling and visualization}
	
	\chapter{Testing results}\label{chap:conclusion}
%This chapter discusses the results of the tests and what conclusions could be gathered from them. All tests were based on real scenarios, but the performance tests push the platform to its limits by replicating rare and improbable edge cases.

%Before the technical discussion of results can be started, the status of the goals needs to be presented. Table \ref{tab:goal_com} shows that, while most goals were completed, two are still left. The Prophet implementation could not be started, because, as of completing this report, the feature still has not been released. Disseminating the findings of the project is an activity that has been started already, but can only be completed after the results have been presented officially. This item will be changed from "Ongoing" to "Completed" soon after the completion and submission of this report.

\begin{table}[ht]
\centering
\begin{adjustbox}{angle=0}
\begin{tabular}{|l|l|l|}
	\hline
	Number & Task                                                                                                                                                                                     & Status                              \\ \hline
	1.1    & \begin{tabular}[c]{@{}l@{}}Deploy and configure at least \\ one OpenRemote instance\end{tabular}                                                                                         & \cellcolor[HTML]{34FF34}Completed   \\ \hline
	1.2    & \begin{tabular}[c]{@{}l@{}}Establish communication to \\ the OpenRemote instance \\ using the HTTP and \\ MQTT protocols\end{tabular}                                                    & \cellcolor[HTML]{34FF34}Completed   \\ \hline
	2.1    & \begin{tabular}[c]{@{}l@{}}Simulate IoT devices (smart homes)\\ that send and receive concurrent MQTT \\ data to OpenRemote and measure the \\ latency of the transmissions\end{tabular} & \cellcolor[HTML]{34FF34}Completed   \\ \hline
	2.2    & \begin{tabular}[c]{@{}l@{}}Create a physical IoT device \\ setup using a platform like ESP32 \\ or Arduino and recreate the \\ tests from Goal 1.2\end{tabular}                          & \cellcolor[HTML]{34FF34}Completed   \\ \hline
	2.3    & Integrate and test OpenRemote's Prophet project                                                                                                                                          & \cellcolor[HTML]{FE0000}Not started \\ \hline
	2.4    & Test performance and create visualizations                                                                                                                                               & \cellcolor[HTML]{34FF34}Completed   \\ \hline
	3.1    & Document technical progress                                                                                                                                                              & \cellcolor[HTML]{34FF34}Completed   \\ \hline
	3.2    & Reflect on personal and professional development                                                                                                                                         & \cellcolor[HTML]{34FF34}Completed   \\ \hline
	3.3    & Communicate project results                                                                                                                                                              & \cellcolor[HTML]{F8FF00}Ongoing     \\ \hline
\end{tabular}
\end{adjustbox}
\caption{Goal completion}
\label{tab:goal_com}
\end{table}



\section{Test goals and performance metrics}
%Verifying the scalability of OpenRemote is the main goal of the project and consequently the tests. There are two factors that reflect the scalability of a platform. The first one is the speed or how fast can the user and OpenRemote communicate. The second important factor is how reliable this communication is. Reliability, in this case, means the consistency of the data reception and transmission functionalities.

%Because the project uses 2 APIs for communication, there are 4 possible API parameters that can be tested: HTTP speed, HTTP reliability, MQTT speed and MQTT reliability. However, the MQTT transmission speed was not considered as important enough to investigate by the stakeholders, so it was omitted from the testing activities. There might be merit in testing it with a remote OpenRemote deployment, as this will provide some more useful insight than on the local deployment that is currently in use. Even then, there is little real-world insight to gain from MQTT speed data, because smart home management does not require low-latency communication.

%There is one additional metric, which helps to contextualize the speed and reliability of the platform, but also has importance as a standalone test. This metric is the resource usage of the system. Out of all system resources, CPU and memory usage are the two most important ones.

\begin{table}[ht]
	\centering
		\begin{tabular}{|c|c|c|c|}
			\hline
			Metric                      & API  & Unit(s) of measurement                                                 & Explanation                                                                               \\ \hline
			Device provisioning latency & HTTP & seconds (s)                                                            & \begin{tabular}[c]{@{}c@{}}Time between creation \\ request and confirmation\end{tabular} \\ \hline
			Message loss percentage     & MQTT & percent (\%)                                                           & \begin{tabular}[c]{@{}c@{}}Percent of MQTT \\ messages which are lost\end{tabular}        \\ \hline
			Resource usage              & -    & \begin{tabular}[c]{@{}c@{}}percent \\ (CPU \%, RAM \%)\end{tabular} & \begin{tabular}[c]{@{}c@{}}CPU and RAM usage \\ when running tests\end{tabular}           \\ \hline
		\end{tabular}
	\caption{Metrics of the OpenRemote scalability tests}
	\label{tab:metrics}
\end{table}



\section{Test setup}


\section{Results and discussion}



\section{Known limitations}


\chapter{Conclusion}


\chapter{Recommendations}


	
	
	\begin{appendices}

\chapter{Code}\label{chap:appendix:code}

%\begin{figure}
%	\centering
%	\begin{subfigure}
%		\begin{lstlisting}[frame=single,
%			numbers=left,
%			style=Matlab-Pyglike]
%% requirements
%V_o_tia = .5; % out voltage of tia
%V_o = 5; % out voltage of non-inverting amp
%
%% OPA795 specs
%GBW = 1.5*10^6;
%A_o_dB = 100; % DC open-loop gain in dB
%A_o = 10^(A_o_dB/20); % DC open-loop gain
%f_o = GBW/A_o; % open-loop cutoff frequency
%
%% BPW34 specs
%I_f = 50*10^-9; % max current from diode
%C_d = 25*10^-12; % diode capacitance 
%
%% parasitic capacitances
%C_cm = 2.2*10^-12; % common-mode capacitance
%C_df = 2*10^-12; % differential capacitance
%C_i = C_d + C_cm + C_df; % total parasitic capacitance
%		\end{lstlisting}
%		\subcaption{Requirements and given values}
%		\label{code:tia:consts}
%	\end{subfigure}
%	\\
%	\begin{subfigure}
%		\begin{lstlisting}[frame=single,
%			numbers=left,
%			style=Matlab-Pyglike]
%% R1 calculations
%gain = V_o_tia/I_f;
%gain_db = 20*log10(gain); % uncomment if you need gain in dB
%R_1 = gain*(1 + A_o)/A_o;
%%R_1 = 10e6; % realistic value for BPW34
%
%% C1 calculation
%omega_o = 2*pi*f_o;
%C_1_num = -2*omega_o*R_1*C_i + 1 + sqrt(12*A_o*C_i*R_1*omega_o - 3);
%C_1_den = 2*omega_o*R_1*(1 + A_o);
%C_1 = C_1_num/C_1_den;
%
%% system transfer function
%divisor = (C_i + C_1)*R_1; % normalization coefficient
%num = A_o*omega_o/divisor; % numerator
%s1 = (1 + omega_o*R_1*(C_i + (1 + A_o)*C_1))/divisor; % first-order s
%s0 = omega_o*(1 + A_o)/divisor; % zero-order s
%den = [1, s1, s0];
%tia = tf(num, den);
%Q = sqrt(s0)/s1;
%
%% cutoff calculations
%omega_c = bandwidth(tia);
%f_c = omega_c/(2*pi);
%
%% noise transfer function
%num = A_o*omega_o*[1 1/divisor];
%s1 = 1/divisor + omega_o*(1 + A_o*C_1*(C_1 + C_i));
%s0 = omega_o*(1 + A_o)/divisor; % 0th oreder term remains the same
%den = [1 s1 s0];
%tia_noise = tf(num, den);
%tia_noise1 = tia_noise;
%		\end{lstlisting}
%		\subcaption{$R_1$, $C_1$, gain and noise calculations}
%		\label{code:tia:calcs}
%	\end{subfigure}
%	\\
%	\begin{subfigure}
%		\begin{lstlisting}[frame=single,
%			numbers=left,
%			style=Matlab-Pyglike]
%disp('Component values')
%disp('R1: ' + R_1/10^6 + ' MOhm')
%disp('C1: ' + C_1*10^12 + ' pF')
%disp('System characteristics')
%disp('fc = ' + f_c/10^3 + ' kHz = ' +  omega_c/10^3 + ' krad/s')
%disp('Q (actual): ' + Q)
%disp('Q (ideal) : ' + 1/sqrt(3) + ' = sqrt(3)/3')
%% bode plot system
%figure
%opts = bodeoptions;
%opts.FreqUnits = 'Hz';
%opts.Title.String = 'Frequency response';
%bodeplot(tia, {10, 10^6}, opts);
%% impulse plot system
%figure
%ip = impulseplot(tia);
%% bode plot noise
%figure
%opts_n = bodeoptions;
%opts_n.FreqUnits = 'Hz';
%opts_n.Title.String = 'Noise frequency response';
%bodeplot(tia_noise, {10, 10^6}, opts_n);
%grid on
%		\end{lstlisting}
%		\subcaption{Results and plots}
%		\label{code:tia:plots}
%	\end{subfigure}
%	\caption{\glsfmtshort{tia} calculations}
%	\label{code:tia}
%\end{figure}

\begin{figure}[ht]
	\centering
		\begin{lstlisting}[frame=single,
numbers=left,
style=Matlab-Pyglike]
% requirements
V_o_tia = .5; % out voltage of tia
V_o = 5; % out voltage of non-inverting amp

% OPA795 specs
GBW = 1.5*10^6;
A_o_dB = 100; % DC open-loop gain in dB
A_o = 10^(A_o_dB/20); % DC open-loop gain
f_o = GBW/A_o; % open-loop cutoff frequency

% BPW34 specs
I_f = 50*10^-9; % max current from diode
C_d = 25*10^-12; % diode capacitance 

% parasitic capacitances
C_cm = 2.2*10^-12; % common-mode capacitance
C_df = 2*10^-12; % differential capacitance
C_i = C_d + C_cm + C_df; % total parasitic capacitance

% R1 calculations
gain = V_o_tia/I_f;
gain_db = 20*log10(gain); % uncomment if you need gain in dB
R_1 = gain*(1 + A_o)/A_o;
%R_1 = 10e6; % realistic value for BPW34

% C1 calculation
omega_o = 2*pi*f_o;
C_1_num = -2*omega_o*R_1*C_i + 1 + sqrt(12*A_o*C_i*R_1*omega_o - 3);
C_1_den = 2*omega_o*R_1*(1 + A_o);
C_1 = C_1_num/C_1_den;

% system transfer function
divisor = (C_i + C_1)*R_1; % normalization coefficient
num = A_o*omega_o/divisor; % numerator
s1 = (1 + omega_o*R_1*(C_i + (1 + A_o)*C_1))/divisor; % first-order s
s0 = omega_o*(1 + A_o)/divisor; % zero-order s
den = [1, s1, s0];
tia = tf(num, den);
Q = sqrt(s0)/s1;

% cutoff calculations
omega_c = bandwidth(tia);
f_c = omega_c/(2*pi);

% noise transfer function
num = A_o*omega_o*[1 1/divisor];
s1 = 1/divisor + omega_o*(1 + A_o*C_1*(C_1 + C_i));
s0 = omega_o*(1 + A_o)/divisor; % 0th oreder term remains the same
den = [1 s1 s0];
tia_noise = tf(num, den);
tia_noise1 = tia_noise;			
		\end{lstlisting}
	\caption{\glsfmtshort{tia} calculations}
	\label{code:tia}
\end{figure}

\begin{figure}[ht]
	\begin{lstlisting}[frame=single,
			numbers=left,
			style=Matlab-Pyglike]
% results and plots
disp('Component values')
disp('R1: ' + R_1/10^6 + ' MOhm')
disp('C1: ' + C_1*10^12 + ' pF')
disp('System characteristics')
disp('fc = ' + f_c/10^3 + ' kHz = ' +  omega_c/10^3 + ' krad/s')
disp('Q (actual): ' + Q)
disp('Q (ideal) : ' + 1/sqrt(3) + ' = sqrt(3)/3')
% bode plot system
figure
opts = bodeoptions;
opts.FreqUnits = 'Hz';
opts.Title.String = 'Frequency response';
bodeplot(tia, {10, 10^6}, opts);
% impulse plot system
figure
ip = impulseplot(tia);
% bode plot noise
figure
opts_n = bodeoptions;
opts_n.FreqUnits = 'Hz';
opts_n.Title.String = 'Noise frequency response';
bodeplot(tia_noise, {10, 10^6}, opts_n);
grid on
	\end{lstlisting}
	\caption{Plots of the \glsfmtshort{tia} calculations}
	\label{code:tia:out}
\end{figure}


\begin{figure}[ht]
	\begin{lstlisting}[frame=single,
		numbers=left,
		style=Matlab-Pyglike]
% values from the TIA calculation script
R_1 = 10e6; % transimpedance (DC component)
V_o_tia = .5; % out voltage of tia
V_o = 5; % out voltage of non-inverting amp

V_cc = 5; % positive rail
V_ee = -5; % negative rail
I_q_a = 1.5e-3; % maximum quiescent current of AD795
I_q_b = 0.8e-3; % maximum quiescent current of AD820
R_2 = 11e3; % see report
R_3 = 110e3; % see report
	
% TIA-stage power
I_f_a = V_o_tia/R_1;
P_l_a = (V_cc - V_o_tia)*I_f_a; % load power
P_q_a = (V_cc - V_ee)*I_q_a; % quiescent power
P_a = P_l_a + P_q_a; % total power
		
% NIA-stage power
I_f_b = (V_o - V_o_tia)/R_3;
P_l_b = (V_cc - V_o)*I_f_b; % load power
P_q_b = (V_cc - V_ee)*I_q_b; % quiescent power
P_b = P_l_b + P_q_b; % total power

% R2 power
P_R2 = V_o_tia*I_f_b;
	
% total power
P = P_a + P_b + P_R2;
I = P/(V_cc - V_ee);
disp('TIA power     : ' + P_a*10^3 + ' mW')
disp('NIA power     : ' + P_b*10^3 + ' mW')
disp('R2 power      : ' + P_R2*10^3 + ' mW')
disp('Total power   : ' + P*10^3 + ' mW')
disp('Supply current: ' + I*10^3 + 'mA')
	\end{lstlisting}
	\caption{Power calculation}
	\label{code:power}
\end{figure}

\begin{figure}[ht]
	\begin{lstlisting}[frame=single,
		numbers=left,
		style=Matlab-Pyglike]
tp = 20e-6; % pulse duration
scaling_f = 1500;
td = 10; % normalized dark  time
T = scaling_f * tp; % period
t = 1:scaling_f; 
y = zeros(1, length(t)); 
y(1) = 1; % init pulse cancel
y(scaling_f/2 + 1) = 1; % init pulse
y(scaling_f/2 + td + 1) = 1; % readout pulse
plot(t, y)
writematrix(y, 'pulse_20usD.csv')
disp('Pulse duration    : ' + tp*10^6 + ' us')
disp('Dark time         : ' + td*10^6*tp + ' us')
disp('Sequence period   : ' + T*10^3 + ' ms')
	\end{lstlisting}
	\caption{Power calculation}
	\label{code:t1_pulse}
\end{figure}

\end{appendices}
	
	
	\printglossary[type=\acronymtype]
	\printbibliography
	
\end{document}