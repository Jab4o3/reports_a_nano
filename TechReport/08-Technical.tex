\chapter{Technical design} \label{chap:tech_design}
This chapter discusses the implementation and integration of the various parts of the project. Even though the main focus here is on design and simulation, implementation is also covered where necessary.
% describe specific implementation details
%\section{Quantum sensing setup}
\section{Photodetector design}\label{chap:td:tia_design}
As an integral part of the sensing setup, the photodetector needs a lot of attention. Good amplification is crucial to increasing the \gls{snr} of the quantum setup, however the materials used need to be accessible and cheap. Based on these considerations and the design ideas presented in the functional design (see Chapter \ref{chap:photodetection_design}), the photodetection circuit can be drawn up. The first three versions presented in this section of the report address the basic requirements of a \gls{cwodmr} detector, which can have a bandwidth as low as 18 \unit{\hertz} \cite{acharya2025compact}. In contrast, the fourth iteration of the photodetector design focuses mostly on providing an increase in bandwidth, which is necessary for pulsed protocols.

To summarize the circuit requirements, the input signal of the diode can range from \numrange{0}{50} \unit{\milli\ampere}. This current needs to be amplified in the range from \numrange{0}{5} \unit{\volt}. In addition, the circuit should consider the quantum protocols that it will be used to detect. For \gls{cwodmr}, low noise levels at low frequencies is desired and a bandwidth of 20 \unit{\hertz} is sufficient. In terms of pulsed protocols, $T_1$ is the main one that is considered. Based on the pulsing sequences shown in Sewani et al. \cite{sewani2020coherent}, it was decided that the detector should be able to detect pulses of 5 \unit{\micro\second}. Noise is harder to minimize with the broader bandwidth, which is why it was decided to have separate detectors for $T_1$ and \gls{cwodmr}.

\subsection{First iteration}
As previously mentioned, the first iteration of the \gls{pcb} uses component values provided by the client. Although its main purpose is to enable the setup to do \gls{cwodmr} measurements, it also provides a reference for the following designs, which optimize the detector for the sensing setup and explore pulsing protocol support. In addition, the design was made so that the implementation price is as low as possible, at the request of the client.

There is no specific information for the design process, aside from the fact that the original designer made the photodetector for input currents in the range of \numrange{0}{50} \unit{\nano\ampere}. Such low inputs require high $R_1$ values, however increasing the resistor value results in more biasing current going to the amplifier and a more limited dynamic range \cite{black2021tia}. Furthermore, it is entirely possible that the original design did not thoroughly consider the bandwidth needed for different protocols, as \gls{cwodmr} setups do not require broad bandwidths \cite{acharya2025compact}. Biasing is another concern with this photodiode setup. The configuration shown in Figures \ref{fig:tia} and \ref{fig:td:tia_circuit_parasitic_v1} has the diode in the photovoltaic mode, which is suitable for low-frequency, low-light operation. In spite of this setup being suited to \gls{cwodmr}, pulsed protocols require micro to nanosecond precision, which cannot be achieved with this version of the detector.

As some of the components used in the provided circuit can be used in later redesigns, it is important to discuss their most impactful parameters, which were used to determine if a certain component needed replacing.

Out of all of the components, the performance of the circuit is affected the most by the \gls{tia}-stage amplifier. This version of the photodetector uses an AD795 op-amp, which is suitable for low-frequency, low-noise applications like this one. Figure \ref{fig:td:ad795noise} shows the noise spectral density of the op-amp, which is low at low frequencies, but does not drop off much at higher frequencies. Another important consideration for the \gls{tia} op-amp is the supply current. At 5 \unit{\volt}, the supply current is 1,5 \unit{\milli\ampere}. Lastly, the \gls{gbp} of the AD795 is 1,6 \unit{\mega\hertz} and the open-loop gain is 100 \unit{\decibel}. These last two specifications are the most important for the frequency response of the whole circuit.

After the \gls{tia} op-amp, the post amplifier op-amp is the second most important component. In the photodetector, the non-inverting post-amplification stage is based on an AD820. Its noise density is shown in Figure \ref{fig:td:ad820noise}. While it is significantly higher than the AD795, the AD820 has a much lower supply current of 0,8 \unit{\milli\ampere}.


\begin{figure}[ht]
	\centering
	\begin{subfigure}[1a]{.49\linewidth}
		\includegraphics[width=\linewidth]{img/ad795_noise}
		\caption{Noise spectral density of the AD795 (image credit to the AD795 documentation \cite{ad2019ad795})}
		\label{fig:td:ad795noise}
	\end{subfigure}
	\hfill
	\begin{subfigure}[1b]{.49\linewidth}
		\includegraphics[width=\linewidth]{img/ad820_noise}
		\caption{Noise spectral density of the AD820 (image credit to the AD820 documentation \cite{ad2019ad820})}
		\label{fig:td:ad820noise}
	\end{subfigure}
	\caption{Noise spectral density of the amplifiers used in the pre and post-amplification stages of the photodetector}
	\label{fig:td:opampnoise}
\end{figure}


Lastly, the resistor values of the non-inverting amplifier feedback loop were also considered, although their impact on the signal is minimal. In the post-amplifier in Figure \ref{fig:tia_nia}, the resistor $R_3$ affects the output noise the most \cite{ti2007noise, 1490859} and by reducing its value, the noise can be marginally decreased.

%\begin{equation}\label{eq:gbw}
%	f_{c} = (1 + \frac{C_d + C_{cm} + C_{df}}{C_1})f_{rc} \approx f_{rc}
%\end{equation}
%
%Even though $f_{rc}$ provides an approximation for the cutoff frequency, the parasitic capacitances in the system will cause it to be slightly bigger. Equation \eqref{eq:gbw} shows how a more accurate cutoff frequency $f_c$ can be calculated by taking into account $C_d$ and $C_{cm}$ (differential and common-mode capacitance respectively). In this case, $f_c$ tells us the gain bandwidth or the frequency range in which DC gain is retained.

\subsection{Second iteration} \label{chap:td:calc:v2}
This photodetector iteration aims at redesigning the frequency response of the system for better stability. Due to time constraints, it uses the same topology as the first photodetector version. 

That being said, it is still designed analytically, taking into account the general practices, but also considering more subtle factors that might hinder the performance of the system. This section uses the mathematical principles presented in \cite{margan2012transimpedance, sackinger2017analysis, horsthemke2025all} to optimize the design of the existing amplifier. 

\begin{figure}[!ht]
	\centering
	\resizebox{.7\textwidth}{!}{%
		\begin{circuitikz}
			\tikzstyle{every node}=[font=\normalsize]
			\draw [ line width=0.5pt](1.5,9) to[empty photodiode,l={ \normalsize $D_1$}] (1.5,10.75);
			\draw [ line width=0.5pt](6.5,10.25) node[op amp,scale=1] (opamp2) {};
			\draw [ line width=0.5pt](opamp2.+) to[short] (5,9.75);
			\draw [ line width=0.5pt] (opamp2.-) to[short] (5,10.75);
			\draw [ line width=0.5pt](7.7,10.25) to[short](8,10.25);
			\draw [line width=0.5pt](5,9) to (5,8.25) node[ground]{};
			\draw [ line width=0.5pt](5,8.5) to[short] (5,9.75);
			\draw [ line width=0.5pt](1.5,10.5) to[short] (1.5,10.75);
			\draw [ line width=0.5pt](1.5,10.75) to[short] (2.75,10.75);
			\node at (4.5,10.75) [circ] {};
			\draw [ line width=0.5pt](4.5,10.75) to[short] (4.5,13.5);
			\draw [ line width=0.5pt](4.5,12.25) to[short] (5,12.25);
			\draw [ line width=0.5pt](4.5,13.5) to[short] (5,13.5);
			\draw [ line width=0.5pt](5,12.25) to[european resistor,l={ \normalsize $R_1$}] (7,12.25);
			\node at (4.5,12.25) [circ] {};
			\draw [ line width=0.5pt](7,12.25) to[short] (7.5,12.25);
			\draw [ line width=0.5pt](7.5,12.25) to[short] (7.5,11);
			\draw [ line width=0.5pt](7.5,12.25) to[short] (7.5,13.5);
			\draw [line width=0.5pt](5,13.5) to[C,l={ \normalsize $C_1$}] (7,13.5);
			\node at (7.5,12.25) [circ] {};
			\draw [ line width=0.5pt](7,13.5) to[short] (7.5,13.5);
			\draw [ line width=0.5pt](1.5,10.75) to[short] (1.5,10.5);
			\draw [line width=0.5pt](1.5,9) to (1.5,8.25) node[ground]{};
			\draw [line width=0.5pt, ->, >=Stealth] (2,9.25) -- (2,10.25);
			\node [font=\normalsize] at (2.4,9.75) {$I_f$};
			\draw [ line width=0.5pt](6.5,10.25) node[op amp,scale=1] (opamp2) {};
			\draw [ line width=0.5pt](opamp2.+) to[short] (5,9.75);
			\draw [ line width=0.5pt] (opamp2.-) to[short] (5,10.75);
			\draw [ line width=0.5pt](7.7,10.25) to[short](8,10.25);
			\draw (8,10.25) to[short, -o] (8.25,10.25) node[] {$\ \ \ \ \ \ \ \ \ V_{out}$};
			\draw (7.5,11) to[short] (7.5,10.25);
			\draw [line width=1pt](5,9.75) to[C,l={ \normalsize $C_{df}$}] (5,10.75);
			\draw [line width=1pt](-0.25,9) to[C,l={ \normalsize $C_{d}$}](-0.25,10.5);
			\draw (0.25,10.5) to[short] (1.5,10.5);
			\draw (0.25,9) to[short] (1.5,9);
			\draw (3.75,10.75) to[short] (5,10.75);
			\draw [line width=1pt](3.75,8) to[C,l={ \normalsize $C_{cm}$}](3.75,10.75);
			\draw (3.75,8.5) to (3.75,8.25) node[ground]{};
			\draw [ line width=0.5pt](6.5,10.25) node[op amp,scale=1](opamp2) {};
			\draw [ line width=0.5pt](opamp2.+) to[short] (5,9.75);
			\draw [ line width=0.5pt] (opamp2.-) to[short] (5,10.75);
			\draw [ line width=0.5pt](7.7,10.25) to[short](8,10.25);
			\node at (3.75,10.75) [circ] {};
			\node at (5,10.75) [circ] {};
			\node at (5,9.75) [circ] {};
			\node at (1.5,9) [circ] {};
			\node at (1.5,10.5) [circ] {};
			\node at (7.5,10.25) [circ] {};
			\draw (-0.25,10.5) to[short] (0.25,10.5);
			\draw (-0.25,9) to[short] (0.25,9);
			\draw (2.75,10.75) to[short] (3.75,10.75);
		\end{circuitikz}
	}%
	\caption{Parasitic capacitances in a \gls{tia} circuit}
	\label{fig:td:tia_circuit_parasitic_v1}
\end{figure}

Introducing the real-world parasitic capacitances to the ideal \gls{tia} shown in Figure \ref{fig:tia} results in the circuit in Figure \ref{fig:td:tia_circuit_parasitic_v1}. The diode capacitance $C_d$, combined with the differential and common-mode capacitances of the amplifier ($C_{df}$ and $C_{cm}$), contributes a significant amount of capacitance to the circuit and can result in instability \cite{cherian2016you}.

\begin{equation}\label{eq:td:ci}
	C_i = C_{d} + C_{df} + C_{cm}
\end{equation}

The combined input capacitance (Equation \eqref{eq:td:ci}) is needed when solving the equation arising from \gls{kcl} at the inverting input of the amplifier. $Z_1$ (Equation \eqref{eq:td:z1}), or the combined impedance of $R_1$ and $C_1$, is another prerequisite for deriving the transimpedance $Z_t$ using \gls{kcl}. Additionally, the open-loop gain $A_{ol}$ is needed for the derivation of the transimpedance. Equation \eqref{eq:td:aol} shows the open-loop gain, expressed with the \gls{dc} open-loop gain $A_o$ and the open-loop cutoff frequency $\omega_o$, which is equivalent to $\frac{v_{out}}{v_{in}}$.

\begin{equation}\label{eq:td:z1}
	Z_1(s) = \frac{1}{\frac{1}{R_1} + sC_1}
\end{equation}

\begin{equation}\label{eq:td:aol}
	A_{ol}(s) = A_o\frac{\omega_0}{s + \omega_0} = \frac{v_{out}}{v_{in}}
\end{equation}

Finally, the values in the \gls{kcl} equation (Equation \eqref{eq:td:kcl}) can be substituted, from which $Z_t$ can be derived, as seen in Equation \eqref{eq:td:zt}. 

\begin{equation}\label{eq:td:kcl}
	i_f = \frac{v_i}{\frac{1}{sC_i}} + \frac{v_{in} - v_{out}}{Z_1(s)}
\end{equation}

\begin{equation}\label{eq:td:zt}
	\begin{aligned}
	&Z_t(s) = \frac{v_{out}}{i_f} = (R_1\frac{A_o}{1 + A_o})\frac{\frac{\omega_o(1 + A_o)}{(C_i + C_1)R_1}}{s^2 + s\frac{1 + \omega_o(C_i + (1 + A_o)C_1)R_1}{(C_i + C_1)R_1} + \frac{\omega_o(1 + A_o)}{(C_i + C_1)R_1}}
	\end{aligned}
\end{equation}

Before moving on to the pole analysis, the impact of the quality factor $Q$ on the system response should be considered (see Equation \eqref{eq:td:f_c_q}). Two system poles present when $Q > 0,5$. For $Q = \frac{\sqrt{3}}{3},$ the system has a Bessel response, which has the flattest group delay, however, if $Q = \frac{\sqrt{2}}{2}$, then the system has a Butterworth response, which means it has the flattest possible amplitude response. Although the group delay is not as important as the amplitude response, a Bessel response also results in smaller amount of overshoot and less jitter, which is why it is preferred. In the end, the choice between a Bessel and a Butterworth response only affects the circuit at high frequencies, which is not a concern when running \gls{cwodmr} measurements. Still, it can be important when investigating photodetection for pulsed protocols.

\begin{equation}\label{eq:td:poles_general}
	H(s) = H_0\frac{(-s_1)(-s_2)}{(s - s_1)(s - s_2)} = H_0\frac{s_1s_2}{s^2 + s(-s_1 - s_2) + s_1s_2}
\end{equation}

Based on this consideration, the poles can be found using the general form of a second order transfer function, shown in Equation \eqref{eq:td:poles_general}. The general form also makes it obvious that the \gls{dc} gain $H_0 \approx R_1$, assuming that $A_0$ is very big.


\begin{equation}\label{eq:td:poles}
	\begin{cases}
		-s_1 - s_2 = \frac{1 + \omega_o(C_i + (1 + A_o)C_1)R_1}{(C_i + C_1)R_1} \\
		s_1s_2 = \frac{\omega_o(1 + A_o)}{(C_i + C_1)R_1}
	\end{cases}
\end{equation}

Taking the general-equation poles and their counterparts from the $Z_t$ expression yields the system of equations shown in Equation \eqref{eq:td:poles}. Solving it results in a quadratic equation with roots shown in Equation \eqref{eq:td:pole_roots}.

\begin{equation}\label{eq:td:pole_roots}
	s_{1, 2} = -\frac{1 + \omega_o(C_i(1+A_o)C_1)R_1}{2(C_i + C_1)R_1}\left(1 \pm j\sqrt{\frac{4\omega_o(1 + A_o)(C_i + C_1)R_1}{(1 + \omega_o(C_i + (1 + A_o)C_1)R_1)^2} - 1}\right)
\end{equation}

\begin{equation}\label{eq:td:pole_roots_general}
	s_{1, 2} = -\frac{3}{2}(1 \pm j\frac{\sqrt{3}}{3})
\end{equation}

Achieving the desired system, can be done by getting the roots of the general form of a Bessel filter (as seen in Equation \eqref{eq:td:pole_roots_general}) and using the imaginary parts to get Equation \eqref{eq:td:cap_solve}. From it, a the quadratic equation in Equation \ref{eq:td:cap_quad} can be derived. This expression can be solved with the quadratic formula again, finally resulting in an answer for the compensating capacitor $C_1$. 

\begin{equation}\label{eq:td:cap_solve}
	\frac{4\omega_o(1 + A_o)(C_i + C_1)R_1}{(1 + \omega_o(C_i + (1 + A_o)C_1)R_1)^2} = \frac{4}{3}
\end{equation}

\begin{equation}\label{eq:td:cap_quad}
	C_1^2(R_1(1 + A_o))^2 + C_1\omega_o(1 - C_i)((1 + A_o)R_1) - 3\omega_o(1 + A_o)C_iR_1 + C_i^2 + 2C_i + 1 = 0
\end{equation}

\begin{equation}\label{eq:td:cap_roots}
	C_1 = \frac{-2\omega_oR_1C_i \pm \sqrt{12A_oC_iR_1\omega_o - 3}}{2\omega_oR_1(1 + A_o)}
\end{equation}

Although the quadratic formula in Equation \eqref{eq:td:cap_roots} says there are two solutions for the capacitor, the term under the square root will always be much bigger than the rest of the numerator\footnote{Assuming real-world amplifier and diode specifications, and realistic transimpedance gain requirements}, effectively determining the sign of the roots. Simply put, this leaves only one root possible, as capacitors cannot have negative capacitance values.

To further characterize the system, the cutoff frequency $\omega_c$ and damping factor $Q$ can be calculated, because the system adheres to the general form, shown in Equation \eqref{eq:td:f_c_q}. Substituting in the pole expressions from Equations \eqref{eq:td:poles_general} and \eqref{eq:td:poles}.

\begin{equation}\label{eq:td:f_c_q}
	H(s) = H_o\frac{\omega_c^2}{s^2 + s\frac{\omega_c}{Q} + \omega_c^2}
\end{equation}

\begin{equation}
	\begin{cases}
		\omega_c = \sqrt{\frac{\omega_o(1 + A_o)}{(C_i + C_1)R_1}}\\
		Q = \omega_c\frac{(C_i + C_1)R_1}{1 + \omega_o(C_i + (1 + A_o)C_1)R_1}
	\end{cases}
\end{equation}


In the process of designing the \gls{tia} circuit, the noise should be considered. Figure \ref{fig:td:design:tia_noise} shows the thermal noise generated by the resistor (modeled as the current source $i_{nR}$) and the differential noise of the amplifier (modeled as the current source $i_{nA}$)\footnote{Other noise generators, like the gate shot noise, channel noise and induced gate noise can also be calculated, but because they largely depend on the underlying technology of the \gls{fet} op amp, they have not been investigated}.

\begin{figure}[!ht]
	\centering
	\resizebox{.7\textwidth}{!}{%
		\begin{circuitikz}
			\tikzstyle{every node}=[font=\normalsize]
			\draw [ line width=0.5pt](1.5,9) to[empty photodiode,l={ \normalsize $D_1$}] (1.5,10.75);
			\draw [ line width=0.5pt](6.5,10.25) node[op amp,scale=1] (opamp2) {};
			\draw [ line width=0.5pt](opamp2.+) to[short] (5,9.75);
			\draw [ line width=0.5pt] (opamp2.-) to[short] (5,10.75);
			\draw [ line width=0.5pt](7.7,10.25) to[short](8,10.25);
			\draw [line width=0.5pt](5,9) to (5,8.25) node[ground]{};
			\draw [ line width=0.5pt](5,8.5) to[short] (5,9.75);
			\draw [ line width=0.5pt](1.5,10.5) to[short] (1.5,10.75);
			\draw [ line width=0.5pt](1.5,10.75) to[short] (2.75,10.75);
			\node at (4.5,10.75) [circ] {};
			\draw [ line width=0.5pt](4.5,10.75) to[short] (4.5,13.5);
			\draw [ line width=0.5pt](4.5,12.25) to[short] (5,12.25);
			\draw [ line width=0.5pt](4.5,13.5) to[short] (5,13.5);
			\draw [ line width=0.5pt](5,12.25) to[european resistor,l={ \normalsize $R_1$}] (7,12.25);
			\node at (4.5,12.25) [circ] {};
			\draw [ line width=0.5pt](7,12.25) to[short] (7.5,12.25);
			\draw [ line width=0.5pt](7.5,12.25) to[short] (7.5,11);
			\draw [ line width=0.5pt](7.5,12.25) to[short] (7.5,13.5);
			\draw [line width=0.5pt](5,13.5) to[C,l={ \normalsize $C_1$}] (7,13.5);
			\node at (7.5,12.25) [circ] {};
			\draw [ line width=0.5pt](7,13.5) to[short] (7.5,13.5);
			\draw [ line width=0.5pt](1.5,10.75) to[short] (1.5,10.5);
			\draw [line width=0.5pt](1.5,9) to (1.5,8.25) node[ground]{};
			\draw [line width=0.5pt, ->, >=Stealth] (2,10.25) -- (2,9.25);
			\node [font=\normalsize] at (2.25,9.75) {$I_f$};
			\draw [ line width=0.5pt](6.5,10.25) node[op amp,scale=1] (opamp2) {};
			\draw [ line width=0.5pt](opamp2.+) to[short] (5,9.75);
			\draw [ line width=0.5pt] (opamp2.-) to[short] (5,10.75);
			\draw [ line width=0.5pt](7.7,10.25) to[short](8,10.25);
			\draw (8,10.25) to[short, -o] (8.25,10.25) node[] {$\ \ \ \ \ \ \ \ \ V_{out}$};
			\draw (7.5,11) to[short] (7.5,10.25);
			\draw (3.75,10.75) to[short] (5,10.75);
			\draw [ line width=0.5pt](6.5,10.25) node[op amp,scale=1] (opamp2) {};
			\draw [ line width=0.5pt](opamp2.+) to[short] (5,9.75);
			\draw [ line width=0.5pt] (opamp2.-) to[short] (5,10.75);
			\draw [ line width=0.5pt](7.7,10.25) to[short](8,10.25);
			\node at (7.5,10.25) [circ] {};
			\draw (2.75,10.75) to[short] (3.75,10.75);
			\draw [ line width=0.7pt](4.5,15) to[american current source,l={ \normalsize $i_{nR}$}] (7.5,15);
			\draw (4.5,13.5) to[short] (4.5,15);
			\draw (7.5,15) to[short] (7.5,13.5);
			\node at (4.5,13.5) [circ] {};
			\node at (7.5,13.5) [circ] {};
			\draw [ line width=0.7pt](3.75,10.75) to[american current source,l={ \normalsize $i_{nA}$}] (3.75,8.75);
			\draw (3.75,8.75) to (3.75,8.25) node[ground]{};
		\end{circuitikz}
	}%
	\caption{Noise current sources in a \gls{tia} circuit}
	\label{fig:td:design:tia_noise}
\end{figure}

Calculating $i_{nR}$ can be done using the Johnson-Nyquist formula, as shown in Equation \eqref{eq:td:noise_res}. Given the fact that the temperature $T$ and noise cutoff frequency $\delta{f}$ cannot be modified in this case, $R_1$ is  the only variable that can be used to lower the thermal noise current.

\begin{equation}\label{eq:td:noise_res}
	i_{nR} = \sqrt{\frac{4kT\Delta{f}}{R_1}}
\end{equation}

Combining the noise sources leads to Equation \eqref{eq:td:noise_sources}, which also shows that the noise can be modeled as voltage sources.  

\begin{equation}\label{eq:td:noise_sources}
		i_n(f) = \sqrt{i_{nR}^2 + i_{nA}^2} \Leftrightarrow
		v_n(f) = \sqrt{v_{nR}^2 + v_{nA}^2}
\end{equation}

The noise gain is the ratio of the output noise and the input noise. This ratio gives a better understanding of what frequencies of noise are the most problematic if they appear at the input of the \gls{tia}. In order to get a better understanding of the noise gain, its transfer function can be calculated in much the same way as the transimpedance (Equations \ref{eq:td:aol} - \ref{eq:td:zt}), leading to Equation \eqref{eq:td:noise_tf}. The transfer function shown in it can be broken down into two components: a low-pass and a band-pass component. In the numerator expression $sA_o\omega_o$ contributes the band-pass component and the expression $\frac{A_o\omega_o}{(C_i + C_1)R_1}$ is responsible for the low-pass component. Figure \ref{fig:td:noiseexample} shows an example of the frequency response of the noise transfer function.

\begin{equation}\label{eq:td:noise_tf}
	H_n(s) = \frac{v_o}{v_n} = \frac{A_o\omega_o(s + \frac{1}{(C_i + C_1)R_1})}{s^2 + s(\frac{1}{(C_i + C_1)R_1} + \omega_o(1 + A_o\frac{C_1}{C_i + C_1})) + \frac{\omega_o(1 + A_o)}{(C_i + C_1)R_1}}
\end{equation}


\begin{figure}[ht]
	\centering
	\includegraphics[width=0.6\linewidth]{matlab/noise_example}
	\caption{Plot of the components of the noise and the total noise}
	\label{fig:td:noiseexample}
\end{figure}


While it might not be immediately obvious from the mathematical expression, the noise gain is \num{0} \unit{\decibel} at low frequencies, but it peaks when $f \approx f_c$. 

Firstly, the low-frequency response of the system can most easily be demonstrated by calculating it at a frequency $\omega$ = 0 \unit{\radian\per\second}. This gives us the \gls{dc} gain shown in Equation \eqref{eq:td:noise_dc}, which is close to 1 for real amplifiers, in which the gain is finite.

\begin{equation}\label{eq:td:noise_dc}
	H_n(0) = \frac{\frac{A_o\omega_o}{(C_i + C_1)R_1}}{\frac{\omega_o(1 + A_o)}{(C_i + C_1)R_1}} = \frac{A_o}{A_o + 1}
\end{equation}


\begin{figure}[ht]
	\centering
	\includegraphics[width=0.6\linewidth]{matlab/noise_example_vs_signal}
	\caption{Comparison between the noise frequency response and the normalized signal frequency response}
	\label{fig:td:noiseexamplevssignal}
\end{figure}

Secondly, the peak is caused by the combination of both components. The center frequency of the band-pass component corresponds to the cutoff frequency of the low-pass, which makes the gain at that frequency the highest. What is more, the peak-gain frequency of the noise matches the signal transfer function cutoff, which can be seen in Figure \ref{fig:td:noiseexamplevssignal}.




\subsection{Third iteration}
While the second version of the photodetector focuses on improving the TIA performance, the third iteration is more about adapting the design for the setup. In particular, this iteration tries to simplify the power delivery and optimize the size of the design, while keeping the amplifying circuit unchanged. In spite of the added power management, this version of the system is made so that it still costs around the same as the previous ones.

The most important considerations for the setup are function and size. Form factor constraints arise from the fact that the setup needs to be as small and adhere to the Thorlabs mounting standard. Photodetection also needs to not require many external connections or ideally be standalone. This is feasible for the data flow, as the photodetection \gls{pcb} only needs light and an output to work. Unlike the data connections, version two of the design has a suboptimal power delivery system, as it uses three pins for the supply of power. 

Three alternatives methods for power delivery were considered, all of which only need power and ground to work. The first one is an onboard split-rail power supply, which would take the power input and would output a fixed positive and negative voltage. Figure \ref{fig:td:des:adafruitsplitrailboost} shows an example of what a split-rail supply implementation might look like. There are also external option \cite{clark2025boost}, which might be necessary in the future, if the setup size needs to be reduced even more.  

\begin{figure}[ht]
	\centering
	\includegraphics[width=0.4\linewidth]{img/adafruit_split_rail_boost}
	\caption{Example of a split-rail power supply implementation using the TPS65133 \glsfmtshort{ic} (image credit to \cite{ti2015tps65133})}
	\label{fig:td:des:adafruitsplitrailboost}
\end{figure}


A simpler solution would be to use a virtual ground driver \gls{ic}, as it would provide a reference point at half of the supply voltage. Such a supply would also be good for decreasing the size, as the \gls{pcb} needs minimal additions to satisfy the power needs of the photodetector. Additionally, a virtual ground circuit, which behaves similarly, can be made using two voltage regulators. The example in Figure \ref{fig:td:des:virtground} is a slightly more stable implementation


\begin{figure}[ht]
	\centering
	\includegraphics[width=0.3\linewidth]{img/virt_ground}
	\caption{Example of a virtual ground driver using the TLE2426 \glsfmtshort{ic} (image credit to \cite{ts2022ground})}
	\label{fig:td:des:virtground}
\end{figure}


Another possibility is to implement a charge pump (see Figure \ref{fig:td:des:chrgpump}), which inverts the input voltage. This setup would only need half of the input voltage required for the virtual ground solution, effectively enabling it to be powered via a standard \gls{usb} connector.

\begin{figure}[ht]
	\centering
	\includegraphics[width=0.5\linewidth]{img/chrg_pump}
	\caption{Example of a charge pump inverter using a TPS6040x family \glsfmtshort{ic} (image credit to \cite{instruments2015tps6040x})}
	\label{fig:td:des:chrgpump}
\end{figure}



\begin{table}[ht]
	\centering
	\begin{tabular}{|c||c|c|c|}
		\hline
		& Split-rail power supply 	& Virtual ground & Charge pump  \\
		\hline
		\glsfmtshort{ic}					& TPS65133		 			& TLE2426 	     & TPS60400 	\\
		\hline
		$V_{in}$ (\unit{\volt}) 			& 5  			 			& 10 	   		 & 5  			\\
		\hline
		$V_{out}$ (\unit{\volt})			& \textpm 5 				& \textpm 5 	 & -5 			\\
		\hline
		$I_{out}$ (\unit{\milli\A})			& 250  						& 40 			 & 60 			\\
		\hline
		Current draw (\unit{\mu\A})			& 15						& 300 	 		 & 210  \\
		\hline
	\end{tabular}
	\caption{Comparison of power supply options}
	\label{tab:power_supply}
\end{table}

The specifications of the different power supply solutions are shown in Table \ref{tab:power_supply}, from which can be deduced that there is a correlation between complexity and performance. The split-rail power supply, as the most complex solution, has the biggest current output range and lowest current consumption, but as illustrated in Figure \ref{fig:td:des:adafruitsplitrailboost}, there are a number of other components that are required for proper implementation. In contrast, the virtual ground supply concept has relatively poor specifications compared to the other options, but it is the only solution that does not employ a switching mechanism, thus avoiding any possible noise. Lastly, the charge pump inverter offers reasonable performance, while not needing as many extra components as the split-rail supply. Furthermore, it addresses the stability concerns that might arise when using a virtual ground \gls{ic}. 

Ultimately, the charge pump is the most fit for this project, because it is relatively simple, while also offering excellent stability. The decision was reached only after taking into account the power calculations from Chapter \ref{chap:fd:power_req}, according to which all devices can handle the power requirements of the circuit. Another important factor when making the decision was the efficiency of the switching solutions. Based on slight fluctuations in the amount of current drawn, the charge pump can have as much as twice the efficiency of the split-rail power supply \cite{instruments2015tps6040x, ti2015tps65133}, due to its high efficiency at low currents.

The design in Figure \ref{fig:td:des:chrgpump} is recommended by the manufacturer. Unlike the flying capacitor $C_{fly}$, which does not affect the behavior of the \gls{ic}, both the input $C_I$ and output $C_O$ capacitors have to be chosen carefully. Too low of an input or output capacitance can cause an unwanted ripple. What is more, the \gls{esr} of $C_O$ also has a significant effect on the output ripple, as is evident from Equation \eqref{eq:td:v3:ripple} \cite{instruments2015tps6040x}. In it, $I_O$ is the output current and $f_{osc}$ is the switching frequency of the pump.

\begin{equation}\label{eq:td:v3:ripple}
V_{Or} = \frac{I_O}{f_{osc}C_O} + 2I_O(ESR_{CO})
\end{equation}

\subsection{Fourth iteration}\label{chap:td:v4_design}
Although the previous iterations provided a cheap and reliable photodetector for \gls{cwodmr}, the narrow bandwidth prevents them from measuring pulsed protocols. This iteration is specifically designed for $T_1$ measurement support. Although it is somewhat more expensive than the previous iteration, version four justifies the price by offering significant sensing performance gains.

\begin{figure}[ht]
	\centering
	\includegraphics[width=0.6\linewidth]{matlab/bode_v4_v3_10Mohm_1Mohm}
	\caption{Bandwidth comparison of the third iteration of the \gls{tia} (based on AD795) and one based on an ADA4637 op-amp with a 10 \unit{\mega\ohm} and a 1 \unit{\mega\ohm} $R_1$}
	\label{fig:td:bodev410mohm1mohm}
\end{figure}

The simplest way to increase the bandwidth of the \gls{tia} is to decrease its transimpedance gain. Additionally, an op-amp with a higher \gls{gbp} will also contribute to a substantial increase in gain. Figure \ref{fig:td:bodev410mohm1mohm} shows the ideal version of the \gls{tia}, used in the second and third iteration of the system (in blue). It uses an AD795 amplifier, which has a 1,6 \unit{\mega\hertz} \gls{gbp}. The resulting cutoff frequency is \num{19,69} \unit{\kilo\hertz}. Only swapping the AD795 for an ADA4637, which has a 79 \unit{\mega\hertz} \gls{gbp}, results in the system response in orange. Its cutoff frequency is \num{138,87} \unit{\kilo\hertz}. Both lines show a system with the same transimpedance gain and consequently $R_1$ value. On the other hand, the yellow line shows the increase in bandwidth with a 10 times smaller gain\footnote{The plot shows normalized gain of each system, which is why the first two systems (140 \unit{\decibel}) have the same gain as the last system (120 \unit{\decibel})}. The ADA4637 system with 1 \unit{\mega\ohm} $R_1$ has a cutoff frequency of \num{437,32} \unit{\kilo\hertz}.


While these changes contribute to an improved bandwidth, they are limited by the compensation capacitor $C_1$. Its purpose is to counteract the negative effects of the diode capacitance $C_d$ and stabilize the frequency response of the system. In doing so, however, it also reduces the bandwidth of the system. Decreasing the capacitance of $C_1$ outright can lead to unwanted resonance, but it is possible to stabilize $C_d$ and consequently reduce the value of $C_1$ by using bootstrapping. The design discussed in this section is based on \cite{sackinger2017analysis, hoyle1998bootstrapping, hoyle1999shunt, idrus2006performance}.



\begin{figure}[ht]
	\centering
	\resizebox{.5\textwidth}{!}{%
		\begin{circuitikz}
			\tikzstyle{every node}=[font=\normalsize]
			\draw [ line width=0.5pt](3.75,9) to[empty photodiode,l={ \normalsize $D_1$}] (3.75,10.75);
			\draw [ line width=0.5pt](6.5,10.25) node[op amp,scale=1] (opamp2) {};
			\draw [ line width=0.5pt](opamp2.+) to[short] (5,9.75);
			\draw [ line width=0.5pt] (opamp2.-) to[short] (5,10.75);
			\draw [ line width=0.5pt](7.7,10.25) to[short](8,10.25);
			\draw [line width=0.5pt](5,9) to (5,8.25) node[ground]{};
			\draw [ line width=0.5pt](5,8.5) to[short] (5,9.75);
			\draw [ line width=0.5pt](3.75,10.5) to[short] (3.75,10.75);
			\draw [ line width=0.5pt](3.75,10.75) to[short] (5,10.75);
			\node at (4.5,10.75) [circ] {};
			\draw [ line width=0.5pt](4.5,10.75) to[short] (4.5,13.5);
			\draw [ line width=0.5pt](4.5,12.25) to[short] (5,12.25);
			\draw [ line width=0.5pt](4.5,13.5) to[short] (5,13.5);
			\draw [ line width=0.5pt](7.5,10.25) to[short] (7.5,11);
			\draw [ line width=0.5pt](5,12.25) to[european resistor,l={ \normalsize $R_1$}] (7,12.25);
			\node at (4.5,12.25) [circ] {};
			\draw [ line width=0.5pt](7,12.25) to[short] (7.5,12.25);
			\draw [ line width=0.5pt](7.5,12.25) to[short] (7.5,11);
			\draw [ line width=0.5pt](7.5,12.25) to[short] (7.5,13.5);
			\draw [line width=0.5pt](5,13.5) to[C,l={ \normalsize $C_1$}] (7,13.5);
			\node at (7.5,12.25) [circ] {};
			\draw [ line width=0.5pt](7,13.5) to[short] (7.5,13.5);
			\draw [ line width=0.5pt](8,10.25) to[short, -o] (8.25,10.25) node {$\ \ \ \ \ \ \ \ \ V_{out}$};
			\draw [ line width=0.5pt](3.75,10.75) to[short] (3.75,10.5);
%			\draw [line width=0.5pt](3.75,9) to (3.75,8.25) node[ground]{};
			\draw [line width=0.5pt, ->, >=Stealth] (4.25,9.25) -- (4.25,10.25);
			\node [font=\normalsize] at (4.60,9.75) {$I_f$};
			\draw (1.25,12) node[op amp,scale=1, xscale=-1 , yscale=-1] (opamp2) {};
			
			% second amp and feedback
			\draw [line width=0.5pt] (opamp2.+) to [short] (4.5, 12.5)
			(opamp2.-) to [short] (2.5, 11.5)
			to [short] (2.5, 10.5)
			to [short] (0.25, 10.5)
			to [short] (0.25, 12);
			\node at (0.25, 12) [circ] {};
			% connection to diode
			\draw [line width=0.5pt] (0.5, 12) to [short] (0,12)
			to [short] (0, 9)
			to [short] (3.75, 9);
		\end{circuitikz}
	}%
	\caption{Shunt bootstrap \gls{tia} circuit}
	\label{fig:td:tia_bootstrap}
\end{figure}

Figure \ref{fig:td:tia_bootstrap} shows the bootstrapped \gls{tia}. The same analysis as in Chapter \ref{chap:td:calc:v2} can be applied to the circuit, which leads to the transimpedance $Z_t$ in Equation \eqref{eq:td:v4:trans}. The additional op-amp buffer creates an extra pole and zero. In the expression, $C_i$ is the sum of the common-mode and differential capacitance of the amplifier, $C_d$ is the photodiode capacitance, $A_o$ is the intrinsic open-loop gain of the amplifier, $\omega_o$ is the open-loop cutoff frequency and all other parameters correspond to the ones in Figure \ref{fig:td:tia_bootstrap}.

\begin{equation}\label{eq:td:v4:trans}
	Z_t(s) = \frac{v_{out}}{i_f} = \frac{R_1}{s^2\frac{\left(C_i + C_1 + C_d\right)R_1}{\omega_oA_o} + s\left(\frac{\left(C_i + C_1 + C_d\right)R_1}{A_o} + C_1R_1 + \frac{1}{\omega_oA_o}\right) - \frac{s\left(s + \omega_o\right)C_dR_1}{s + \omega_o(1 + A_o)} + \frac{A_o + 1}{A_o}}
\end{equation}

Determining the ideal $C_1$ by solving the characteristic equation of the system analytically is even harder than in the previous designs, which is why the system equation was solved by trial and error, and with the use of MATLAB to plot the system response with different capacitor values. The solution is discussed in more detail in Chapter \ref{chap:td:v4_sim}.

While the improvements to the \gls{tia} are crucial for broadening its bandwidth, the changes also affect the post-amplification stage. It needs to compensate for the gain reduction that was previously discussed. Despite, a simple gain increase being a viable solution, this will also reduce the bandwidth. Similarly to the \gls{tia}, the non-inverting amplifier is also affected by the gain-bandwidth relationship, as dictated by the \gls{gbp}, shown in Equation \eqref{eq:td:v4:gbp+gain}, where $A_v$ is the gain of the stage.

\begin{equation}\label{eq:td:v4:gbp+gain}
	f_c = \frac{GBP}{A_v}
\end{equation}

An alternative to increasing the gain is to add a second non-inverting stage with the same gain. By doing this, the bandwidth does not decrease as a result of the gain-bandwidth relationship, but the cutoff frequency $f_{c\alpha}$ of single-stage non-inverting amplifier even increases as shown in Equation \eqref{eq:td:v4:nia_cascading}\footnote{The formula assumes the non-inverting stages have the same parameters and use the same components as in the actual design} \cite{ti2020cascading}. In the equation, $A_{df}$ is the bandwidth increase due to the single-stage gain reduction.

\begin{equation}\label{eq:td:v4:nia_cascading}
	f_c \approx A_{df}\left(\frac{1}{f_{c\alpha}^2} + \frac{1}{f_{c\alpha}^2}\right)^{-\frac{1}{2}} = \frac{A_{df}}{\sqrt{2}}f_{c\alpha}
\end{equation}

It should also be noted that the equivalent resistance of the gain-setting resistors, combined with the parasitic capacitances of the op-amp, create a zero, the impact of which can be reduced by decreasing the value of the resistors. This can be observed from the relationship of the resistance to the phase margin, as seen in Equation \eqref{eq:td:v4:nia_cascading:r3r2}. In it, resistors $R_2$ and $R_3$ correspond to the ones in Figure \ref{fig:tia_nia}. $C_{cm}$ and $C_{df}$ denote the common-mode and differential capacitance of the op-amp.

\begin{equation}\label{eq:td:v4:nia_cascading:r3r2}
	\frac{1}{2\pi\left(C_{cm} + C_{df}\right)\frac{R_2R_3}{R_2 + R_3}} > \frac{GBP}{1 + \frac{R_3}{R_2}}
\end{equation}

It is possible to simplify the formula, because when designing the amplifier, the ratio of the resistors is fixed ($R_3 = A_rR_2$, where $A_r$ is the resistor ratio). Solving for $R_2$ then gives Equation \ref{eq:td:v4:nia_cascading:r2_simplified}.

\begin{equation}\label{eq:td:v4:nia_cascading:r2_simplified}
	R_2 < \frac{(1 + A_r)^2}{2\pi A_r\left(C_{cm} + C_{df}\right)GBP} 
\end{equation}

%\begin{equation}\label{eq:td:v4:trans_re}
%	\begin{split}
%	Z_t(s) = \frac{sR_1 + R_1\omega_o(1 + A_o)}{s ^3(\frac{(C_i + C_1 + C_d)R_1}{A_o}) + s^2(\frac{2 + A_o}{A_o}(C_i + C_1 + C_d)R_1 + (C_1 - C_d)R_1 + \frac{1}{\omega_oA_o})} + \\ 
%	\frac{sR_1 + R_1\omega_o(1 + A_o)}{s(\omega_o\frac{1 + A_o}{A_o}(C_i + C_1 + C_d)R_1  + \omega_o(1 + A_o)C_1R_1 - \omega_oC_dR_1 + 2\frac{1 + A_o}{A_o}) + \omega_o(A_o + 1)\frac{A_o + 1}{A_o}}
%	\end{split}
%\end{equation}

%\begin{equation}\label{eq:td:v4:trans}
%	Z_t(s) = \frac{v_{out}}{i_f} = \frac{R_1}{s^2\frac{(C_i + C_1 + C_d)R_1}{\omega_o} + s(\frac{(C_i + C_1 + C_d)R_1\omega_a}{\omega_o} + C_1R_1 + \frac{1}{\omega_o}) - \frac{s(s + \omega_a)C_dR_1}{s + \omega_o + \omega_a} + \frac{\omega_a}{\omega_o} + 1}
%\end{equation}

Finally, Figure \ref{fig:td:v4schematic} shows the schematic of the system after all the modifications. Because of noise concerns, the resistor values were reduced significantly, which, as previously mentioned, contributes to lower noise in the post-amplification stage.

\begin{figure}[ht]
	\centering
	\includegraphics[width=0.7\linewidth]{img/v4_schematic}
	\caption{Schematic of the fourth iteration of the \gls{tia}}
	\label{fig:td:v4schematic}
\end{figure}

\subsection{Final \glsfmtshort{cwodmr} iteration}\label{chap:td:v5_design}
At the beginning of the project, the client wanted an all-encompassing solution for the detection of all protocols. Although this is possible, as is clear from the photodetectors on the market, such a solution can cost in the range of hundreds to thousands of euros. Bringing costs down is important for the client, which is why it was decided that separate photodetectors for \gls{cwodmr} and \gls{podmr} were needed.

Detecting \gls{cwodmr} is possible with a much lower bandwidth than pulsed protocols, which removes the need for post amplification and thus brings down the overall noise of the system, as well as its noise susceptibility. This is good, because the client is interested in having a portable \gls{cwodmr} demonstrator and reducing noise susceptibility in such a noisy environment is crucial. In the end, the design of this system seeks to minimize noise as much as possible, without necessarily considering the bandwidth.

As the second iteration of the photodetector has a serviceable design, the final version of the photodetector is based on it. The first and most important change for reducing the overall noise of the system is the removal of the non-inverting stage, which necessarily results in an increase of the transimpedance gain. Removing the post amplifier contributes most of the noise reduction, but the transimpedance gain increase, which is equivalent to the value of the gain-setting resistor of the \gls{tia}, also reduces its thermal noise, according to Formula \eqref{eq:td:noise_res}.

Furthermore, significant noise reduction can be brought about by making changes to the \gls{tia}. The op-amp choice is the most impactful of all. The OPA827 was chosen, due to its excellent noise characteristics. Other alternatives, like the OPA1656, AD743 and ADA4627 were considered as well, because of their good noise characteristics, but their low-frequency spectral noise density is higher than the OPA827. \footnote{ADA4627: 4,8 \unit[power-half-as-sqrt]{\nano\volt\per\hertz\tothe{0.5}} at 10 \unit{\kilo\hertz}, but 16,5 \unit[power-half-as-sqrt]{\nano\volt\per\hertz\tothe{0.5}} at 10 \unit{\hertz}; OPA1656: 2,9 \unit[power-half-as-sqrt]{\nano\volt\per\hertz\tothe{0.5}} at 10 \unit{\kilo\hertz}, but 37 \unit[power-half-as-sqrt]{\nano\volt\per\hertz\tothe{0.5}} at 10 \unit{\hertz}; AD743: 2,9 \unit[power-half-as-sqrt]{\nano\volt\per\hertz\tothe{0.5}} at 10 \unit{\kilo\hertz}, but 22 \unit[power-half-as-sqrt]{\nano\volt\per\hertz\tothe{0.5}} at 1 \unit{\hertz}; OPA827: 3,8 \unit[power-half-as-sqrt]{\nano\volt\per\hertz\tothe{0.5}} at 10 \unit{\kilo\hertz} and 7,1 \unit[power-half-as-sqrt]{\nano\volt\per\hertz\tothe{0.5}} at 10 \unit{\hertz}}.
In addition, the LT1128 was also explored as a possible alternative, as its noise performance is even better than the OPA827. However, it has a significantly higher input bias current\footnote{25 \unit{\nano\ampere} compared to the 10 \unit{\pico\ampere} of the OPA827}, which will affect the photoconductive-mode diode and thus substantially increase its noise. Lower input bias current is the reason why the ADA4627 is the best alternative to the OPA827, despite its somewhat worse noise characteristics.

Finally, the compensation capacitor of the \gls{tia} can also be adjusted to reduce unwanted noise. Figure \ref{fig:noisev5} shows the model of the voltage gain of the noise. The large spike created by the band-pass component can be filtered out by increasing the capacitance, thus substantially reducing the high-frequency noise of the system. For the final design, various standardized values of increasing capacitance were tested with simulation software, until a fitting value was found.

\begin{figure}[ht]
	\centering
	\includegraphics[width=0.7\linewidth]{matlab/noise_v5}
	\caption{Theoretical noise model of the OPA827-based \gls{tia}}
	\label{fig:noisev5}
\end{figure}

\section{Photodetection simulation}
Simulation is an important part of the designing process, as digital models of devices can be more detailed than the mathematical models presented thus far. Using different simulation programs, the real-world performance can be approximated and the math can be validated.


\subsection{First iteration}
\begin{figure} [!ht]
	\centering
	\includegraphics[width=0.8\linewidth]{img/TIA_TINA_SIM_1P6}
	\caption{First iteration of the photodetector circuit}
	\label{fig:photodetecog}
\end{figure}


Figure \ref{fig:photodetecog} shows the schematic of the circuit. The capacitor $C_4$, as well as the current source $I_1$ are used to simulate the behavior of a photodiode.

Tina-TI was used to simulate and visualize the DC gain and frequency response of the system, as seen in Figure \ref{fig:tina2probes}. The signals $V_{ot}$ and $V_{out}$ correspond to the output of the transimpedance and non-inverting amplification stage respectively. The AC plot (Figure \ref{fig:tinaac2probes}) shows the cutoff, at \num{10,77} \unit{\kilo \hertz}, and the gain inside the gain bandwidth, which is \num{160,82} \unit{\decibel}. The DC plot (Figure \ref{fig:tinadc2probes}) shows the voltage with respect to the current and demonstrates the linearity of the system in the range of \numrange{0}{46,31} \unit{\nano \ampere}. After $V_{out}$ reaches \num{5} \unit{V}, the output remains fixed, because it cannot exceed the voltage provided to the amplifier. 

\begin{figure}[ht]
	\centering
	\begin{subfigure}[1a]{.49\linewidth}
		\includegraphics[width=\linewidth]{img/tina_AC_2probes}
		\caption{Frequency response}
		\label{fig:tinaac2probes}
	\end{subfigure}
	\hfill
	\begin{subfigure}[1b]{.49\linewidth}
		\includegraphics[width=\linewidth]{img/tina_DC_2probes}
		\caption{DC gain}
		\label{fig:tinadc2probes}
	\end{subfigure}
	\caption{Simulation of the first iteration of the photodetector}
	\label{fig:tina2probes}
\end{figure}

In addition to simulating the signal parameters of the amplifier, the noise was also checked. Figure \ref{fig:td:tinanoisev1}, shows that at 1 \unit{\hertz} the noise is 4,63 \unit[power-half-as-sqrt]{\micro\volt\per\hertz\tothe{0.5}} and remains about the same in the whole bandwidth of the amplifier. At around 1 \unit{\mega\hertz}, the minimum noise of \num{240,92} \unit[power-half-as-sqrt]{\nano\volt\per\hertz\tothe{0.5}} is achieved. 


\begin{figure}[ht]
	\centering
	\includegraphics[width=0.6\linewidth]{img/tina_noise_v1}
	\caption{Spectral noise density of the \gls{tia}}
	\label{fig:td:tinanoisev1}
\end{figure}



\subsection{Second iteration}\label{chap:td:v2}

A MATLAB script was written to calculate the component values based on the calculations in Chapter \ref{chap:td:calc:v2}. 

MATLAB calculates the component values with high precision and sourcing components with the exact values is not feasible, which is why components with standard values will be used in the setup. Simulations with standardized components were also done to compare the mathematically-ideal setup to the one in practice. 

\begin{table}[!ht]
	\centering
	\begin{tabular}{|c||c|c|c|}
		\hline
		\glsfmtshort{tia} version & v2 (Ideal) & v2 (Real.)	    & v1 		\\
		\hline
		$C_1$ (\unit{\pico \farad}) & \num{0,96904} & \num{1}	    & \num{1,6} 		\\
		\hline
		$f_c$ (\unit{\kilo \hertz}) & \num{22,0768} & \num{21,167}  &  \num{11,2422}	\\
		\hline
		
		$Q$  					    & \num{0,57735} & \num{0,55996} &  \num{0,35487}   \\
		\hline
	\end{tabular}
	\caption{\gls{tia} parameters with a \num{10} \unit{\mega \ohm} feedback resistor $R_1$}
	\label{tab:td:params}
\end{table}

Table \ref{tab:td:params} contains the MATLAB calculation results and shows the values of $f_c$ and $Q$ when only $C_1$ changes. $R_1$ is kept the same, as it is equivalent the \gls{dc} transimpedance $Z_{tDC}$\footnote{For realistic op-amps, $\frac{A_o}{A_o + 1} \to 1$, which means $Z_{tDC} \equiv R_1$ can be assumed to simplify explanations}. Importantly, the results show that the previous iteration had an overdamped response. Although the quality factor with realistic component values is also lower than the ideal $Q = \frac{\sqrt{3}}{3}$, the system sill behaves approximately like a Bessel filter.

\begin{figure}[ht]
	\centering
	\includegraphics[width=0.6\linewidth]{matlab/bode_v1_v2_comparison}
	\caption{Bode plot of the system with different capacitors}
	\label{fig:td:bodev1v2comparison}
\end{figure}

Using these values, the system model was calculated and plotted in Figure \ref{fig:td:bodev1v2comparison}. Certain differences can be seen between the two iterations. Most notably, the cutoff frequency $f_c$ of the first iteration is somewhat smaller than that of the second iteration. Furthermore, it can be observed that both the ideal and realistic versions of the system behave similarly. 


\begin{figure}[ht]
	\centering
	\begin{subfigure}[1a]{.49\linewidth}
		\includegraphics[width=\linewidth]{img/tina_AC_2probes_1p}
		\caption{Frequency response}
		\label{fig:tinaac2probes1p}
	\end{subfigure}
	\hfill
	\begin{subfigure}[1b]{.49\linewidth}
		\includegraphics[width=\linewidth]{img/tina_DC_2probes_1p}
		\caption{DC gain}
		\label{fig:tinadc2probes1p}
	\end{subfigure}
	\caption{Simulation of the second iteration of the photodetector}
	\label{fig:tina2probes1p}
\end{figure}

Figure \ref{fig:tina2probes1p} confirms the calculations, with some slight differences that can be attributed to the \gls{spice} models. As predicted, the frequency response of the circuit shows an increased cutoff frequency of \num{19,69} \unit{\kilo\hertz}, \num{8,92} \unit{\kilo\hertz} more than the previous iteration. Up to the cutoff frequency, the transimpedance gain is \num{160,83} \unit{\decibel}, which is almost the same as the former gain of \num{160,82} \unit{\decibel}. Furthermore, Figure \ref{fig:tinadc2probes1p} shows that, similarly to the first iteration, the system displays linearity in the range of \numrange{0}{45,86} \unit{\nano \ampere}. This is a reduction of \num{450} \unit{\pico\ampere} compared to the previous version and also \num{4,14} \unit{\nano \ampere} less than the ideal upper limit of the linear region.

\begin{figure}[ht]
	\centering
	\begin{subfigure}[1a]{.49\linewidth}
		\includegraphics[width=\linewidth]{matlab/noise_v2}
		\caption{Calculated voltage noise gain}
		\label{fig:td:v2_noise_calc}
	\end{subfigure}
	\hfill
	\begin{subfigure}[1b]{.49\linewidth}
		\includegraphics[width=\linewidth]{img/tina_noise_v2}
		\caption{Simulated spectral noise density}
		\label{fig:td:v2_noise_sim}
	\end{subfigure}
	\caption{Noise of the \gls{tia} using an AD795, 10 \unit{\mega\ohm} resistor and an ideal compensating capacitor}
	\label{fig:td:v2_noise_both}
\end{figure}


Additionally, a MATLAB script was used to calculate the \gls{tia} noise transfer function, the output of which is shown in Figure \ref{fig:td:v2_noise_calc}. As discussed in Chapter \ref{chap:td:calc:v2}, the noise gain spikes at the cutoff of the signal transfer function. The noise density was also plotted and is shown in Figure \ref{fig:td:v2_noise_sim}. At 1 \unit{\hertz}, the noise is 4,63 \unit[power-half-as-sqrt]{\micro\volt\per\hertz\tothe{0.5}} and it only settles at around 1 \unit{\mega\hertz}. The behavior is similar to the first iteration, but the increased bandwidth also results in an increase in noise in the extended bandwidth region.


\subsection{Third iteration}
Before deciding on a power supply system, the power requirements needed to be compared to the specifications listed in Table \ref{tab:power_supply}. A MATLAB script, which can be found in Appendix \ref{chap:appendix:code}, was written and used to calculate that, given a \textpm 5 \unit{\volt} supply, the system requires \num{2,3021} \unit{\milli\ampere} of current.  All solutions can provide that amount of current with reasonable overhead, which is why the choice was based on size, complexity and stability.

\subsection{Fourth iteration}\label{chap:td:v4_sim}
\begin{figure}[ht]
	\centering
	\includegraphics[width=0.6\linewidth]{matlab/bode_bootstrap_vs_nostrap}
	\caption{Comparison of a bootstrapped \gls{tia} and a \gls{tia} with only a compensating capacitor, both with the same transimpedance gain}
	\label{fig:td:bodebootstrapvsnostrap}
\end{figure}

As previously discussed, the value of the capacitor could not be determined analytically, however an approximate value was calculated using MATLAB simulations. Figure \ref{fig:td:bodebootstrapvsnostrap} shows that the bootstrapped system demonstrates a significantly increased bandwidth. According to Table \ref{tab:td:v4_params}, the bootstrapped amplifier still has a quality factor $Q$ close to the ideal \num{0,57735}, which is even better than the previous iterations of the system. Furthermore, the capacitor value is standard, meaning that the value can be used in the physical system without further approximations.

\begin{table}[!ht]
	\centering
	\begin{tabular}{|c||c|c|c|}
		\hline
		\glsfmtshort{tia} version 	& v4 (Bootstrapped) & v4 (No bootstrap) & v2/v3 			\\
		\hline
		$C_1$ (\unit{\pico \farad}) & \num{0,3} 		& \num{0,49265}    	& \num{1}			\\
		\hline
		$R_1$ (\unit{\mega\ohm}) 	& \num{1} 			& \num{1}	    	& \num{10}			\\
		\hline
		$f_c$ (\unit{\kilo \hertz}) & \num{724,69}	 	& \num{437,32} 		&  \num{21,167}		\\
		\hline
		$Q$  					    & \num{0,57814} 	& \num{0,57735} 	&  \num{0,55996}   \\
		\hline
	\end{tabular}
	\caption{Comparison of \gls{tia} parameters of the fourth and second/third iteration of the system}
	\label{tab:td:v4_params}
\end{table}



\begin{figure}[!ht]
	\centering
	\begin{subfigure}[1a]{.9\linewidth}
		\includegraphics[width=\linewidth]{img/LTSpice_sim_v4_1stage}
		\caption{Bootstrapped \gls{tia} with a single output stage (40 \unit{\decibel} gain)}
		\label{fig:td:v4_nia:one}
	\end{subfigure}
	\hfill
	\begin{subfigure}[2a]{.9\linewidth}
		\includegraphics[width=\linewidth]{img/LTSpice_sim_v4_2stage}
		\caption{Bootstrapped \gls{tia} with two cascaded output stages (20 \unit{\decibel} gain each)}
		\label{fig:td:v4_nia:cascaded}
	\end{subfigure}
	\hfill
	\begin{subfigure}[3a]{.9\linewidth}
		\includegraphics[width=\linewidth]{img/LTSpice_sim_v4_no_post}
		\caption{Bootstrapped \gls{tia} without post-amplification}
		\label{fig:td:v4_nia:none}
	\end{subfigure}
	\caption{Simulation of the effect of cascading on the bandwidth of the fourth iteration of the photodetector}
	\label{fig:td:v4_nia}
\end{figure}

Additionally, the post-amplification part of the system was simulated with and without the cascaded stages. Shown in Figure \ref{fig:td:v4_nia}, the results demonstrate a significant increase in bandwidth. The plot of the bootstrapped \gls{tia} discussed in the design section (Chapter \ref{chap:td:v4_design}) with a single 40 \unit{\decibel} non-inverting stage shows a flat response with a cutoff frequency of approximately \num{194,45} \unit{\kilo\hertz}. This is significantly lower than \num{716,67} \unit{\kilo\hertz}, which is the bandwidth of the \gls{tia} without any post-amplification (Figure \ref{fig:td:v4_nia:none}). Figure \ref{fig:td:v4_nia:cascaded} shows the increased bandwidth of \num{638,08} \unit{\mega\hertz} when two stages are used instead of one. In spite of the significantly bigger cutoff frequency, the response is not maximally flat, which means the linearity at high frequencies is subpar.

Here it is important to mention that while cascading increases the bandwidth of the system, it also results in more noise. Figure \ref{fig:td:v4_noise} shows the effect of cascading. Aside from the obvious increase around the cutoff frequency, the noise at 1 \unit{\hertz} also increases from 14,5 \unit[power-half-as-sqrt]{\nano\volt\per\hertz\tothe{0.5}} to 18 \unit[power-half-as-sqrt]{\nano\volt\per\hertz\tothe{0.5}}.

\begin{figure}[!ht]
	\centering
	\begin{subfigure}[1a]{.8\linewidth}
		\includegraphics[width=\linewidth]{img/LTSpice_sim_v4_1stage_noise}
		\caption{Noise simulation of a bootstrapped \gls{tia} with a single output stage}
		\label{fig:td:noise_1stage}
	\end{subfigure}
	\hfill
	\begin{subfigure}[2a]{.8\linewidth}
		\includegraphics[width=\linewidth]{img/LTSpice_sim_v4_2stage_noise}
		\caption{Noise simulation of a bootstrapped \gls{tia} with two cascaded output stage}
		\label{fig:td:noise_2stage}
	\end{subfigure}
	\caption{Simulation of the effect of cascading on the noise of the fourth iteration of the photodetector}
	\label{fig:td:v4_noise}
\end{figure}

\subsection{Final \glsfmtshort{cwodmr} iteration}
After implementing all the changes, the simulation circuit in Figure \ref{fig:td:photodetecv5sim} was used to verify the noise improvements. Additionally, the same circuit was tested with the ADA4627, because in the initial design stage it was seen as the second most promising op-amp.

\begin{figure}[ht]
	\centering
	\includegraphics[width=0.7\linewidth]{img/photodetec_v5_sim}
	\caption{Circuit used for the simulation of the \gls{cwodmr} photodetector}
	\label{fig:td:photodetecv5sim}
\end{figure}

Figure \ref{fig:td:v5_noise} shows a comparison of the noise of the OPA827-based \gls{tia} with the alternative ADA4627-based \gls{tia}. The results show the OPA827 has marginally better noise performance at higher frequency. However, it also performs marginally worse at lower frequencies. In addition, the two possible implementations of the \gls{cwodmr} photodetector were compared to the simulations of the second iteration. Table \ref{tab:v2_v5_v5_alt} shows both new versions demonstrate major noise improvements over the old one in the frequency range of 1 \unit{\hertz} to 100 \unit{\kilo\hertz}.

\begin{figure}[ht]
	\centering
	\begin{subfigure}[1a]{.49\linewidth}
		\includegraphics[width=\linewidth]{img/tina_noise_v5}
		\caption{Main (OPA827) version}
		\label{fig:td:v5_main_noise}
	\end{subfigure}
	\hfill
	\begin{subfigure}[1b]{.49\linewidth}
		\includegraphics[width=\linewidth]{img/LTSpice_sim_v5_noise_alt}
		\caption{Alternative (ADA4627) version}
		\label{fig:td:v5_alt_noise}
	\end{subfigure}
	\hfill
	\begin{subfigure}[1b]{.49\linewidth}
		\includegraphics[width=\linewidth]{img/tina_noise_v2}
		\caption{Old (second iteration; AD795) version}
		\label{fig:td:v2_noise}
	\end{subfigure}
	\caption{Simulation of the noise of the photodetector with different op-amps}
	\label{fig:td:v5_noise}
\end{figure}

\begin{table}[ht]
	\centering
	\begin{tabular}{|c||c|c|c|}
		\hline
		& v2 	& v5 & Alternative v5  \\
		\hline
		\glsfmtshort{ic}					& AD795		 			& OPA827 	     & ADA4627 	\\
		\hline
		Bandwidth (\unit{\kilo\hertz}) 		& 21,167		 			& 3,2 	   		 & 3,2  			\\
		\hline
		Noise at 1 \unit{\hertz} (\unit[power-half-as-sqrt]{\micro\volt\per\hertz\tothe{0.5}})	& 4,63 				& 1,31 	 & 1,30 			\\
		\hline
		Noise at 100 \unit{\kilo\hertz} (\unit[power-half-as-sqrt]{\nano\volt\per\hertz\tothe{0.5}})	& 1830 				& \num{230,41} 	 & \num{347,87} 			\\
		\hline
	\end{tabular}
	\caption{Comparison of the final \gls{cwodmr} photodetector with its alternatives}
	\label{tab:v2_v5_v5_alt}
\end{table}







\section{Photodetection implementation}
Creating the physical \gls{pcb} is a more straightforward process than the design and simulation, but it still needs to be discussed. Aside from the boards, this section also covers the essential components.

\subsection{First iteration}

\begin{figure}[ht]
	\centering
	\includegraphics[width=0.6\linewidth]{img/photodetec_og_pcb}
	\caption{First iteration of the photodetection \gls{pcb}}
	\label{fig:photodetecogpcb}
\end{figure}


Figure \ref{fig:photodetecogpcb} shows the back side of the \gls{pcb}. It hosts all components, except for the photodiode, which sits unobstructed on the front side. Requirements for the physical dimensions were also set by the client. The \gls{pcb} needs to fit the Thorlabs mount standard, since the rest of the setup also uses it. For photodetection, the system uses a BPW34 through-hole diode. Additionally, the \gls{tia} uses an AD795 op-amp and the post-amplifier uses an AD820.

\subsection{Second iteration}

\begin{figure}[ht]
	\centering
	\begin{subfigure}[1a]{.49\linewidth}
		\includegraphics[width=\linewidth]{img/photodetec_v2_gp_pcb2}
		\caption{\gls{ecad} model}
		\label{fig:td:imp:photodetecv2gppcb2}
	\end{subfigure}
	\hfill
	\begin{subfigure}[1b]{.49\linewidth}
		\includegraphics[width=\linewidth]{img/photodetec_v2_gp_sch}
		\caption{Schematic}
		\label{fig:td:imp:photodetecv2gppcb2sch}
	\end{subfigure}
	\caption{Second iteration of the photodetection \gls{pcb}}
	\label{fig:td:imp:photodetecv2}
\end{figure}




Due to the fact that the topology of the first iteration was kept, the \gls{pcb} does not need any modifications in order to work with the circuit parameters. However, after a discussion with the client, a ground plane was added. Usually such an addition would be needed in high-speed and/or high-power use cases, but the circuit only deals with low-speed, very-low-power signals. The reason for the ground plane in the updated design is broader quantum protocol support. While \gls{cwodmr} operates at low frequencies, future expansion of the sensing setup might require support for different high-speed pulsing sequences. In that case, the component values can be modified again, without the need for making a new board. Figure \ref{fig:td:imp:photodetecv2gppcb2} shows the new board design. The updated board also has better routing than the previous version, which was done to optimize the noise performance.


\subsection{Third iteration}

\begin{figure}[ht]
	\centering
	\begin{subfigure}[1a]{.49\linewidth}
		\includegraphics[width=\linewidth]{img/photodetec_v3}
		\caption{\gls{ecad} model}
		\label{fig:td:imp:photodetecv3pcb}
	\end{subfigure}
	\hfill
	\begin{subfigure}[1b]{.49\linewidth}
		\includegraphics[width=\linewidth]{img/photodetec_v3sch}
		\caption{Schematic}
		\label{fig:td:imp:photodetecv3sch}
	\end{subfigure}
	\caption{Second iteration of the photodetection \gls{pcb}}
	\label{fig:td:imp:photodetecv3}
\end{figure}




The third version has an optimized power system that employs a TPS60400 charge pump to invert the 5 \unit{\volt} power supplied to it. The charge pump, combined with the fixed board dimensions, made it more difficult to do the placing and routing. Additionally, the clearance of the power and ground nets was increased. Figure \ref{fig:td:imp:photodetecv3} shows the \gls{ecad} view of the third iteration of the photodetector. 



\subsection{Fourth iteration}

\begin{figure}[ht]
	\centering
	\begin{subfigure}[1a]{.49\linewidth}
		\includegraphics[width=\linewidth]{img/photodetec_v4}
		\caption{\gls{ecad} model}
		\label{fig:td:imp:photodetecv4pcb}
	\end{subfigure}
	\hfill
	\begin{subfigure}[1b]{.49\linewidth}
		\includegraphics[width=\linewidth]{img/photodetec_v4sch}
		\caption{Schematic}
		\label{fig:td:imp:photodetecv4sch}
	\end{subfigure}
	\caption{Second iteration of the photodetection \gls{pcb}}
	\label{fig:td:imp:photodetecv4}
\end{figure}



\begin{figure}[!ht]
	\centering
	\begin{subfigure}[1a]{.49\linewidth}
		\includegraphics[width=\linewidth]{img/ad795_noise}
		\caption{Noise spectral density of the AD795 (image credit to the AD795 documentation \cite{ad2019ad795})}
		\label{fig:td:ad795noisesecond}
	\end{subfigure}
	\hfill
	\begin{subfigure}[1b]{.49\linewidth}
		\includegraphics[width=\linewidth]{img/ad4637_noise}
		\caption{Noise spectral density of the ADA4637 (image credit to the ADA4637 documentation \cite{ad2019ada4637})}
		\label{fig:td:ada4637noise}
	\end{subfigure}
	\caption{Noise spectral density of the amplifiers used in the \gls{tia} circuit of the photodetector}
	\label{fig:td:opampnoisev4}
\end{figure}

This iteration of the photodetector is less concerned with minor improvements and adaptations for the sensing setup than the previous iterations. In fact, the differences are so significant that the fourth photodetector version requires mostly new components. Instead of an AD795-based \gls{tia} and an AD820-based post-amplifier, the fourth iteration uses ADA4637 and ADA4627 op-amps. The ADA4637 offers significantly higher \gls{gbp} (79 \unit{\mega\hertz} in comparison to 1,6 \unit{\mega\hertz}), however there is an increase in the noise levels (see Figure \ref{fig:td:opampnoisev4}) and this is why it was used for the \gls{tia}. The bootstrap buffer was also matched to use the same op-amp. An amplifier of the same family, namely the ADA4627, is employed by the output stages, because of its much better noise performance\footnote{The ADA4627 has 16 \unit[power-half-as-sqrt]{\nano\volt\per\hertz\tothe{0.5}} at 1 \unit{\hertz}, which steadily decreases to 5 \unit[power-half-as-sqrt]{\nano\volt\per\hertz\tothe{0.5}} at 10 \unit{\kilo\hertz}. The AD820 has 100 \unit[power-half-as-sqrt]{\nano\volt\per\hertz\tothe{0.5}} at 1 \unit{\hertz}, which, similar to the ADA4627, decreases to 15 \unit[power-half-as-sqrt]{\nano\volt\per\hertz\tothe{0.5}} at 10 \unit{\kilo\hertz}. The ADA4637 was also considered as an option for the non-inverting stages, but its noise is higher than the ADA4627}. Decreasing noise later in the signal chain is crucial to maintaining a clear output. With all these considerations in mind, the circuit was designed in a similar manner to the previous iterations (see Figure \ref{fig:td:imp:photodetecv4}).



\subsection{Final \glsfmtshort{cwodmr} iteration}
\begin{figure}[ht]
	\centering
	\begin{subfigure}[1a]{.49\linewidth}
		\includegraphics[width=\linewidth]{img/photodetec_v5}
		\caption{\gls{ecad} model}
		\label{fig:td:imp:photodetecv5pcb}
	\end{subfigure}
	\hfill
	\begin{subfigure}[1b]{.49\linewidth}
		\includegraphics[width=\linewidth]{img/photodetec_v5sch}
		\caption{Schematic}
		\label{fig:td:imp:photodetecv5sch}
	\end{subfigure}
	\caption{Final \gls{cwodmr} iteration of the photodetection \gls{pcb}}
	\label{fig:td:imp:photodetecv5}
\end{figure}

Figure \ref{fig:td:imp:photodetecv5} shows the schematic and the \gls{ecad} model of the final \gls{cwodmr} detector. An additional version that did not have an inverting charge pump was developed as well. For the \gls{cwodmr} demonstrator, having an external split-rail power supply will greatly increase its size and decrease its portability, which is why having a charge pump, with the trade-off of increased noise, is preferred. This way, a microcontroller can power the photodetector. However, when using the lab setup, as little noise as possible is desired. What is more, supplying the circuit with \textpm 5 \unit{\volt} power is not a problem. Consequently, a version without a charge pump is preferred in this scenario.

Aside from making the circuit itself less noisy, extra care was taken to reduce external noise susceptibility. Although having connection pins for the power supply is convenient for testing, it also increases noise, which is why a barrel connector was chosen instead. Moreover, additional vias were added to the ground plane as a means of reducing electromagnetic interference.

%\section{OLIA implementation}
\section{Software application}

\begin{figure}[ht]
	\centering
	\begin{tikzpicture}[
		% define box styles
		boxr/.style={rectangle, thick, draw=red!50!black, fill=red!5, minimum size=5mm},
		boxg/.style={rectangle, thick, draw=green!50!black, fill=green!5, minimum size=5mm},
		boxb/.style={rectangle, thick, draw=blue!50!black, fill=blue!5, minimum size=5mm},
		]
		% draw MW nodes
		\node[boxg] (ad2)								{Analog Discovery 2};
		\node[boxg] (app)		[above=of ad2] 			{Application};
		% draw main system nodes
		\node[boxb] (driver) 	[below=of ad2]			{Current driver};
		\node[boxb] (laser) 	[right=40 pt of driver]	{Laser};
		\node[boxb] (nv) 		[right=of laser] 		{NV sample};
		\node[boxb] (detect) 	[right=of nv]			{Photodetector};
		\node[boxg] (lockin)	[above=of detect]		{Lock-in amplifier};
		
		
		% draw app lines
		\draw[->, thick]			(app.south) 		to node[midway, left]{\glsfmtshort{ttl}}	(ad2.north);
		\draw[->, thick]			(ad2.north) 		to node[midway, right]{Oscilloscope}		(app.south);
		% draw main system lines
		\draw[->, thick]			(ad2.south)			to node[midway, left]{\glsfmtshort{ttl}} 	(driver.north);
		\draw[->, thick] 			(driver.east) 		to node[midway, above]{Current} 			(laser.west);
		\draw[->, thick]			(laser.east) 		to node[midway, above]{Light} 				(nv.west);
		\draw[->, thick]			(nv.east) 			to node[midway, above]{Light}				(detect.west);		
		\draw[->, thick]			(detect.north) 		to node[midway, right]{Signal}				(lockin.south);	
		% draw long lines
		\draw[->, thick] (ad2.east) to node[midway, below]{Reference signal} (lockin.west);
		\draw[->, thick] (lockin.west) to node[midway, above]{Processed photodetector data}(ad2.east);
	\end{tikzpicture}
	\caption{High-level overview of the $T_1$ measurements quantum sensing setup}
	\label{fig:td:ad2}
\end{figure}

A quick look at the goals laid out in Chapter \ref{chap:goals} makes it clear that they separate the pulsing sequence software from the \gls{daq} software. Despite this, during talks with the client it was decided that a \gls{gui} application that has both functions is desireable. The main reason for this is that an all-encompassing application will make it easier for physics students with little coding experience to interact with the setup. This is also the reason why a the application was implemented as a \gls{gui} instead of a \gls{cli}, which is easier to program, but can be harder to work with. Another reason for integrating pulse generation and data acquisition is that, instead of reading data directly from the lock-in amplifier, its output can be connected to the oscilloscope of the Analog Discovery. Because the application only needs to interact with one device, a single \gls{api} needs to be used and consequently coding separate applications will only result in a lot of copied code and a less user-friendly experience. 

While a single application for both functionalities is possible, it has also become clear at this point in the report that different quantum protocols require different physical setups. This makes it is difficult to foresee what hardware changes are implemented and as a result only one protocol can be implemented at a time. With this information in mind, the client requested a Python application based on the Tk \gls{gui} kit that combines $T_1$ measurements pulse generation and data acquisition using the Analog Discovery 2. A high-level diagram of the setup the application is written for can be seen in Figure \ref{fig:td:ad2}.

\subsection{\glsfmtshort{gui}}
Figure \ref{fig:td:appgui} shows what the front-end looks like when running the application. There are four panels that make up the program window: the device panel, the status log, the parameter panel and lastly the result plot.

\begin{figure}[ht]
	\centering
	\includegraphics[width=1\linewidth]{img/app_gui}
	\caption{The \gls{gui} of the $T_1$ measurements application}
	\label{fig:td:appgui}
\end{figure}

The device panel is there to monitor the connection with the Analog Discovery 2. On startup, a scan is conducted to find a device, but devices can also be plugged in later and, after pressing scan, the test can be ran. If no device is connected, the tests cannot be ran.

As Figure \ref{fig:td:appgui} shows, the status log contains messages pertaining to different subsystems. In addition to system statuses, it also tracks tests by reporting which data point is being executed at a given time. If there are any issues with the Analog Discovery while running $T_1$ measurements, they will be shown in the log as well.

Perhaps most important is the $T_1$ parameter panel. It allows the user to set custom values for tests. The laser pulse width $\tau_{las}$ is set manually, but it also implicitly sets the sequence period $T_{seq}$ and the minimum dark time $\tau_{dmin}$ according to Equation \eqref{eq:td:tlas}. Setting the maximum dark time $\tau_{dmax}$ can also be done from this panel, however it should be noted that it cannot exceed $2000\tau_{dmin}$. The number of data points, or alternatively the number of dark times for which the test needs to be ran, can also be set. Based on the provided number, an array of dark times will be generated and then used when creating the sequences.

\begin{equation}\label{eq:td:tlas}
	\tau_{las} = \frac{T_{seq}}{819,2} = 5\tau_{dmin}
\end{equation}

The last panel contains the final $T_1$ plot. After running all data points and measuring the results, a plot of the dark times $\tau_d$ and their respective signal magnitude will be plotted. As of finalizing the report, the physical setup is incomplete, so the plot is only an example of how a \lstinline|numpy| plot will be rendered by the application.

Additionally, it is important to keep in mind the whole \gls{gui} is an instance of a class. This allows for sharing object attributes across all methods of the class, which is particularly useful when implementing \lstinline|Tkinter| widget callbacks.

\subsection{Device connection}
There are two events that call the function that connects to an Analog Discovery: startup and pressing the "Scan" button (see Figure \ref{fig:td:appgui}). It is important to verify the Analog Discovery connection before running $T_1$ measurements by scanning for devices, because there is no disconnection handler and even if the device was connected on startup, there is no guarantee it will be connected before running the test.

A simplified version of the connection method that does not handle connection errors can be seen in Figure \ref{fig:td:find_device}. What the snippet of code shows is the required \gls{api} calls. After retrieving the library version, the device is configured so that, even if the program is exited, the inputs and outputs will finish executing and measuring the current $T_1$ sequence. Afterwards, the first device that is found is opened and its handle is passed to the instance of the application class. Finally, the last line of code disables automatic reconfiguration. It is important that it is turned off, otherwise the device will be reconfigured on every call to a \lstinline|FDwf*Set| function, which will significantly worsen the performance of the Analog Discovery.

\begin{figure}[ht]
	\begin{lstlisting}[language=pythe]
def find_device(self):
	# version required for operation
	version = create_string_buffer(16)
	dwf.FDwfGetVersion(version)
		
	# define behavior on close (0 = run, 1 = idle, 2 = off)
	dwf.FDwfParamSet(DwfParamOnClose, c_int(0))
		
	# open device and get interface reference
	dwf.FDwfDeviceOpen(c_int(-1), byref(self.hdwf))
		
	# disable auto config for better performance
	dwf.FDwfDeviceAutoConfigureSet(self.hdwf, c_int(0))
	\end{lstlisting}
	\caption{Simplified device discovery and connection function}
	\label{fig:td:find_device}
\end{figure}


\subsection{Message logging}
The message log is an important medium for communicating with the user. As previously mentioned, the log contains status messages from the different subsystems of the application, in addition to the $T_1$ measurement information.  Logging is done to a \lstinline|ttk.Treeview|\footnote{The code in Figure \ref{fig:td:treeview} uses a \lstinline|ttk| widget, which is not one of the standard \lstinline|tk| widgets. \lstinline|ttk|, aside from providing extra widgets, offers more modern versions of many \lstinline|tk| widgets, which is why it is used whenever possible.} widget, which is configured to function like a table. Figure \ref{fig:td:treeview} shows how the tree is created and assigned to the \lstinline|tree_running_log| attribute. After the \gls{gui} is created, a message is logged using the \lstinline|log_message| method to notify the user.

\begin{figure}[ht]
	\begin{lstlisting}[language=pythe]
# create tree
self.tree_running_log = ttk.Treeview(lf_status, columns=("Source", "Status", "Value"))

# log first status message
self.log_message("GUI", "Startup", "GUI created")
	\end{lstlisting}
	\caption{Simplified message log widget creation}
	\label{fig:td:treeview}
\end{figure}

Figure \ref{fig:td:log_message_function} contains the code for logging a message. A call to the function should contain the source, status (also known as short message) and value (also known as long message). There are several log sources. \lstinline|GUI| indicates a message is about \lstinline|tkinter| or more broadly about the \gls{gui} application. If the source is \lstinline|AD2|, then the message is about the device status. Furthermore, if the Analog Discovery sends an error (e.g. data loss) during execution, it will be logged as \lstinline|AD2|. Messages about the status of the $T_1$ measurements are logged with the sources set to \lstinline|T1|.

\begin{figure}[ht]
	\begin{lstlisting}[language=pythe]
def log_message(self, source, status, value):
	timestamp = datetime.now().strftime("%H:%M:%S")
	self.tree_running_log.insert('', tk.END, text=timestamp, values=(source, status, value))
	\end{lstlisting}
	\caption{Simplified message logging function}
	\label{fig:td:log_message_function}
\end{figure}


\subsection{Pulse generation}
The fundamentals of pulse generation using the Analog Discovery, which were laid out in Chapter \ref{chap:fd:laser_driver}, still hold true, even when using the WaveForms \gls{sdk} for Python.

That being said, there is one major difference. The buffer size, which was limited to 100 when set through the WaveForms app, is actually 4096. This allows for pulse durations that fit the formula $\tau_{las} = t_{las}\frac{T_{seq}}{4096}$, for $t_{las} \in \mathbb{Z}^+$ and $t_{las} \leq 4096$.

As a consequence of the similarities, the MATLAB code for pulse generation from Chapter \ref{chap:fd:laser_driver} (see Figure \ref{code:t1_pulse}) was rewritten in Python, with the calls to the WaveForms \gls{api} replacing the \gls{csv} file writer. 

Before doing anything else, the pulse generation function calls the function for determining the dark time for every data point. Ideally, this function would return many small values and some bigger ones. Such data point spacing can be achieved using \lstinline|numpy.logspace|, which provides logarithmically-spaced values, thus satisfying the aforementioned conditions. Despite its theoretical applicability, in actuality it cannot be used because it is designed with floating-point numbers in mind. In this case, the buffer slot count $t_d$ is a rounding of the actual dark time $\tau_d$, which makes them integers. Additionally, rounding functions cannot be used in combination with \lstinline|numpy.logspace|, because for a lot of data points, the function can return multiple of the same $t_d$, which is not desirable. To get around these problems, a function based on \lstinline|numpy.logspace| was implemented. 

\begin{equation}\label{eq:logpspace_custom}
	t_{d} = t_{dmin} + round(10^{x\frac{\log(t_{dmax})}{n - 1}})\ for\ x \in [1,n]\cap \mathbb{Z}
\end{equation}

\begin{figure}[ht]
	\begin{lstlisting}[language=pythe]
		def gen_log_space(self, td_min, td_max, n):
		result = [td_min]
		# return 1 if bad n
		if n <= 1:
		return result
		
		for i in range(1, n):
		td_curr = td_min + round(10 ** (i * log10(td_max) / (n - 1)))
		# exits prematurely if min and max are too close, otherwise replaces last value with td_max
		if td_curr >= td_max:
		result.append(td_max)
		return result
		
		# check if dark time is repeating last value and increment if yes
		if td_curr - result[-1] >= 1:
		result.append(td_curr)
		else:
		result.append(result[-1] + 1)
		return result
	\end{lstlisting}
	\caption{Custom \lstinline|logspace| function implementation}
	\label{fig:td:logspace_gen_func}
\end{figure}


Equation \eqref{eq:logpspace_custom} shows that, provided with the minimum and maximum dark time $t_{dmin}$ and $t_{dmax}$, as well as the number of data points $n$, the function returns logarithmically-spaced values. To address the problem with repetitions, $t_d$ is incremented whenever it is equal to the $t_d$ of the previous data point. This increment, however, results in the final value being bigger than the $t_{dmax}$. In the worst case, this might result in writing data outside the buffer range, which is why the last value is manually set to $t_{dmax}$. 


Figure \ref{fig:td:logspace_gen_func} shows the custom \lstinline|logspace| function. To compare its performance to the original function, the plots in Figure \ref{fig:td:logspace} were made. It is clear that both functions demonstrate similar behaviors for big values, but the custom function also presents linear behavior for low $t_d$ values, as can be seen in Figure \ref{fig:td:logspace_small}.


\begin{figure}[!ht]
	\centering
	\begin{subfigure}[1a]{.49\linewidth}
		\centering
		\includegraphics[width=\linewidth]{matlab/logspace_large_td}
		\caption{Plot of all data points}
		\label{fig:td:logspace_whole}
	\end{subfigure}
	\hfill
	\begin{subfigure}[1b]{.49\linewidth}
		\centering
		\includegraphics[width=\linewidth]{matlab/logspace_small_td}
		\caption{Plot of the first ten data points}
		\label{fig:td:logspace_small}
	\end{subfigure}
	\caption{Generating buffer dark times $t_{d}$ using \lstinline|numpy.logspace| and a custom function from 1 to 1000 with 25 data points}
	\label{fig:td:logspace}
\end{figure}





\begin{figure}[ht]
	\begin{lstlisting}[language=pythe]		
pattern = self.sequences[0]

# enable
dwf.FDwfAnalogOutNodeEnableSet(self.hdwf, self.laser_channel, AnalogOutNodeCarrier, c_int(1))

# set function to custom
dwf.FDwfAnalogOutNodeFunctionSet(self.hdwf, self.laser_channel, AnalogOutNodeCarrier, funcCustom)

# set pattern of custom function
dwf.FDwfAnalogOutNodeDataSet(self.hdwf, self.laser_channel, AnalogOutNodeCarrier, pattern,
c_int(self.pattern_size))

# set frequency of the whole pattern
dwf.FDwfAnalogOutNodeFrequencySet(self.hdwf, self.laser_channel, AnalogOutNodeCarrier, c_double(f_pattern))

# set amplitude of the whole pattern to 5V
dwf.FDwfAnalogOutNodeAmplitudeSet(self.hdwf, self.laser_channel, AnalogOutNodeCarrier, amplitude)

# set the run and wait time (in seconds)
dwf.FDwfAnalogOutRunSet(self.hdwf, self.laser_channel, c_double(run_time / f_pattern))
dwf.FDwfAnalogOutWaitSet(self.hdwf, self.laser_channel, c_double(wait_time / f_pattern))

# start outputting
dwf.FDwfAnalogOutConfigure(self.hdwf, self.laser_channel, c_int(1))
	\end{lstlisting}
	\caption{Outputting the first sequence from the \lstinline|sequences| attribute of the \gls{gui} object using the WaveForms \gls{api}}
	\label{fig:td:pulsing_sequence}
\end{figure}

\subsection{Data acquisition}

\subsection{Data processing and visualization}