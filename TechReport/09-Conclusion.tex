\chapter{Testing and results}\label{chap:conclusion}
%This chapter discusses the results of the tests and what conclusions could be gathered from them. All tests were based on real scenarios, but the performance tests push the platform to its limits by replicating rare and improbable edge cases.

Table \ref{tab:test:tasks} shows all the tasks discussed in Chapter \ref{chap:goals} and their current status. While working on the project, the client decided that purchasing a lock-in amplifier is more worthwhile than implementing \gls{olia}, which is why tasks associated with its implementation are marked as canceled. The items with a reassigned status were given over to two students, who are also involved with the quantum sensing setup. The laser driver was reassigned, because a completely new design was needed, the making of which would require too much time. The pulse sequence script was finished and it was passed on to one of the other students for further refinement and possibly for the implementation of different protocols.

\begin{table}[!ht]
	\centering
	\begin{tabular}{|c||c|c|}
		\hline
		Number 	& Task 									& Status 							\\
		\hline
		1.1 	& Design photodetector 					& Completed							\\
		\hline
		1.2 	& Build \glsfmtshort{olia} 				& Canceled	 						\\
		\hline
		1.3 	& Set up laser driver					& Partially completed; reassigned	\\
		\hline
		2.1 	& Develop data acquisition software 	& Canceled							\\
		\hline
		2.2 	& Program pulse sequences 				& Mostly completed; reassigned	 	\\
		\hline
		3.1 	& Compare lock-in amplifiers			& Canceled  						\\
		\hline
		3.2 	& Test quantum sensing setup 			& In-progress  						\\
		\hline
	\end{tabular}
	\caption{List of tasks and their completion status}
	\label{tab:test:tasks}
\end{table}

%\begin{table}[ht]
%\centering
%\begin{adjustbox}{angle=0}
%\begin{tabular}{|l|l|l|}
%	\hline
%	Number &Task                                                                                                                                                                                     & Status  

%                            \\ \hline
%	1.1    & \begin{tabular}[c]{@{}l@{}}Deploy and configure at least \\ one OpenRemote instance\end{tabular}                                                                                         & \cellcolor[HTML]{34FF34}Completed   \\ \hline
%	1.2    & \begin{tabular}[c]{@{}l@{}}Establish communication to \\ the OpenRemote instance \\ using the HTTP and \\ MQTT protocols\end{tabular}                                                    & \cellcolor[HTML]{34FF34}Completed   \\ \hline
%	2.1    & \begin{tabular}[c]{@{}l@{}}Simulate IoT devices (smart homes)\\ that send and receive concurrent MQTT \\ data to OpenRemote and measure the \\ latency of the transmissions\end{tabular} & \cellcolor[HTML]{34FF34}Completed   \\ \hline
%	2.2    & \begin{tabular}[c]{@{}l@{}}Create a physical IoT device \\ setup using a platform like ESP32 \\ or Arduino and recreate the \\ tests from Goal 1.2\end{tabular}                          & \cellcolor[HTML]{34FF34}Completed   \\ \hline
%	2.3    & Integrate and test OpenRemote's Prophet project                                                                                                                                          & \cellcolor[HTML]{FE0000}Not started \\ \hline
%	2.4    & Test performance and create visualizations                                                                                                                                               & \cellcolor[HTML]{34FF34}Completed   \\ \hline
%	3.1    & Document technical progress                                                                                                                                                              & \cellcolor[HTML]{34FF34}Completed   \\ \hline
%	3.2    & Reflect on personal and professional development                                                                                                                                         & \cellcolor[HTML]{34FF34}Completed   \\ \hline
%	3.3    & Communicate project results                                                                                                                                                              & \cellcolor[HTML]{F8FF00}Ongoing     \\ \hline
%\end{tabular}
%\end{adjustbox}
%\caption{Goal completion}
%\label{tab:goal_com}
%\end{table}



%\section{Test goals and performance metrics}
\section{Test goals}

%\subsection{\glsfmtshort{tia}}
There are two suitable methods of characterizing the \gls{tia}. As the main point of interest is the linearity of the amplifier, the first characterization method is to simulate photodiode readings at different light intensities by modeling it as a current source. By varying the diode model current, either by sweeps or manually, the performance of the physical device can be compared to the simulations.

The second method supplements the results of the first one. It involves measuring the scattering parameters $S_{11}$ and $S_{21}$ and calculating the frequency response of the system. Equation \eqref{eq:test:s11s22} shows the formula used to approximate the transimpedance $Z_t$ for a load resistor $R_l$ \cite{sackinger2017analysis}. This method provides a better overview of the transimpedance over the whole bandwidth of the photodetector.

\begin{equation}\label{eq:test:s11s22}
	Z_t(f) \approx \frac{S_{21}(f)R_l}{1 - S_{11}(f)}
\end{equation}

Neither of these methods uses a real photodiode, which is why an integration test is also necessary. Using the laser driver, a square-wave signal is pulsed at different frequencies within the amplifier bandwidth. By comparing the signal output by the photodetector to the signal output by the function generator, the performance of the system can be determined.

%\begin{table}[ht]
%	\centering
%		\begin{tabular}{|c|c|c|c|}
%			\hline
%			Metric                      & API  & Unit(s) of measurement                                                 & Explanation                                                                               \\ \hline
%			Device provisioning latency & HTTP & seconds (s)                                                            & \begin{tabular}[c]{@{}c@{}}Time between creation \\ request and confirmation\end{tabular} \\ \hline
%			Message loss percentage     & MQTT & percent (\%)                                                           & \begin{tabular}[c]{@{}c@{}}Percent of MQTT \\ messages which are lost\end{tabular}        \\ \hline
%			Resource usage              & -    & \begin{tabular}[c]{@{}c@{}}percent \\ (CPU \%, RAM \%)\end{tabular} & \begin{tabular}[c]{@{}c@{}}CPU and RAM usage \\ when running tests\end{tabular}           \\ \hline
%		\end{tabular}
%	\caption{Metrics of the OpenRemote scalability tests}
%	\label{tab:metrics}
%\end{table}



\section{Test setup}


\subsection{Photodiode-model characterization}
Characterizing the system requires precision, due to the small currents at its input. As the diode detects ambient light and cannot be set to one current level reliably, a model needs to be used instead (see Figure \ref{fig:test:diode_equiv}). A voltage source can then be used to simulate the nanoampere photovoltaic current $I_f$ generated by the diode. This model assumes the parallel shunt resistance is so high that it can be omitted.

\begin{figure}[ht]
	\centering
	\resizebox{.5\textwidth}{!}{%
		\begin{circuitikz}
			\tikzstyle{every node}=[font=\normalsize]
			\draw[line width=0.5pt] (1,0) 
			to [european resistor, l={$R_i$}] (1,-2)
			(1, 0) node[circ]{}
			(1,-2) node[ground]{};
			
			\draw [line width=0.5pt] (0,0) 
			to [short, o-] (1, 0)
			to [capacitor, l={$C_i$}] (3,0)
			to [european resistor, l={$R_d$}] (5,0)
			to [capacitor, l={$C_d$}] (5,-2)
			(5, 0) node[circ]{}
			(5, -2) node[ground]{};
			
			\draw [line width=0.5pt] (5, 0)
			to [short] (6,0)
			(6,0) node[currarrow]{};
			
			\node at (-0.5, 0) {$V_i$};
			\node at (6.5, 0) {$I_f$};

		\end{circuitikz}
	}%
	\caption{Photodiode model}
	\label{fig:test:diode_equiv}
\end{figure}

Using this model, the diode current $I_f$ can be approximated using Equation \eqref{eq:test:i_f}. The approximation is true only if the input impedance $Z_i$ (see Equation \eqref{eq:test:z_i}) of the \gls{tia} is significantly lower than $R_d$ \cite{razavi2019transimpedance}. Keeping this in mind, a resistor six to seven orders of magnitude bigger was selected for the physical circuit.

\begin{equation}\label{eq:test:i_f}
	I_f = \frac{V_i}{R_d}
\end{equation}

\begin{equation}\label{eq:test:z_i}
	Z_i = \frac{R_1}{A_o + 1}
\end{equation}

\begin{figure}[ht]
	\centering
	\begin{tikzpicture}[
		% define box styles
		boxr/.style={rectangle, thick, draw=red!50!black, fill=red!5, minimum size=5mm},
		boxg/.style={rectangle, thick, draw=green!50!black, fill=green!5, minimum size=5mm},
		boxb/.style={rectangle, thick, draw=blue!50!black, fill=blue!5, minimum size=5mm},
		]
		% draw source/scope nodes
		\node[boxg] (fgen)								{Function generator};
		\node[boxg] (scope)		[right=15 mm of fgen] 	{Oscilloscope};
		% draw main system nodes
		\node[boxb] (phm) 	[below=of fgen]			{Photodiode model};
		\node[boxb] (amps) 	[below=of scope]		{Amplifiers};
		
		
		% draw lines
		\draw[->, thick]			(fgen.south)		to node[midway, left]{Signal} 	(phm.north);
		\draw[->, thick] 			(phm.east) 			to node[midway, above]{Current} 			(amps.west);
		\draw[->, thick]			(amps.north) 		to node[midway, right]{Signal} 				(scope.south);
	\end{tikzpicture}
	\caption{\gls{tia} characterization with photodiode model test setup}
	\label{fig:test:tia_dc:setup}
\end{figure}

\subsection{S-parameter characterization}

\subsection{Integration test}
The integration test has a similar test setup to the complete quantum sensing setup. However, the light from the laser directly illuminates the photodiode instead of a diamond sample \gls{nv}. This was done on purpose, because the bigger laser signals provide more clear information on the functioning of the photodetector. When the tests were done, the new version of the laser driver was available and its \gls{pcb} included a laser, as seen in Figure \ref{fig:test:tia_integ:setup}. In addition to the diagram, Figure \ref{fig:test:setupinteg} shows what the physical measurement setup looks like. 

\begin{figure}[ht]
	\centering
	\begin{tikzpicture}[
		% define box styles
		boxr/.style={rectangle, thick, draw=red!50!black, fill=red!5, minimum size=5mm},
		boxg/.style={rectangle, thick, draw=green!50!black, fill=green!5, minimum size=5mm},
		boxb/.style={rectangle, thick, draw=blue!50!black, fill=blue!5, minimum size=5mm},
		]
		% draw source/scope nodes
		\node[boxg] (fgen)								{Function generator};
		\node[boxg] (scope)		[right=10 mm of fgen] 	{Oscilloscope};
		% draw main system nodes
		\node[boxb] (driver) 	[below=of fgen]			{Driver and laser};
		\node[boxb] (detec) 	[below=of scope]		{Photodetector};
		
		
		% draw lines
		\draw[->, thick, dashed] 	(fgen.east) 		to node[midway, above]{\glsfmtshort{ttl}} 	(scope.west);
		\draw[->, thick]			(fgen.south)		to node[midway, left]{\glsfmtshort{ttl}} 	(driver.north);
		\draw[->, thick] 			(driver.east) 		to node[midway, above]{Light} 			(detec.west);
		\draw[->, thick]			(detec.north) 		to node[midway, right]{Signal} 				(scope.south);
	\end{tikzpicture}
	\caption{\gls{tia} integration test setup}
	\label{fig:test:tia_integ:setup}
\end{figure}


\begin{figure}[ht]
	\centering
	\includegraphics[width=0.7\linewidth]{img/setup_integ}
	\caption{Integration test setup}
	\label{fig:test:setupinteg}
\end{figure}



\section{Results}
\subsection{Photodiode-model characterization}
\subsection{S-parameter characterization}
\subsection{\glsfmtshort{tia} integration test}
Testing with 1 and 500 \unit{\hertz} results in the readings shown in Figure \ref{fig:test:integ_v1}. From the data, it can be seen that the photodetector exhibits a lower bandwidth than what was calculated for the \gls{tia}. This is caused by the photovoltaic-mode diode and the high intensity of the light, which lengthen the on time and fall time of the detector.


\begin{figure}[ht]
	\centering
	\begin{subfigure}[1a]{.49\linewidth}
		\centering
		\includegraphics[width=\linewidth]{img/integ_test_1hz}
		\caption{Integration test with the laser pulsing at 1 \unit{\hertz}}
		\label{fig:test:integtest1hz}
	\end{subfigure}
	\hfill
	\begin{subfigure}[1b]{.49\linewidth}
		\centering
		\includegraphics[width=\linewidth]{img/integ_test_500hz}
		\caption{Integration test with the laser pulsing at 500 \unit{\hertz}}
		\label{fig:test:integtest500hz}
	\end{subfigure}
	\caption{Integration test of the first iteration of the photodetector (cutoff frequency $f_c$ = \num{10,77} \unit{\kilo\hertz}) with a focused close-proximity laser}
	\label{fig:test:integ_v1}
\end{figure}


The previous tests were done with the laser placed less than 1 \unit{cm} away from the photodiode. Better measurements can be achieved by moving the laser further away and not aligning the center of the beam with the diode, both of which reduce the light intensity at the diode. This happens because overexposure in photovoltaic mode severely lengthens the rise and fall time of the photodiode, and this effect is further exacerbated by the extremely high sensitivity of the circuit. Figure \ref{fig:test:v1_test} shows how the photodetector performs under the improved conditions. These measurements present a cleaner response at higher frequency, but the low-frequency measurement in Figure \ref{fig:test:v1_test_50} demonstrates the ambient light effect on the precision of the circuit. When there is no direct laser light present, the photocurrent still generated over 200 \unit{\milli\volt} of voltage. This effect is not noticeable in the higher-frequency measurements, however, they demonstrate a significant increase in signal fall time.

\begin{figure}[ht]
	\centering
	\begin{subfigure}[1a]{.49\linewidth}
		\centering
		\includegraphics[width=\linewidth]{img/v1_test_50}
		\caption{Integration test with the laser pulsing at 50 \unit{\hertz}}
		\label{fig:test:v1_test_50}
	\end{subfigure}
	\hfill
	\begin{subfigure}[1b]{.49\linewidth}
		\centering
		\includegraphics[width=\linewidth]{img/v1_test_5k}
		\caption{Integration test with the laser pulsing at 5 \unit{\kilo\hertz}}
		\label{fig:test:v1_test_5k}
	\end{subfigure}
	
	\begin{subfigure}[2a]{.49\linewidth}
		\centering
		\includegraphics[width=\linewidth]{img/v1_test_10k}
		\caption{Integration test with the laser pulsing at 10 \unit{\kilo\hertz}}
		\label{fig:test:v1_test_10k}
	\end{subfigure}
	\hfill
	\begin{subfigure}[2b]{.49\linewidth}
		\centering
		\includegraphics[width=\linewidth]{img/v1_test_20k}
		\caption{Integration test with the laser pulsing at 20 \unit{\kilo\hertz}}
		\label{fig:test:v1_test_20k}
	\end{subfigure}
	
	\caption{Integration test of the first iteration of the photodetector (cutoff frequency $f_c$ = \num{10,77} \unit{\kilo\hertz}) with an unfocused medium-distance laser}
	\label{fig:test:v1_test}
	
\end{figure}

\section{Discussion}
As a result of the tests, the first iteration of the photodetector can be used for \gls{cwodmr} measurements. The protocol can be executed with extremely low bandwidth photodetectors (e.g. 18 \unit{\hertz} \cite{acharya2025compact}). That being said, pulsed protocols cannot be achieved with the first iteration of the circuit due to their high frequency requirements. 

\section{Known limitations}
There are many factors that affect the measurements, but perhaps most prominent are the environmental lighting sources. This includes ambient sunlight, which gets through the blinds of the lab, and reflections from the laser. Both significantly affect the photodetector, especially when there is no focused light present. Reflections, in particular, also contributed to the long falling edges in the integration tests.

Furthermore, the light source in the medium-distance integration tests had to be manually adjusted, meaning that the output signals do not represent the ideal performance. To achieve ideal readings, the laser would have to be affixed and the orientation adjusted, while the intensity at the given distance is measured.


\chapter{Conclusion}
In conclusion, the goal of the project, to enable detection of light signals for quantum sensing, was achieved. While there is still work to be done in order to achieve pulsed quantum protocols, the photodetector platform can be used as a reference for future designs.

According to the introduction in Chapter \ref{chap:intro}, the main purpose of the project was create a photodetector, but integration of the sensing setup was also important. Both were tackled during this project, with the focus being on photodetection. While setup integration also holds importance for the research group, most of the tasks related to it were redistributed among other students. Further refinements to the photodetector and the setup as a whole are possible. Some of the most promising ones are discussed in the recommendations in Chapter \ref{chap:recs}. Complete integration of the whole setup necessitates more work, but is achievable in the near future.


\chapter{Recommendations}\label{chap:recs}
There are several recommendations that are important for future implementations, as they may contribute to the precision, noise-suppression and stability of the photodetector and the quantum sensing setup as a whole. 

Firstly, increasing the bandwidth is the most important future addition. To do that, the amplifier configuration needs to be changed completely. The alternative would be to significantly decrease the transimpedance gain of the \gls{tia}, while increasing the gain of the non-inverting amplifier. This will introduce more noise earlier in the system, which will be amplified by the rest of the circuitry, resulting in a significant output noise increase. 

Secondly, exploring avalanche photodiodes as an alternative to the p-i-n diodes currently in use, might contribute to better measurements. Although they are widely used in low-intensity applications, avalanche diodes are quite expensive. The client wanted to keep the cost low for this project, but in the future more precise pulsed protocol measurements might benefit from the internal amplification and increased precision that are offered by avalanche photodiodes.

%Lastly, if in the future 

