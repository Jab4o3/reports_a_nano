\documentclass{report}
\usepackage[utf8]{inputenc}
%easy styling
\usepackage[a4paper,top=2cm,bottom=2cm,left=3cm,right=3cm,marginparwidth=1.75cm]{geometry}
%IEEE-style paper references
\usepackage[style=ieee]{biblatex}
%more math
\usepackage{amsmath}
\usepackage{amssymb}
%better references
\usepackage{hyperref}
%images
\usepackage{graphicx}

\pagestyle{headings}


\title{Learning report}
\author{Vladislav Serafimov}
%\date{June 2025}


\begin{document}
\maketitle

\chapter{Introduction}
\begin{figure}[h]
    \centering
    \includegraphics[width=.8\textwidth]{vmodel.png}
    \caption{V-model}
    \label{fig:vmodel}
\end{figure}

There are various reasons for using the V-model (Figure \ref{fig:vmodel}) for this project. My previous thesis was based on the waterfall model, but this lead to some outcomes which were not to my liking and which I believe the V-model addresses more than adequately. 

As an extension of the waterfall methodology, the V-model offers a linear approach to problems, which suits hardware projects particularly well. Consequently, both models are reliant on well-established requirements and solid planning. Testing is also an important differentiating aspect of the models. Unlike more software-oriented methodologies, the two offer the more straightforward approach of sequential testing, instead of more convoluted alternatives like iterative testing. 

While similar, there are differences that make the V-model more suitable for the project. One of its main advantages is the much more thorough testing system. In contrast with the waterfall model, which is entirely linear, the V-model provides verification and validation opportunities. The extra focus on testing is exactly what I previously found lacking in the waterfall model. Another benefit is the slightly less rigid structure of the methodology, as this gives more room for further refinement of the goals and the more involved participation of the client.

Although the V-model is better suited to the project, I found my implementation of it lacked in some regards. For example, it asks for well-defined goals from the start, however I could not define the specifics of the assignment early on, in part due to my limited amount of knowledge I had in the field of quantum physics.

%The waterfall model (see Figure \ref{fig:vmodel}) was used for this project. There are various reasons for that, but one of the main ones is that the sequential approach was more suited to the way I wanted to approach the project. Finishing each phase before moving to the next guaranteed the structure my work and gave me intermediate milestones to aim for. Another major reason for picking the waterfall model was the documentation structure. Having every phase flow makes it so that every following document is based on the previous. This aids in creating a more structured workflow. Alternatively, I could have used the Scrum model, which is much more focused on continuous, cyclical improvement. However, using it would mean spending much more time on planning and project management than with the waterfall model. Furthermore, using Scrum would suggest a project which involves implementing and improving a product in an iterative manner. I do not need to build upon already implemented features, which makes this aspect of the model unnecessary and maybe even counterproductive. 


%In spite of how useful the waterfall model was, I did encounter some issues with the way I applied it. Because I have previously only worked on very low-level projects, I did not anticipate the amount of testing I needed to do alongside the implementation and integration phases of the project. If I work on high-level coding project like this one in the future, I will make sure to plan for more testing during the realization phase.


\chapter{Self-evaluation}
% structure: .1 - expectation, .2 - evidence, .3 - evaluation
This chapter discusses what I did well during the project and what I could improve on by relating back to the competences that need to be covered during a graduation assignment. Every competence is discussed and the activities related to it are put forward as evidence of its application.

\section{Analyze}\label{chap:analyze}%done
% The analysis of relationships between processes, products and data flows within the context of the environment
\subsection{Expectation}
Analytical skills have many application and are a vital part of a graduation project. They need to be used to contextualize knowledge, which is the first step to applying it to the particular project. One of the most obvious uses of these skills is when doing the analysis of the requirements, but an analytical approach needs to be taken with the products as well. This also involves relating them back to the goals and requirements, in the reflective manner of the V-Model.

\subsection{Evidence}
The analysis competence played a big part in this thesis. I used it to extract the goals and project boundaries from the requirements, as I have done in all of my previous projects. Furthermore, I had to apply this competence extensively to the photodetector. As there was an existing design that needed improvement, I had to decide which possible upgrades would benefit the client and which were not as useful.

\subsection{Evaluation}
Analyzing various aspects of a project is one of my favorite tasks and I also believe I am quite proficient at it. While I think I delivered on the expectations of the client, I also think there might have been other solutions that, for example, would improve the photodetection circuit even more. A broader analysis would have shown a better overview that showcased more alternatives, but due to the time constraints, I was not eager to spend more time on exploring other options.

\section{Design} \label{chap:design}%done
%Design of an ICT system on the basis of set specifications, and within predefined frameworks.
\subsection{Expectation}
Similar to the analyze competence, design heavily relies on previous knowledge. Where they differ is how and what they are applied to. Designing involves the application of theoretical knowledge, in combination with practical guidelines where necessary, in order to come up with a system and predict its behavior. This competence can be applied to tests and measurements as well, because making a standardized routine uses theoretical and practical knowledge much like system design.

\subsection{Evidence}
As the main goal of the project was to develop a photodetector, I spent a lot of time on designing a solution that combined my knowledge of control systems and analog electronics. The different iterations of the system all had intentional design choices behind them, a detailed explanation of which is presented in the technical report. In addition, all tests were also meticulously designed to measure the metrics that mattered most for the client while also being replicable and following testing protocols.

\subsection{Evaluation}
There are a number of similarities between design and analysis, which is why I think I implemented this competence well. Additionally, design is even more technical, which is why I think it suits me. The main thing I need to improve on, in my opinion, is the scope of considerations for the design. Just like in Chapter \ref{chap:analyze}, I believe I should broaden the number of variables I consider during the design process. Currently, I have a very focused approach, which is good for saving time up front, but results in more system iterations and thus more time spent on a single system over time.

\section{Realize}%done
%Build an ICT system on the basis of a specific design and within a predefined framework
\subsection{Expectation}
Realization is one of the most practical competences. It is exercised extensively during the implementation and integration phases of the project, which require the application of practical skills and a proficiency at using laboratory equipment. Another important aspect of realization is that it follows the previously-established design.

\subsection{Evidence}
The project had a big focus on implementation. I had to make boards I designed, as well as solder ones that were later put to use in the setup. The integration of the photodetector subsystems was also carried out. Afterwards, it was integrated with the rest of the quantum sensing setup. Lastly, the software component was also realized in line with the competence expectations. 

\subsection{Evaluation}
As I mentioned, I worked on realizing both software and hardware solutions for the project, as well as the integration with the other systems of the sensing setup. This process went well, although I saw that there are some areas for improvement. Most importantly, I think I need to invest more effort in honing my practical skills. For example, other students from different projects in the same research group could achieve analogous results to mine, but in a shorter time frame. The theoretical knowledge was not an issue, but applying it was not always as straightforward as I would like it to be. 

\section{Control/Verify}%done
%Management and control of all activities aimed at the process of development and deployment of ICT systems and of ICT service management.
\subsection{Expectation}
Control and verification plays an important role in ensuring the systematic tackling of the project in its various stages. Tight control of activities is needed to mitigate any problems that might arise while working on the project. The verification of results also builds on the control structures by creating a feedback loop, which enables the data-driven management of the project. In particular, this competence is most clearly present in the testing activities.  


\subsection{Evidence}
To maintain structure in my work, I documented every stage of the project while I was executing it. This helped me to systematized the work for every individual phase by coming up with the tasks that needed to be completed at the start of a phase. Another benefit of this method of documentation is that the structure of the report often resembles the structure of the working process. Identically, the testing activities also employed a meticulous methodology based on setting and tracking key performance metrics, which were first declared in the report before being implemented.

\subsection{Evaluation}
Over the numerous projects I have done during my degree, I think I have become proficient at controlling the activities of projects. However, during this project I ran into issues that could not be foreseen or avoided, like for example equipment availability. They showed me that I still need to come up with a systematic way of dealing with unforeseen problems. In terms of testing, I feel like I used a solid methodology, but I think the measurements would have been even better if I accounted for and controlled the environmental factors.


\section{Manage}%done
%Monitoring and organising (time, money, quality, information, organisation), cooperating and communicating.
\subsection{Expectation}
Project management is an extension of the control competence, but on a more general level. Instead of being concerned with the individual stages of the project, this competence deals with organizational matters. Communication with the client and dissemination of information is also a part of this competence.

\subsection{Evidence}
I used several management techniques, among which the daily task sheet was the most important. I used it to keep track of what I did in a day, but I also put project milestones in it. Additionally, I employed version control software for data integrity, but it also doubled as a system to keep track of what was done and what needed to be done. The company coach had access to both in order to ensure complete transparency. Other more direct methods were more commonly used to communicate with the client. In particular, biweekly  meetings were held with other interns from the research group where everyone shared their progress. More informal one-on-one sessions with the company coach were also held.

\subsection{Evaluation}
Manage is one of the competences I have put the most work into improving, because it has been my weakest competence in past projects. There has been a significant change in the way I organize projects now and I believe it is for the better. However, there were still times during this project when I mishandled communication. I also should have accounted for the Christmas holidays and their impact on the scheduling of the implementation, integration and testing stages of the project.

\section{Advise}%done
%Advice regarding the reorganisation of processes and / or data flows. Advise on new ICT systems to be developed or purchased on the basis of an analysis and in consultation with stakeholders.
\subsection{Expectation}
Advice is an important part of a graduation assignment. As the student is supposed to be very knowledgeable on the topic of their thesis, they should also be able to use that knowledge to advise the client on how to continue with the project or what to do after it is done. Additionally, the student should be able to recommend equipment and software that can aid the continuation of the project.

\subsection{Evidence}
During this project, I was often asked for advice on equipment for the setup my project was a part of, but also for the lab in general. I also shared unsolicited advice about what might benefit the setup. At the end of the project, I presented my most important advice in the recommendations chapter.

\subsection{Evaluation}
The advise competence is important and I believe I addressed it sufficiently during this project. In spite of this, I did have some communication issues again. Similarly to the manage competence, I recognize that at times I should have been clearer and more intentional with my message, like when I told the company coach that the bandwidth of the photodetector might not be sufficient for some protocols without explaining the implications in detail.

\section{Research}\label{chap:research}
%Formulating / answering main- and sub-questions, using different sources (i.e. literature, web-pages, publications), validating sources.
\subsection{Expectation}
Research is another important competence, because it accounts for the ability to do academic research responsibly and scientifically. An important part of research is finding adequate information and citing it properly. Furthermore, good research should be systematic. Questions asked at the beginning need to be answered without bias and with a significant scientific backing.

\subsection{Evidence}
This competence played an important role in the design of the photodetector. Because of the very specific requirements for the system, sources were not easy to find. The interdisciplinary nature of the setup meant that I also needed to research the general working of the other systems to create the best detector possible. I also made sure to refer back to the research task when testing the result of the development process.

\subsection{Evaluation}
I am reasonably proficient at doing research, because I have done it for multiple projects, including a research project at the University of Twente. Even then, I still run into some issues occasionally. My biggest struggle when it comes to research is balancing it with the more practical tasks. For example, I spent a longer period of time researching the behavior of transimpedance amplifiers than I did implementing one.


\section{Professionalize}
%Developing and improving skills necessary for the other competences, communicating, reflecting.

\subsection{Expectation}
While professionalization is a separate competence, it shares much of the skills of the other competences. As the name might suggest, this competence focuses on the soft skills necessary to thrive in a professional environment. Communication is perhaps the most important ability relating to this competence, but the student should also be able to reflect on and develop their technical abilities.

\subsection{Evidence}
This project provided a number of opportunities for me to test and train my professional skills. The research group I did my project for included interns in stand-ups, update session and other events, which immersed me in a real work environment. I also presented my own progress in front of my supervisor and other students on multiple occasions, which made me reflect on my work and improve my communication skills.

\subsection{Evaluation}
This is another competence I struggle with to an extent. Presenting is something I have been working on over my last couple of projects, but it still one of my weakest skills. Communicating technical topics in a simple language is something that is particularly difficult for me. 


\chapter{Self-analysis}
%The student should complete a written self-analysis of their own qualities and weaknesses in functioning in a professional context. This does not involve any subject-related aspects of the functioning but rather the professional aspects of the functioning.

Before starting my graduation assignment, I already had some experience with applying different competences to projects, both in academic and professional environments. However, as a thesis assignment, this project had stricter requirements and required better utilization of competences.

I think my skills were sufficient for the needs of the project. I applied my theoretical and practical knowledge, just like I always have. Furthermore, I intentionally worked on my soft skills related to professionalization and management in order to improve them. Additionally, I controlled every stage of the project as well as I could and I applied standardized testing procedures to verify the working of the product.

As I mentioned previously, there are certain shortcomings I observed in my skills. Most notably, my communication skills still need to be improved. I am a technical person naturally, but effectively communicating my results has always been hard for me, especially with people who are not familiar with my project or are not engineers in general. I also struggled with scheduling somewhat, because I did not properly account for the Christmas holidays. Other issues I had and mistakes I made also hindered the project but to a much smaller extent. I should have perhaps broadened the scope of my analysis and research, but I am still not sure if that would have resulted in a better outcome. 


\chapter{Personal development plan}
%The student describes in what direction he/she wishes to develop and how, in a 2 years outlook. In this context the student uses the outcome of the self-reflection, using the subject-related competences and the self-analysis for functioning in a professional context.

This graduation assignment was valuable for my professional development. Although I want to develop in a significantly different professional area in the future, I believe it helped me understand my strengths and weaknesses better. 

The internship I did in the third year of my studies was on the topic embedded audio software and in the future I want to work on something similar. In the next two years I will do a master's degree in embedded systems. I have already applied and made sure that the curricula of the universities I applied to will fulfill my desire for improving my technical skills. All the programs also have projects, which means I will get several opportunities to work on my teamwork and develop better soft skills. Following a master's degree will also allow me to hone my research skills, as all the universities I applied to have a focus on research. In addition to my future studies, I also want to do extracurricular activities that focus on communication, like student councils. This will also help me refine my management skills. After completing a master's study, I can join the workforce, hopefully in my preferred niche. There I will practice the hard and soft skills that are covered by all the competences. Another alternative I am considering is continuing with a more research-oriented profession or study, like a university research position or a doctoral study. Either way, this will contribute significantly to my research skills, as well as the other competences.

%In the relatively short time interval of 2 years, I want to improve many of my professional and technical abilities. However, I will mainly focus on communication. As previously mentioned, I want to get better at communicating with other technical people outside of my field of expertise. Because I need to do a second graduation assignment, I will have another chance to improve during my study. Afterwards, I will work on this skill in the workforce, as a part of a team. I will also work on my technical skills then. As this project was very high-level, I want to improve my technical skills by working a job with a focus on low-level programming or even circuit design. Another possibility I am considering is enrolling in a master's degree in the filed of embedded systems, which will help me gain more knowledge and also some practical skills. Finally, I will also try to create a more structured workflow, to which I am able to adhere in my future projects. I have had some issues while trying to be more consistent and I believe an organized routine will help me with that. 

%professional profile: what do I wanna be known/viewed as professionally


\end{document}