\chapter{Testing and results}\label{chap:conclusion}
%This chapter discusses the results of the tests and what conclusions could be gathered from them. All tests were based on real scenarios, but the performance tests push the platform to its limits by replicating rare and improbable edge cases.

Table \ref{tab:test:tasks} shows all the tasks discussed in Chapter \ref{chap:goals} and their current status. While working on the project, the client decided that purchasing a lock-in amplifier is more worthwhile than implementing \gls{olia}, which is why tasks associated with its implementation are marked as canceled. The items with a reassigned status were given over to two students, who are also involved with the quantum sensing setup. The laser driver was reassigned, because a completely new design was needed, the making of which would require too much time. The pulse sequence script was finished and it was passed on to one of the other students for further refinement and possibly for the implementation of different protocols.

\begin{table}[!ht]
	\centering
	\begin{tabular}{|c||c|c|}
		\hline
		Number 	& Task 									& Status 							\\
		\hline
		1.1 	& Design photodetector 					& Completed							\\
		\hline
		1.2 	& Build \glsfmtshort{olia} 				& Canceled	 						\\
		\hline
		1.3 	& Set up laser driver					& Partially completed; reassigned	\\
		\hline
		2.1 	& Develop data acquisition software 	& Canceled							\\
		\hline
		2.2 	& Program pulse sequences 				& Mostly completed; reassigned	 	\\
		\hline
		3.1 	& Compare lock-in amplifiers			& Canceled  						\\
		\hline
		3.2 	& Test quantum sensing setup 			& In-progress  						\\
		\hline
	\end{tabular}
	\caption{List of tasks and their completion status}
	\label{tab:test:tasks}
\end{table}

%\begin{table}[ht]
%\centering
%\begin{adjustbox}{angle=0}
%\begin{tabular}{|l|l|l|}
%	\hline
%	Number &Task                                                                                                                                                                                     & Status  

%                            \\ \hline
%	1.1    & \begin{tabular}[c]{@{}l@{}}Deploy and configure at least \\ one OpenRemote instance\end{tabular}                                                                                         & \cellcolor[HTML]{34FF34}Completed   \\ \hline
%	1.2    & \begin{tabular}[c]{@{}l@{}}Establish communication to \\ the OpenRemote instance \\ using the HTTP and \\ MQTT protocols\end{tabular}                                                    & \cellcolor[HTML]{34FF34}Completed   \\ \hline
%	2.1    & \begin{tabular}[c]{@{}l@{}}Simulate IoT devices (smart homes)\\ that send and receive concurrent MQTT \\ data to OpenRemote and measure the \\ latency of the transmissions\end{tabular} & \cellcolor[HTML]{34FF34}Completed   \\ \hline
%	2.2    & \begin{tabular}[c]{@{}l@{}}Create a physical IoT device \\ setup using a platform like ESP32 \\ or Arduino and recreate the \\ tests from Goal 1.2\end{tabular}                          & \cellcolor[HTML]{34FF34}Completed   \\ \hline
%	2.3    & Integrate and test OpenRemote's Prophet project                                                                                                                                          & \cellcolor[HTML]{FE0000}Not started \\ \hline
%	2.4    & Test performance and create visualizations                                                                                                                                               & \cellcolor[HTML]{34FF34}Completed   \\ \hline
%	3.1    & Document technical progress                                                                                                                                                              & \cellcolor[HTML]{34FF34}Completed   \\ \hline
%	3.2    & Reflect on personal and professional development                                                                                                                                         & \cellcolor[HTML]{34FF34}Completed   \\ \hline
%	3.3    & Communicate project results                                                                                                                                                              & \cellcolor[HTML]{F8FF00}Ongoing     \\ \hline
%\end{tabular}
%\end{adjustbox}
%\caption{Goal completion}
%\label{tab:goal_com}
%\end{table}



%\section{Test goals and performance metrics}
\section{Test goals}

%\subsection{\glsfmtshort{tia}}
There are two suitable methods of characterizing the \gls{tia}. As the main point of interest is the linearity of the amplifier, the first characterization method is to simulate photodiode readings at different light intensities by modeling it as a current source. By varying the diode model current, either by sweeps or manually, the performance of the physical device can be compared to the simulations.

The second method supplements the results of the first one. It involves measuring the scattering parameters $S_{11}$ and $S_{21}$ and calculating the frequency response of the system. Equation \eqref{eq:test:s11s22} shows the formula used to approximate the transimpedance $Z_t$ for a load resistor $R_l$ \cite{sackinger2017analysis}. This method provides a better overview of the transimpedance over the whole bandwidth of the photodetector.

\begin{equation}\label{eq:test:s11s22}
	Z_t(f) \approx \frac{S_{21}(f)R_l}{1 - S_{11}(f)}
\end{equation}

Neither of these methods uses a real photodiode, which is why an integration test is also necessary. Not only is the integration test needed to evaluate the performance of the diode, but it also gives insight into how the diode interacts with the laser, which will be used to conduct experiments with the sensing setup. Using the laser driver, a square-wave signal is pulsed at different frequencies within the amplifier bandwidth. By comparing the signal output by the photodetector to the signal output by the function generator, the performance of the system can be determined. Although the laser pulsing limits the test to square waves only, the filtering of high-frequency components at near-cutoff frequencies gives a deeper understanding of the behavior of the circuit.

%\begin{table}[ht]
%	\centering
%		\begin{tabular}{|c|c|c|c|}
%			\hline
%			Metric                      & API  & Unit(s) of measurement                                                 & Explanation                                                                               \\ \hline
%			Device provisioning latency & HTTP & seconds (s)                                                            & \begin{tabular}[c]{@{}c@{}}Time between creation \\ request and confirmation\end{tabular} \\ \hline
%			Message loss percentage     & MQTT & percent (\%)                                                           & \begin{tabular}[c]{@{}c@{}}Percent of MQTT \\ messages which are lost\end{tabular}        \\ \hline
%			Resource usage              & -    & \begin{tabular}[c]{@{}c@{}}percent \\ (CPU \%, RAM \%)\end{tabular} & \begin{tabular}[c]{@{}c@{}}CPU and RAM usage \\ when running tests\end{tabular}           \\ \hline
%		\end{tabular}
%	\caption{Metrics of the OpenRemote scalability tests}
%	\label{tab:metrics}
%\end{table}



\section{Test setup}
As different test methods require slightly different physical setups, this section outlines the parts comprising the setups and how they operate. Each setup is dissected separately and the necessary devices and components are shown in diagrams and pictures.

\subsection{Photodiode-model characterization}
Characterizing the system well requires fixed currents, due to the big difference in output voltage small currents differences at the input make. As the diode detects ambient light and cannot be set to one current level reliably, a model needs to be used instead. A voltage source can then be used to simulate the nanoampere photovoltaic current $I_f$ generated by the diode. According to Säckinger \cite{sackinger2017analysis}, the resistor $R_d$ and the capacitor $C_d$ in Figure \ref{fig:test:diode_equiv} are enough to characterize a \gls{tia} circuit. By inputting a voltage at the node $V_i$, $R_d$ functions as a current source, similar to a photodiode. This model assumes the parallel shunt resistance is so high that it can be omitted, while the series resistance can be neglected due to its limited impact on the system. However the diode capacitance is central to \gls{tia} analysis and design, which is why $C_d$ is there to mimic it. Additionally, the matching resistor $R_i$ can be replaced by a coaxial terminator and the capacitor $C_i$ is used to isolate the input from the rest of the circuit.

\begin{figure}[ht]
	\centering
	\resizebox{.5\textwidth}{!}{%
		\begin{circuitikz}
			\tikzstyle{every node}=[font=\normalsize]
			\draw[line width=0.5pt] (1,0) 
			to [european resistor, l={$R_i$}] (1,-2)
			(1, 0) node[circ]{}
			(1,-2) node[ground]{};
			
			\draw [line width=0.5pt] (0,0) 
			to [short, o-] (1, 0)
			to [capacitor, l={$C_i$}] (3,0)
			to [european resistor, l={$R_d$}] (5,0)
			to [capacitor, l={$C_d$}] (5,-2)
			(5, 0) node[circ]{}
			(5, -2) node[ground]{};
			
			\draw [line width=0.5pt] (5, 0)
			to [short] (6,0)
			(6,0) node[currarrow]{};
			
			\node at (-0.5, 0) {$V_i$};
			\node at (6.5, 0) {$I_f$};

		\end{circuitikz}
	}%
	\caption{Photodiode model}
	\label{fig:test:diode_equiv}
\end{figure}

Using this model, the diode current $I_f$ can be approximated using Equation \eqref{eq:test:i_f}. The approximation is true only if the input impedance $Z_i$ (see Equation \eqref{eq:test:z_i}) of the \gls{tia} is significantly lower than $R_d$ \cite{razavi2019transimpedance}. Keeping this in mind, a resistor six to seven orders of magnitude bigger was selected for the physical circuit.

\begin{equation}\label{eq:test:i_f}
	I_f = \frac{V_i}{R_d}
\end{equation}

\begin{equation}\label{eq:test:z_i}
	Z_i = \frac{R_1}{A_o + 1}
\end{equation}

Finally, the full setup diagram can be seen in Figure \ref{fig:test:tia_dc:setup}. Using the function generator feature of the Analog Discovery 2, a voltage can be generated and, using the current through $R_d$, the oscilloscope output can be used to characterize the system.

\begin{figure}[ht]
	\centering
	\begin{tikzpicture}[
		% define box styles
		boxr/.style={rectangle, thick, draw=red!50!black, fill=red!5, minimum size=5mm},
		boxg/.style={rectangle, thick, draw=green!50!black, fill=green!5, minimum size=5mm},
		boxb/.style={rectangle, thick, draw=blue!50!black, fill=blue!5, minimum size=5mm},
		]
		% draw source/scope nodes
		\node[boxg] (fgen)								{Function generator};
		\node[boxg] (scope)		[right=15 mm of fgen] 	{Oscilloscope};
		% draw main system nodes
		\node[boxb] (phm) 	[below=of fgen]			{Photodiode model};
		\node[boxb] (amps) 	[below=of scope]		{Amplifiers};
		
		
		% draw lines
		\draw[->, thick]			(fgen.south)		to node[midway, left]{Signal} 	(phm.north);
		\draw[->, thick] 			(phm.east) 			to node[midway, above]{Current} 			(amps.west);
		\draw[->, thick]			(amps.north) 		to node[midway, right]{Signal} 				(scope.south);
	\end{tikzpicture}
	\caption{\gls{tia} characterization with photodiode model test setup diagram}
	\label{fig:test:tia_dc:setup}
\end{figure}

For the tests, it was decided to test the \gls{dc} characteristics of the amplifier separately from the frequency response. Because of $C_i$, a \gls{dc} voltage will not generate a current.Instead, to verify that the \gls{dc} amplification matches the simulations, a low-frequency sine wave is used and its amplitude is varied, all while the output voltage is measured. The frequency response test measures the response of the system at near-cutoff frequencies, similar to the integration test. The difference is that instead of pulsing the laser, sine waves generate current using the photodiode model, which is then amplified. The change of waveforms is necessary because of the distorting effect of $C_i$ on square waves, especially at lower frequencies.

\begin{figure}[ht]
	\centering
	\includegraphics[width=0.6\linewidth]{img/setup_model_alt}
	\caption{Photodiode model test setup}
	\label{fig:test:setupmodel}
\end{figure}

Figure \ref{fig:test:setupmodel} shows the assembled setup. A 1 \unit{\mega\ohm} resistor was used for $R_d$. Although the diode capacitance $C_d$ is very important for the frequency response, a 27 \unit{\pico\farad} capacitor was used instead of a 25 \unit{\pico\farad} one. This small of a difference should not affect the results much.


\subsection{S-parameter characterization}
Scattering parameters, also known S-parameters, are often used to characterize \gls{rf} circuits and antennas by using a \gls{vna}. For this particular test, the $S_{11}$ and $S_{21}$ parameters are needed to calculate the transimpedance using Equation \eqref{eq:test:s11s22}. In order to measure both, the \gls{vna} needs to be configured as shown in Figure \ref{fig:test:s_params:setup}. $S_{11}$ can be measured with only port 1 of the \gls{vna} being connected to the photodetector. However, port 2 is needed to measure $S_21$ at the output of the photodetector.

\begin{figure}[ht]
	\centering
	\begin{tikzpicture}[
		% define box styles
		boxr/.style={rectangle, thick, draw=red!50!black, fill=red!5, minimum size=5mm},
		boxg/.style={rectangle, thick, draw=green!50!black, fill=green!5, minimum size=5mm},
		boxvna/.style={rectangle, thick, draw=green!50!black, fill=green!5, minimum size=10mm},
		boxb/.style={rectangle, thick, draw=blue!50!black, fill=blue!5, minimum size=5mm},
		]
		% draw source/scope nodes
		\node[boxg, minimum width=30mm] (vna) at (0, 0)							{\glsfmtshort{vna}};
		\node[boxg, minimum width=15mm]	(vna-in)[below left=0mm and -15.25mm of vna]	{Port 1};
		\node[boxg, minimum width=15mm] (vna-out)[below right=0mm and -15.25mm of vna]	{Port 2};
		\node[boxb] (detec) 	[below=10mm of vna]		{Photodetector};
		
		% draw lines
		\draw[thick] (vna-in.west) |- (-20mm, -5mm) |- (detec.west);
		\draw[thick] (vna-out.east) |- (20mm, -5mm) |- (detec.east);
		
		
	\end{tikzpicture}
	\caption{S-parameter characterization of a \gls{tia} using a two-port \gls{vna}}
	\label{fig:test:s_params:setup}
\end{figure}

\subsection{Integration test}
The integration test has a similar test setup to the complete quantum sensing setup. However, the light from the laser directly illuminates the photodiode instead of a diamond sample \gls{nv}. This was done on purpose, because this test layout provides realistic information on the functioning of the photodetector when used in the quantum sensing setup. This test is particularly interesting for pulsed protocols, as a photodetector for them needs to be able to have a well-defined rising edge. A filter was placed in front of the photodetector to remove the green component of the laser and reduce the overall light intensity. However, some of the tests were done without filtering, as with manual adjustment the light intensity at the sensor can be set to an ideal level. When the tests were done, the new version of the laser driver was available and its \gls{pcb} included a laser, as seen in Figure \ref{fig:test:tia_integ:setup}. 

\begin{figure}[ht]
	\centering
	\begin{tikzpicture}[
		% define box styles
		boxr/.style={rectangle, thick, draw=red!50!black, fill=red!5, minimum size=5mm},
		boxg/.style={rectangle, thick, draw=green!50!black, fill=green!5, minimum size=5mm},
		boxb/.style={rectangle, thick, draw=blue!50!black, fill=blue!5, minimum size=5mm},
		]
		% draw source/scope nodes
		\node[boxg] (fgen)								{Function generator};
		\node[boxg] (scope)		[right=10 mm of fgen] 	{Oscilloscope};
		% draw main system nodes
		\node[boxb] (driver) 	[below=of fgen]			{Driver and laser};
		\node[boxb] (detec) 	[below=of scope]		{Photodetector};
		
		
		% draw lines
		\draw[->, thick, dashed] 	(fgen.east) 		to node[midway, above]{\glsfmtshort{ttl}} 	(scope.west);
		\draw[->, thick]			(fgen.south)		to node[midway, left]{\glsfmtshort{ttl}} 	(driver.north);
		\draw[->, thick] 			(driver.east) 		to node[midway, above]{Light} 			(detec.west);
		\draw[->, thick]			(detec.north) 		to node[midway, right]{Signal} 				(scope.south);
	\end{tikzpicture}
	\caption{Integration test setup diagram}
	\label{fig:test:tia_integ:setup}
\end{figure}

In addition to the diagram, Figure \ref{fig:test:setupinteg} shows what the physical measurement setup looks like. Because of reasons discussed in the section containing the results (see Chapter \ref{chap:test:integ_res}), the between the laser and the detector had to be increased and the alignment manually adjusted.


\begin{figure}[ht]
	\centering
	\includegraphics[width=0.5\linewidth]{img/setup_integ_alt}
	\caption{Integration test setup}
	\label{fig:test:setupinteg}
\end{figure}



\section{Results}
This section of the report covers the previously discussed tests and provides images of the measurements. Furthermore, the simulation results are also used for comparison.

\subsection{Photodiode-model characterization}
\begin{figure}[!ht]
	\centering
	\begin{subfigure}[1a]{.49\linewidth}
		\centering
		\includegraphics[width=\linewidth]{img/v1_model_5nA}
		\caption{Model test with the input at 5 \unit{\milli\volt} and a 5 \unit{\nano\ampere} current through $R_d$ (\num{551,89}\unit{\milli\volt} in simulation)}
		\label{fig:test:v1_model_5nA}
	\end{subfigure}
	\hfill
	\begin{subfigure}[1b]{.49\linewidth}
		\centering
		\includegraphics[width=\linewidth]{img/v1_model_15nA}
		\caption{Model test with the input at 15 \unit{\milli\volt} and a 15 \unit{\nano\ampere} current through $R_d$ (\num{1,65}\unit{\volt} in simulation)}
		\label{fig:test:v1_model_15nA}
	\end{subfigure}
	
	\begin{subfigure}[2a]{.49\linewidth}
		\centering
		\includegraphics[width=\linewidth]{img/v1_model_25nA}
		\caption{Model test with the input at 25 \unit{\milli\volt} and a 25 \unit{\nano\ampere} current through $R_d$ (\num{2,75}\unit{\volt} in simulation)}
		\label{fig:test:v1_model_25nA}
	\end{subfigure}
	\hfill
	\begin{subfigure}[2b]{.49\linewidth}
		\centering
		\includegraphics[width=\linewidth]{img/v1_model_40nA}
		\caption{Model test with the input at 40 \unit{\milli\volt} and a 40 \unit{\nano\ampere} current through $R_d$ (\num{4,4}\unit{\volt} in simulation)}
		\label{fig:test:v1_model_40nA}
	\end{subfigure}
	
	\caption{\Gls{dc} characteristics test with a photodiode model ($R_d$ = 1 \unit{\mega\ohm}, $C_d$ = 27 \unit{\pico\farad}) of the first iteration of the photodetector with a 50 \unit{\hertz} input signal}
	\label{fig:test:v1_model}
\end{figure}

Figure \ref{fig:test:v1_model} shows how the first iteration of the photodetector performed under the photodiode model test. In the plots, the amplitude. or half of the peak-to-peak voltage, corresponds to the output signal magnitude that needs to be compared to the simulations. Knowing this, it is clear that the system displays the expected linear behavior\footnote{It should be noted that the $V_{pp}$ values shown in the oscilloscope measurements are approximate. During the tests, it was determined that they can deviate as much as 15\% from the signal value}.

To gain a better understanding of the noise, it was measured with the diode model connected, but turned off. The results are shown in Figure \ref{fig:test:v1modelnoise} and show sporadic changes in voltage with no discernible periodic component, which means the noise is most likely caused by the diode model and significantly amplified by the amplifiers.


\begin{figure}[!ht]
	\centering
	\includegraphics[width=0.5\linewidth]{img/v1_model_noise}
	\caption{Noise of the first iteration when connected to the diode model}
	\label{fig:test:v1modelnoise}
\end{figure}

Figure \ref{fig:test:v1_model_freq} shows that the first version of the photodetector exhibits slightly more attenuation at near-cutoff frequencies than expected. Simulations were conducted and they verified that the main cause was not the slightly larger $C_d$. Its increase only accounts for around 100 \unit{\hertz} of difference in the bandwidth. It is possible that parasitic capacitance, which were not accounted for, are present, which would explain the slight bandwidth reduction.

\begin{figure}[!ht]
	\centering
	\begin{subfigure}[1a]{.49\linewidth}
		\centering
		\includegraphics[width=\linewidth]{img/v1_model_5k}
		\caption{Model test with the input at 5 \unit{\kilo\hertz} (\num{2,5} \unit{\volt} in simulation)}
		\label{fig:test:v1_model_5k}
	\end{subfigure}
	\hfill
	\begin{subfigure}[1b]{.49\linewidth}
		\centering
		\includegraphics[width=\linewidth]{img/v1_model_10k}
		\caption{Model test with the input at 10 \unit{\kilo\hertz} (\num{2,02} \unit{\volt} in simulation)}
		\label{fig:test:v1_model_10k}
	\end{subfigure}
	
	\begin{subfigure}[2a]{.49\linewidth}
		\centering
		\includegraphics[width=\linewidth]{img/v1_model_20k}
		\caption{Model test with the input at 20 \unit{\kilo\hertz} (\num{1,29} \unit{\volt} in simulation)}
		\label{fig:test:v1_model_20k}
	\end{subfigure}
	
	\caption{Near-cutoff characteristics test with a photodiode model ($R_d$ = 1 \unit{\mega\ohm}, $C_d$ = 27 \unit{\pico\farad}) of the first iteration of the photodetector with a 25 \unit{\nano\ampere} input signal}
	\label{fig:test:v1_model_freq}
\end{figure}

Figure \ref{fig:test:v2_model} shows the results of the \gls{dc} characteristics tests of the second version of the photodetector. Despite the measurements showing results similar to the simulations, the output signal contains significant amounts of noise, which exceed the expected levels. Furthermore, the noise of the system is more pronounced than the first iteration, which was also unexpected.

\begin{figure}[!ht]
	\centering
	\begin{subfigure}[1a]{.49\linewidth}
		\centering
		\includegraphics[width=\linewidth]{img/v2_model_5nА}
		\caption{Model test with the input at 5 \unit{\milli\volt} and a 5 \unit{\nano\ampere} current through $R_d$ (\num{551,89}\unit{\milli\volt} in simulation)}
		\label{fig:test:v2_model_5nA}
	\end{subfigure}
	\hfill
	\begin{subfigure}[1b]{.49\linewidth}
		\centering
		\includegraphics[width=\linewidth]{img/v2_model_15nA}
		\caption{Model test with the input at 15 \unit{\milli\volt} and a 15 \unit{\nano\ampere} current through $R_d$ (\num{1,65}\unit{\volt} in simulation)}
		\label{fig:test:v2_model_15nA}
	\end{subfigure}
	
	\begin{subfigure}[2a]{.49\linewidth}
		\centering
		\includegraphics[width=\linewidth]{img/v2_model_25nА}
		\caption{Model test with the input at 25 \unit{\milli\volt} and a 25 \unit{\nano\ampere} current through $R_d$ (\num{2,75}\unit{\volt} in simulation)}
		\label{fig:test:v2_model_25nA}
	\end{subfigure}
	\hfill
	\begin{subfigure}[2b]{.49\linewidth}
		\centering
		\includegraphics[width=\linewidth]{img/v2_model_40nA}
		\caption{Model test with the input at 40 \unit{\milli\volt} and a 40 \unit{\nano\ampere} current through $R_d$ (\num{4,4}\unit{\volt} in simulation)}
		\label{fig:test:v2_model_40nA}
	\end{subfigure}
	
	\caption{\Gls{dc} characteristics test with a photodiode model ($R_d$ = 1 \unit{\mega\ohm}, $C_d$ = 27 \unit{\pico\farad}) of the second iteration of the photodetector with a 50 \unit{\hertz} input signal}
	\label{fig:test:v2_model}
\end{figure}


The noise, seen in Figure \ref{fig:test:v2modelnoise}, is not dominated by a periodic signal, which means it is generated by environmental sources. Interestingly, the noise floor of the second iteration is higher than the first one. With the improved layout and added ground plane, the second iteration was expected to have better noise characteristics. It is possible that the increased bandwidth contributes to the noise, but its effect should be minimal. While testing, it was also determined that there were issues with the power connection, which contributed significant noise to the signal, as well.


\begin{figure}[!ht]
	\centering
	\includegraphics[width=0.5\linewidth]{img/v2_model_noisе}
	\caption{Noise of the second iteration when connected to the diode model}
	\label{fig:test:v2modelnoise}
\end{figure}

Similar to the first version, this one exhibits more attenuation at near-cutoff frequencies. That being said, there increase in bandwidth can be seen clearly, in spite of the the excessive attenuation. 

\begin{figure}[!ht]
	\centering
	\begin{subfigure}[1a]{.49\linewidth}
		\centering
		\includegraphics[width=\linewidth]{img/v2_model_5k}
		\caption{Model test with the input at 5 \unit{\kilo\hertz} (\num{4.28} \unit{\volt} in simulation)}
		\label{fig:test:v2_model_5k}
	\end{subfigure}
	\hfill
	\begin{subfigure}[1b]{.49\linewidth}
		\centering
		\includegraphics[width=\linewidth]{img/v2_model_10k}
		\caption{Model test with the input at 10 \unit{\kilo\hertz} (\num{3,96} \unit{\volt} in simulation)}
		\label{fig:test:v2_model_10k}
	\end{subfigure}
	
	\begin{subfigure}[2a]{.49\linewidth}
		\centering
		\includegraphics[width=\linewidth]{img/v2_model_20k}
		\caption{Model test with the input at 20 \unit{\kilo\hertz} (\num{3,09} \unit{\volt} in simulation)}
		\label{fig:test:v2_model_20k}
	\end{subfigure}
	
	\caption{Near-cutoff characteristics test with a photodiode model ($R_d$ = 1 \unit{\mega\ohm}, $C_d$ = 27 \unit{\pico\farad}) of the second iteration of the photodetector with a 40 \unit{\nano\ampere} input signal}
	\label{fig:test:v2_model_freq}
\end{figure}


The tests of the third iteration of the photodetector, seen in Figure \ref{fig:test:v3_model}, show that the amplifier has a similar performance to the previous iteration. It does have more noise, but with the addition of the charge pump, the increase in noise was expected. Otherwise, both the second and the third version demonstrate the same amplification at all current levels.

\begin{figure}[ht]
	\centering
	\begin{subfigure}[1a]{.49\linewidth}
		\centering
		\includegraphics[width=\linewidth]{img/v3_model_5nA}
		\caption{Model test with the input at 5 \unit{\milli\volt} and a 5 \unit{\nano\ampere} current through $R_d$ (\num{551,89}\unit{\milli\volt} in simulation)}
		\label{fig:test:v3_model_5nA}
	\end{subfigure}
	\hfill
	\begin{subfigure}[1b]{.49\linewidth}
		\centering
		\includegraphics[width=\linewidth]{img/v3_model_15nA}
		\caption{Model test with the input at 15 \unit{\milli\volt} and a 15 \unit{\nano\ampere} current through $R_d$ (\num{1,65}\unit{\volt} in simulation)}
		\label{fig:test:v3_model_15nA}
	\end{subfigure}
	
	\begin{subfigure}[2a]{.49\linewidth}
		\centering
		\includegraphics[width=\linewidth]{img/v3_model_25nA}
		\caption{Model test with the input at 25 \unit{\milli\volt} and a 25 \unit{\nano\ampere} current through $R_d$ (\num{2,75}\unit{\volt} in simulation)}
		\label{fig:test:v3_model_25nA}
	\end{subfigure}
	\hfill
	\begin{subfigure}[2b]{.49\linewidth}
		\centering
		\includegraphics[width=\linewidth]{img/v3_model_40nA}
		\caption{Model test with the input at 40 \unit{\milli\volt} and a 40 \unit{\nano\ampere} current through $R_d$}
		\label{fig:test:v3_model_40nA}
	\end{subfigure}
	
	\caption{\Gls{dc} characteristics test with a photodiode model ($R_d$ = 1 \unit{\mega\ohm}, $C_d$ = 27 \unit{\pico\farad}) of the third iteration of the photodetector with a 50 \unit{\hertz} input signal}
	\label{fig:test:v3_model}
\end{figure}

What is more interesting about the tests of the third version of the photodetector is the noise. Figure \ref{fig:test:v3modelnoise} shows that compared to the previous version, which lacks a charge pump, this iteration has a barely-noticeable periodic signal present even when there is no input voltage from the Analog Discovery. Because the integration tests in Chapter \ref{chap:test:integ_res} demonstrate the noise more clearly, detailed examination of charge pump noise is left for later.

\begin{figure}[!ht]
	\centering
	\includegraphics[width=0.5\linewidth]{img/v3_model_noise}
	\caption{Noise of the second iteration when connected to the diode model}
	\label{fig:test:v3modelnoise}
\end{figure}

As expected the near-cutoff behavior of the third version of the system is almost identical to the second one, as seen in Figure \ref{fig:test:v3_model_freq}. What is more interesting about the plots is that there is a clear periodic signal in addition to the noise, especially in Figures \ref{fig:test:v3_model_5k} and \ref{fig:test:v3_model_10k}.

\begin{figure}[!ht]
	\centering
	\begin{subfigure}[1a]{.49\linewidth}
		\centering
		\includegraphics[width=\linewidth]{img/v3_model_5k}
		\caption{Model test with the input at 5 \unit{\kilo\hertz} (\num{4.28} \unit{\volt} in simulation)}
		\label{fig:test:v3_model_5k}
	\end{subfigure}
	\hfill
	\begin{subfigure}[1b]{.49\linewidth}
		\centering
		\includegraphics[width=\linewidth]{img/v3_model_10k}
		\caption{Model test with the input at 10 \unit{\kilo\hertz} (\num{3,96} \unit{\volt} in simulation)}
		\label{fig:test:v3_model_10k}
	\end{subfigure}
	
	\begin{subfigure}[2a]{.49\linewidth}
		\centering
		\includegraphics[width=\linewidth]{img/v3_model_20k}
		\caption{Model test with the input at 20 \unit{\kilo\hertz} (\num{3,09} \unit{\volt} in simulation)}
		\label{fig:test:v3_model_20k}
	\end{subfigure}
	
	\caption{Near-cutoff characteristics test with a photodiode model ($R_d$ = 1 \unit{\mega\ohm}, $C_d$ = 27 \unit{\pico\farad}) of the third iteration of the photodetector with a 40 \unit{\nano\ampere} input signal}
	\label{fig:test:v3_model_freq}
\end{figure}

\subsection{S-parameter characterization}
Unfortunately, the system could not be characterized by means of S-parameters. This is due to the lack of suitable equipment. The laboratory had a \gls{vna} that could only be used to measure S-parameters in the \gls{rf} range. There was another method which was tried, which involves using the waveform and oscilloscope channels of an Analog Discovery 2 as a \gls{vna}. While this method can cover the operating frequencies of the device, it can only measure the $S_{11}$ parameter of the device, because it functions as a network analyzer with a single port. Lastly, attempts at contacting researchers from the University of Twente for their equipment were made, but they did not respond.

\subsection{Integration test}\label{chap:test:integ_res}
Tests were originally done with the laser placed less than 1 \unit{cm} away from the photodiode. However, better measurements can be achieved by moving the laser further away and not aligning the center of the beam with the diode. Doing both reduces the light intensity at the photodiode, which in turn prevents the sensor from being overexposed. In photovoltaic mode, overexposure severely lengthens the rise and fall time of the photodiode, and this effect is further exacerbated by the extremely high sensitivity of the circuit. Figure \ref{fig:test:v1_test} shows how the photodetector performs under close to ideal conditions. In these tests, the laser intensity should be barely enough for the signal to exhibit supply clipping. 

The measurements of the first iteration of the photodetector present a cleaner response at higher frequency, but the low-frequency measurement in Figure \ref{fig:test:v1_test_50} demonstrates the effect of ambient light, as well as electrical interference, on the circuit. When there is no direct laser light present, the photocurrent still generated over 200 \unit{\milli\volt} of voltage. This effect is not noticeable in the higher-frequency measurements, however, they demonstrate a significant increase in signal rise and fall time.

\begin{figure}[ht]
	\centering
	\begin{subfigure}[1a]{.49\linewidth}
		\centering
		\includegraphics[width=\linewidth]{img/v1_test_50}
		\caption{Integration test with the laser pulsing at 50 \unit{\hertz}}
		\label{fig:test:v1_test_50}
	\end{subfigure}
	\hfill
	\begin{subfigure}[1b]{.49\linewidth}
		\centering
		\includegraphics[width=\linewidth]{img/v1_test_5k}
		\caption{Integration test with the laser pulsing at 5 \unit{\kilo\hertz}}
		\label{fig:test:v1_test_5k}
	\end{subfigure}
	
	\begin{subfigure}[2a]{.49\linewidth}
		\centering
		\includegraphics[width=\linewidth]{img/v1_test_10k}
		\caption{Integration test with the laser pulsing at 10 \unit{\kilo\hertz}}
		\label{fig:test:v1_test_10k}
	\end{subfigure}
	\hfill
	\begin{subfigure}[2b]{.49\linewidth}
		\centering
		\includegraphics[width=\linewidth]{img/v1_test_20k}
		\caption{Integration test with the laser pulsing at 20 \unit{\kilo\hertz}}
		\label{fig:test:v1_test_20k}
	\end{subfigure}
	
	\caption{Integration test of the first iteration of the photodetector (cutoff frequency $f_c$ = \num{10,77} \unit{\kilo\hertz}) with an unfocused medium-distance laser}
	\label{fig:test:v1_test}
\end{figure}

The same test was ran on the second version of the photodetector and the results are shown in Figure \ref{fig:test:v2_test}. In comparison to the first iteration, there is noticeable reduction in low-frequency noise at 50 \unit{\hertz} (Figure \ref{fig:test:v2_test_50}). Furthermore, the high-frequency performance has also improved in the second version. The 5 \unit{\kilo\hertz} measurements in Figure \ref{fig:test:v2_test_5k} show significantly better waveform definition. What is more, at 10 \unit{\kilo\hertz} there is less high-frequency noise, in comparison to before, as can be seen in Figure \ref{fig:test:v2_test_10k}. Similarly, the 20 \unit{\kilo\hertz} signal in Figure \ref{fig:test:v2_test_20k} demonstrates a decrease in noise, but it also shows off the increase in bandwidth. Whereas the first iteration amplifies the signal to 2,1 \unit{\volt} peak-to-peak, the second iteration outputs a 4 \unit{\volt} signal.

\begin{figure}[ht]
	\centering
	\begin{subfigure}[1a]{.49\linewidth}
		\centering
		\includegraphics[width=\linewidth]{img/v2_diode_50}
		\caption{Integration test with the laser pulsing at 50 \unit{\hertz} (5 \unit{\volt} in simulation)}
		\label{fig:test:v2_test_50}
	\end{subfigure}
	\hfill
	\begin{subfigure}[1b]{.49\linewidth}
		\centering
		\includegraphics[width=\linewidth]{img/v2_diode_5k}
		\caption{Integration test with the laser pulsing at 5 \unit{\kilo\hertz} (5 \unit{\volt} in simulation)}
		\label{fig:test:v2_test_5k}
	\end{subfigure}
	
	\begin{subfigure}[2a]{.49\linewidth}
		\centering
		\includegraphics[width=\linewidth]{img/v2_diode_10k}
		\caption{Integration test with the laser pulsing at 10 \unit{\kilo\hertz}}
		\label{fig:test:v2_test_10k}
	\end{subfigure}
	\hfill
	\begin{subfigure}[2b]{.49\linewidth}
		\centering
		\includegraphics[width=\linewidth]{img/v2_diode_20k}
		\caption{Integration test with the laser pulsing at 20 \unit{\kilo\hertz}}
		\label{fig:test:v2_test_20k}
	\end{subfigure}
	
	\caption{Integration test of the second iteration of the photodetector (cutoff frequency $f_c$ = \num{21,167} \unit{\kilo\hertz}) with an unfocused medium-distance laser}
	\label{fig:test:v2_test}
\end{figure}


In Figure \ref{fig:test:v3_test}, the third version of the photodetector demonstrates better high-frequency performance and slightly better low-frequency noise attenuation than the first version, but it still performs worse than the second one. Figure \ref{fig:test:v3_test_50} shows marginally less noise at 50 \unit{\hertz}, in comparison with the first iteration, but there is significantly more high-frequency noise than the second iteration. This can be seen when the laser is off. Figures \ref{fig:test:v3_test_5k} and \ref{fig:test:v3_test_10k} show well-defined pulse edges at 5 \unit[]{\kilo\hertz} and 10 \unit[]{\kilo\hertz} respectively, but, when compared to the 4,8 \unit{\volt} signal of the second iteration, it is clear that the peak-to-peak voltage of 3,7 \unit{\volt} is much smaller than it should be. Lastly, the signal in Figure \ref{fig:test:v3_test_20k} demonstrates similar edge definition to the second iteration. However, the magnitude of the signal is significantly lower at 2,6 \unit[]{\volt} peak-to-peak, in comparison to 4 \unit{\volt}. Lastly, at all test frequencies, there is a significant amount of noise present and its shape, as well as its frequency, suggest the noise originates from the charge pump.


\begin{figure}[ht]
	\centering
	\begin{subfigure}[1a]{.49\linewidth}
		\centering
		\includegraphics[width=\linewidth]{img/v3_diode_50}
		\caption{Integration test with the laser pulsing at 50 \unit{\hertz}}
		\label{fig:test:v3_test_50}
	\end{subfigure}
	\hfill
	\begin{subfigure}[1b]{.49\linewidth}
		\centering
		\includegraphics[width=\linewidth]{img/v3_diode_5k}
		\caption{Integration test with the laser pulsing at 5 \unit{\kilo\hertz}}
		\label{fig:test:v3_test_5k}
	\end{subfigure}
	
	\begin{subfigure}[2a]{.49\linewidth}
		\centering
		\includegraphics[width=\linewidth]{img/v3_diode_10k}
		\caption{Integration test with the laser pulsing at 10 \unit{\kilo\hertz}}
		\label{fig:test:v3_test_10k}
	\end{subfigure}
	\hfill
	\begin{subfigure}[2b]{.49\linewidth}
		\centering
		\includegraphics[width=\linewidth]{img/v3_diode_20k}
		\caption{Integration test with the laser pulsing at 20 \unit{\kilo\hertz}}
		\label{fig:test:v3_test_20k}
	\end{subfigure}
	
	\caption{Integration test of the third iteration of the photodetector (cutoff frequency $f_c$ = \num{21,167} \unit{\kilo\hertz}) with an unfocused medium-distance laser}
	\label{fig:test:v3_test}
\end{figure}

Figure \ref{fig:test:v3ripple} shows the supply noise seen at the output, which reaches up to 400 \unit[]{\milli\volt} and has a frequency of 50 \unit {\kilo\hertz}. Because the signal containing the noise suggests insufficient supply filtering, the capacitances $C_O$, $C_{fly}$ and $C_I$ (see Figure \ref{fig:td:des:chrgpump}) were increased to 10 \unit{\micro\farad}, which is 10 times bigger than what the manufacturer recommends and what was used previously. This resulted in a significant improvement, as can be seen in Figure \ref{fig:test:v3ripplenew}, with the new peak-to-peak maximum being 80 \unit{\milli\volt}.

\begin{figure}[ht]
	\centering
	\begin{subfigure}[1a]{.49\linewidth}
		\centering
		\includegraphics[width=\linewidth]{img/v3_ripple}
		\caption{Supply noise with 1 \unit{\micro\farad} filtering capacitors}
		\label{fig:test:v3ripple}
	\end{subfigure}
	\hfill
	\begin{subfigure}[1b]{.49\linewidth}
		\centering
		\includegraphics[width=\linewidth]{img/v3_ripple_new}
		\caption{Supply noise with 10 \unit{\micro\farad} filtering capacitors}
		\label{fig:test:v3ripplenew}
	\end{subfigure}
	\caption{Noise measurements in minimal-lighting conditions of the third iteration of the photodetector}
	\label{fig:test:v3ripple_both}
\end{figure}

After changing the filtering capacitors, the laser pulsing tests were done again and the results are shown in Figure \ref{fig:test:v3_test_new}. 

\begin{figure}[ht]
	\centering
	\begin{subfigure}[1a]{.49\linewidth}
		\centering
		\includegraphics[width=\linewidth]{img/v3_diode_50_new}
		\caption{Integration test with the laser pulsing at 50 \unit{\hertz}}
		\label{fig:test:v3_test_50_new}
	\end{subfigure}
	\hfill
	\begin{subfigure}[1b]{.49\linewidth}
		\centering
		\includegraphics[width=\linewidth]{img/v3_diode_5k_new}
		\caption{Integration test with the laser pulsing at 5 \unit{\kilo\hertz}}
		\label{fig:test:v3_test_5k_new}
	\end{subfigure}
	
	\begin{subfigure}[2a]{.49\linewidth}
		\centering
		\includegraphics[width=\linewidth]{img/v3_diode_10k_new}
		\caption{Integration test with the laser pulsing at 10 \unit{\kilo\hertz}}
		\label{fig:test:v3_test_10k_new}
	\end{subfigure}
	\hfill
	\begin{subfigure}[2b]{.49\linewidth}
		\centering
		\includegraphics[width=\linewidth]{img/v3_diode_20k_new}
		\caption{Integration test with the laser pulsing at 20 \unit{\kilo\hertz}}
		\label{fig:test:v3_test_20k_new}
	\end{subfigure}
	
	\caption{Integration test of the modified (increased power filtering capacitors) third iteration of the photodetector (cutoff frequency $f_c$ = \num{21,167} \unit{\kilo\hertz}) with an unfocused medium-distance laser}
	\label{fig:test:v3_test_new}
\end{figure}

The results in Figure \ref{fig:test:v3_test_new} show the expected noise improvement. At 50 \unit{\hertz} (see Figure \ref{fig:test:v3_test_50_new}), the low-frequency noise is much more evident than supply noise in the signal when the laser is off. However, the high-frequency noise is still apparent at 5 \unit{\kilo\hertz}. Curiously, at 10 \unit{\kilo\hertz}, the peak-to-peak voltage increased from 3,3 to 4,5 \unit{\volt}. Similarly, at 20 \unit{\kilo\hertz} increased from 2,7 to 3,8 \unit{\volt}, which is still slightly less than the output of the second iteration (4 \unit{\volt}).

Lastly, Figure \ref{fig:test:v4_test_new} shows the results of the integration test of the fourth version of the photodetector. They clearly demonstrate a much lower gain than expected, even when the diode is illuminated directly. Additionally, the output signals are much more distorted than the previous versions and the cutoff frequency is much lower than expected. That being said, the noise levels are still within the expected range.

%todo:finish v4 tests and add pics here

\begin{figure}[ht]
	\centering
	\begin{subfigure}[1a]{.49\linewidth}
		\centering
		\includegraphics[width=\linewidth]{img/v4_diode_50}
		\caption{Integration test with the laser pulsing at 50 \unit{\hertz}}
		\label{fig:test:v4_test_50_new}
	\end{subfigure}
	\hfill
	\begin{subfigure}[2a]{.49\linewidth}
		\centering
		\includegraphics[width=\linewidth]{img/v4_diode_10k}
		\caption{Integration test with the laser pulsing at 10 \unit{\kilo\hertz}}
		\label{fig:test:v4_test_10k_new}
	\end{subfigure}	
	\caption{Integration test of the fourth iteration of the photodetector (cutoff frequency $f_c$ = \num{724,69} \unit{\kilo\hertz}) with an unfocused medium-distance laser}
	\label{fig:test:v4_test_new}
\end{figure}

The most likely explanation for the bad results is the photodiode biasing. While in theory there is not voltage drop over a photodiode when in photovoltaic mode, in practice there are several millivolts of voltage across the diode, which still set the bias of the photodiode. According to Säckinger \cite{sackinger2017analysis}, this voltage greatly diminishes when a bootstrap is implemented and as a result the output current diminishes. This would explain the decrease in amplitude and the distorted shapes of the waveforms. To remedy this, a. shown in Figure \ref{}

\begin{figure}[ht]
	\centering
	\begin{subfigure}[1a]{.49\linewidth}
		\includegraphics[width=\linewidth]{img/v4_diode_biased_noise_lowf}
		\caption{High-frequency view}
		\label{fig:td:v4biasedlowf}
	\end{subfigure}
	\hfill
	\begin{subfigure}[1b]{.49\linewidth}
		\includegraphics[width=\linewidth]{img/v4_diode_biased_noise_highf}
		\caption{Low-frequency view}
		\label{fig:td:v4biasedhighf}
	\end{subfigure}
	\caption{Output signal demonstrating photodiode biasing issues when the fourth iteration of the photodetector uses a biasing circuit}
	\label{fig:td:imp:v4biased}
\end{figure}



\subsection{Additional integration test}
During the integration test it was observed that the green laser has such high intensity, that it easily lead to overexposure. Experiments continued to show high intensity, even after putting an optical filter before the detector. Ideally, no light from the laser should be present at the output when conducting quantum protocols, but real-world setups have small amounts of laser light present in the output signal. Because of this, it was decided that measurements of the laser levels with optical filters were needed for future reference when integrating the setup. 

To simulate a compact integration of the setup, the laser was placed as shown in Figure \ref{fig:test:setupintegfilteralt}. However, as discussed previously, close proximity from the laser and direct illumination of the sensor result in overexposure. In a functional setup, the laser light will be filtered, as a result of which the intensity of the green light will be decreased. 

\begin{figure}
	\centering
	\includegraphics[width=0.5\linewidth]{img/setup_integ_filter_alt}
	\caption{Additional integration test setup}
	\label{fig:test:setupintegfilteralt}
\end{figure}

Figure \ref{fig:test:laser_filter} shows the specifications of the laser used in the integration tests, as well as the optical filter. Because the laser emits light with an intensity of approximately 520 \unit{\nano\meter}, the filter should block the light from it almost completely. To best test the filters, it was decided that the most suitable frequency for the tests is 5 \unit{\kilo\hertz}. It is high enough for the unfiltered sensor to be so overexposed that it only outputs a \gls{dc} voltage, but it is not high enough for near-cutoff behavior to affect the results.

\begin{figure}[ht]
	\centering
	\begin{subfigure}[1a]{.49\linewidth}
		\centering
		\includegraphics[width=\linewidth]{img/filter_spec}
		\caption{Transmission rate of the low-pass optical filter (in red)}
		\label{fig:test:filterspec}
	\end{subfigure}
	\hfill
	\begin{subfigure}[1b]{.49\linewidth}
		\centering
		\includegraphics[width=0.7\linewidth]{img/laser_spectral}
		\caption{Spectral emission of the PLT5 520 laser}
		\label{fig:test:laserspectral}
	\end{subfigure}
	\caption{Spectral characteristics of the components in the setup}
	\label{fig:test:laser_filter}
\end{figure}

Contrary to what was expected, the tests, shown in Figure \ref{fig:test:vdiode_filter} demonstrate poor filtering. When testing with a single filter, there is not only clipping, but also overexposure of the sensor. This is apparent from the signal in Figure \ref{fig:test:v2_diode_1filter}, as it resembles a \gls{pwm} signal with a duty cycle of over 60 \%. As a result, there is also a large overshoot after the falling edge. The results of the test with two stacked filters are better, but not enough for quantum sensing protocols. As can be seen in Figure \ref{fig:test:v2_diode_2filter}, the signal still clips. This makes it impossible to execute any quantum protocol, as no fluorescence can be measured when the amplifier output has hit the maximum. Even worse than the clipping is the fact the photodiode is still slightly overexposed. Despite the reasonable shape of the signal, the width of the pulses is still slightly wider than the ideal. Furthermore, there is a slight overshoot, which can be seen after the falling edge. Ultimately, both setups show suboptimal filtering, which causes signal clipping and overexposure of the sensor.

\begin{figure}[ht]
	\centering
	\begin{subfigure}[1a]{.49\linewidth}
		\centering
		\includegraphics[width=\linewidth]{img/v2_diode_1filter}
		\caption{Test with one filter}
		\label{fig:test:v2_diode_1filter}
	\end{subfigure}
	\hfill
	\begin{subfigure}[1b]{.49\linewidth}
		\centering
		\includegraphics[width=\linewidth]{img/v2_diode_2filter}
		\caption{Test with two filters}
		\label{fig:test:v2_diode_2filter}
	\end{subfigure}
	\caption{Laser transmission tests with the second iteration of the photodetector and laser pulsed at 5 \unit{\kilo\hertz}}
	\label{fig:test:vdiode_filter}
\end{figure}

\section{Discussion}
As a result of the tests, it was concluded that even the first iteration of the photodetector can be used for \gls{cwodmr} measurements. The protocol can be executed with extremely-low-bandwidth photodetectors (e.g. 18 \unit{\hertz} \cite{acharya2025compact}). That being said, pulsed protocols cannot be achieved with the first iteration of the circuit due to their high frequency requirements. Higher frequencies also exacerbate the issue of overexposure of the sensor, which leads to decreased performance. The effect of the laser on the sensor needs to be considered when integrating the complete setup, as it can affect the measurements. There are cancellation techniques to measure only the \gls{nv} luminescence \cite{sewani2020coherent}, but if the photodiode is overexposed or the signal is getting limited by the power rail, the output will not be usable. 


Even though the first iteration has passable characteristics, it is worse than the second version in all regards. The improved \gls{tia} design presents a better frequency response. Across all frequencies, the device shows less attenuation of the harmonics of the square wave, as well as more defined rising and falling edges. In addition, there is less noise present in the signal. This can be attributed to the addition of the ground plane, as well as the better layout of the \gls{pcb}. 

The third iteration offers a solution that is more suitable for a compact setup. The frequency response is the same as the second version, but the onboard power supply inverter generates unwanted noise. As seen in Figure \ref{fig:test:v3ripple_both}, there is a large periodic signal, that looks like a supply ripple. It is also possible that the 1 \unit{\micro\farad} filtering capacitors $C_O$, $C_I$ and $C_{fly}$ do not match the design values, because they were found in a capacitor book in the FabLab. Due to the unknown manufacturer and model, the \gls{esr} of $C_O$ might also be too big. Furthermore, there is a possibility that their capacitance is lower than 1 \unit{\micro\farad}, which would also explain the shape of the ripple in Figure \ref{fig:test:v3ripple}. The presence of 50 \unit{\kilo\hertz} noise after increasing the capacitance, in combination with the large noise amplitude, suggests there is interference with the output of the amplifier.




\section{Known limitations}
The most obvious limitation is the lack of equipment for some tests, which lead to using less appropriate alternatives or having to cancel tests. For example, due to the lack of a suitable \gls{vna}, the only method for circuit characterization was using a diode model. The latter would also benefit from a real nanoampere current source instead of the makeshift source that had to be used. The diode model using the 1 \unit{\mega\ohm} resistor also contributed significant, which was amplified by the \gls{tia}.

There are many factors that affect the measurements, but perhaps most prominent are the environmental lighting sources, at least when it comes to the integration tests. This includes ambient sunlight, which gets through the blinds of the lab, and reflections from the laser. Both significantly affect the photodetector, especially when there is no focused light present. Reflections, in particular, majorly contributed to signal distortion at higher frequencies.

Furthermore, the light source in the medium-distance integration tests had to be manually adjusted, meaning that the output signals do not represent the ideal performance. In fact, the measurements present results with slight power rail clipping, which is unavoidable with the testing setup. To achieve ideal readings, the laser would have to be affixed and the orientation adjusted, while the intensity at the given distance is measured.


\chapter{Conclusion}
In conclusion, the goal of the project, to enable detection of light signals for quantum sensing, was achieved. While there is still work to be done in order to achieve pulsed quantum protocols, the photodetector platform can be used as a reference for future designs.

According to the introduction in Chapter \ref{chap:intro}, the main purpose of the project was create a photodetector, but integration of the sensing setup was also important. Both were tackled during this project, with the focus being on photodetection. While setup integration also holds importance for the research group, most of the tasks related to it were redistributed among other students. Further refinements to the photodetector and the setup as a whole are possible. Some of the most promising ones are discussed in the recommendations in Chapter \ref{chap:recs}. Complete integration of the whole setup necessitates more work, but is achievable in the near future.

The first iteration of the photodetector is suitable for \gls{cwodmr} measurements with the quantum sensing setup. Despite its functionality, it also presents some minor integration problems, such the need for a dual-rail power supply. 

Perhaps most suitable of all for low-noise \gls{cwodmr} measurements, the second version showed promising results and is most suited for a lab setup. Out of all improvements, the noise reduction is the most significant. However, the need for a dual-rail power supply limits the usability in more compact and self-contained future implementations.

Version three of the device addresses the major integration concern of having a simple and small power supply solution. At the same time, it also incorporates the improved system response of the second iteration. Although the \gls{ac} characteristics of the circuit are irrelevant for \gls{cwodmr}, they become the most important aspect of the circuit when discussing pulsed protocols. However, the third version of the photodetector suffers a significant noise problem, caused by the inverting charge pump. Although this is a significant problem, increasing the filtering capacitors showed a major improvement. Even if this version is not recommended for a desktop setup, microcontroller-based portable solutions will benefit from the simplified power supply and will not need an external module to supply -5 \unit{\volt} to the detector.

The fourth photodetector iteration presents a solution specifically designed to handle pulsed protocols. Even though the exact frequency and gain requirements for the circuit are unknown\footnote{Different papers often present different pulsing specifications. In addition, the client wants to adjust the pulsing specifications, so the necessary photodetector bandwidth can only be estimated}, the fourth version should be able to do $T_1$ measurements. Whether higher-frequency pulsed protocols can be achieved is still unknown, as the setup has not been completely integrated.


%todo:finish conclusion based on v4 tests

\chapter{Recommendations}\label{chap:recs}
There are several recommendations that are important for future implementations, as they may contribute to the precision, noise-suppression and stability of the photodetector and the quantum sensing setup as a whole. 

\begin{figure}[!ht]
	\centering
	\includegraphics[width=0.6\linewidth]{matlab/opa818_comparison}
	\caption{Comparison of the Ad795, ADA4637 and OPA818 when configured for 140 \unit{\decibel} of transimpedance gain ($R_1$ = 10 \unit{\mega\ohm}) and a Bessel frequency response}
	\label{fig:recs:opa818comparison}
\end{figure}


Firstly, increasing the photodetector bandwidth should be the primary concern of the people working on the setup, as no version of the detector can measure Rabi oscillations, which occur at over 3 \unit{\mega\hertz}. Version four of the photodetector has a much wider bandwidth, but it is bottlenecked by the photodiode, which is only stable at up to 350 \unit{\kilo\hertz} in the photovoltaic configuration \cite{photodiodephotodiode}. However, biasing a BPW34 diode in photoconductive mode will result in an increase in dark current, usually around \num{2} \unit{\nano\ampere}, but it can rise up to 30 \unit{\nano\ampere}, rendering the actual signal unreadable. Alternatively, the bandwidth can be increased by finding an op-amp with bigger \gls{gbp}. Out of all op-amps that were explored, the OPA818 shows the most promise, which is why its mathematical model was used to compare it to the AD795 and ADA4637. Figure \ref{fig:recs:opa818comparison} showcases the significant bandwidth increase from the ADA4637 to the OPA818. Using the former results in a \num{138,87} \unit{\kilo\hertz} bandwidth, while the latter results in a \num{982,44} \unit{\kilo\hertz} bandwidth. As is usually the case with bigger-\gls{gbp} op-amps, the OPA818 has worse noise parameters (25 \unit[power-half-as-sqrt]{\nano\volt\per\hertz\tothe{0.5}} at 100 \unit{\hertz} as compared to 7 \unit[power-half-as-sqrt]{\nano\volt\per\hertz\tothe{0.5}} at 100 \unit{\hertz}).


\begin{figure}[!ht]
	\centering
	\begin{subfigure}[1a]{.49\linewidth}
		\includegraphics[width=\linewidth]{img/ad4637_noise}
		\caption{Noise spectral density of the ADA4637 (image credit to the ADA4637 documentation \cite{ad2019ada4637})}
		\label{fig:recs:ada4637noise}
	\end{subfigure}
	\hfill
	\begin{subfigure}[1b]{.49\linewidth}
		\includegraphics[width=\linewidth]{img/opa818_noise}
		\caption{Noise spectral density of the OPA818 (image credit to the OPA818 documentation \cite{ti2019opa818})}
		\label{fig:recs:opa818noise}
	\end{subfigure}
	\caption{Noise spectral density of the amplifiers used in the \gls{tia} circuit of the photodetector}
	\label{fig:recs:opampnoisev4}
\end{figure}


In the future, if even greater bandwidth is desired, transistor \glspl{tia} should be investigated. Transistors offer finer control over the parameters of the amplifier they constitute. This is, in part, due to the control over the technology of each transistor, but also because they are lower-level components than op-amps, which allows for more intricate layouts. Several sources on the matter were consulted \cite{sackinger2017analysis, santiago2021a, karimi2017silicon, analui2004bandwidth}, but because of time constraints and equipment availability, an op-amp-based solution was chosen instead.

Secondly, exploring avalanche photodiodes as an alternative to the Si diode (BPW34) currently in use, might contribute to better measurements. Although they are widely used in low-intensity applications, avalanche diodes are quite expensive. The client wanted to keep the cost low for this project, but in the future more precise pulsed protocol measurements might benefit from the internal amplification and increased precision that are offered by avalanche photodiodes. There are also Si alternatives that might provide better performance than the BPW34. InGaAs diodes are often used in telecommunication due to their stable, low-noise performance even at high frequencies. Their peak sensitivity is also usually in the red and infrared area of the light spectrum, which means the green light from the laser will be proportionally smaller than with the BPW34. Similar to avalanche photodiodes, the major drawback of InGaAs diodes is their price, which is much bigger than the BPW34. Some of the diode alternatives (InGaAs, avalanche and Si) that were explored as possible replacements for the BPW34 were the Hamamatsu G6854-01, Hamamatsu G17190-003K, Luna PDU-V104, OSI FCI-InGaAs-1000 and Marktech MTAPD-06-009. A cheaper alternative is the OSRAM SFH 203 FA, which detects less green light, has a lower dark current and has better directional characteristics, possibly leading to less ambient light detection. In spite of the improved characteristics, their impact will most likely be marginal.

Additionally there are ways to improve performance without modifying the circuit. The simplest way to increase the \gls{snr} is to move the setup to the dark room in the laboratory. Doing this will result in less ambient light being picked up by the photodiode. For future implementations that are interested in decreasing the size of the setup, an enclosure should be designed. This way, the main noise source will be the dark current.




